\documentclass [aspectratio=169]{beamer}
\usetheme{Boadilla}
\usepackage{textpos} % package for the positioning
\usepackage[]{graphicx}
\usepackage{graphicx}
\usepackage{float}
\usepackage{hyperref}
\usepackage{caption}
\usepackage{subcaption}
\usepackage{algorithm,algpseudocode}
\usepackage[export]{adjustbox}
\usepackage{tikz}
\usetikzlibrary{positioning}
\usetikzlibrary{arrows, shapes, decorations, automata, backgrounds, fit, petri, calc}
\setbeamertemplate{itemize items}[circle]
\setbeamertemplate{enumerate items}[circle]

\newcommand*{\logofont}{\fontfamily{phv}\selectfont}
\definecolor{uoftblue}{RGB}{0,42,92} % official blue color for uoft
\definecolor{deptgreen}{RGB}{114,192,148} 
\definecolor{deptoran}{RGB}{252,103,63} 

\newcommand{\Z}{{\mathbb{Z}}}



\beamertemplatenavigationsymbolsempty

% block
% example block
% alert block


\title[]{Day 1: Proofs \\ {\large Operational math bootcamp}\\ \includegraphics[width=8cm]{dept_logo.png}\vspace{-1em}}
\author[]{Emma Kroell}
\institute[]{University of Toronto}
\date{\today}

% set color
\setbeamercolor{title in head/foot}{bg=white}
\setbeamercolor{author in head/foot}{bg=white}
\setbeamercolor{date in head/foot}{fg=uoftblue}
\setbeamercolor{date in head/foot}{bg=white}
\setbeamercolor{title}{fg=uoftblue}
\setbeamerfont{title}{series=\bfseries}
\setbeamercolor{frametitle}{fg=uoftblue}
\setbeamerfont{frametitle}{series=\bfseries}
\setbeamercolor*{item}{fg=uoftblue}
\setbeamercolor{block title}{bg=uoftblue}
\setbeamercolor{block title}{fg=white}
\setbeamercolor{block body}{bg=uoftblue!9!white}
\setbeamercolor{block title example}{bg=deptgreen}
\setbeamercolor{block title example}{fg=white}
\setbeamercolor{block body example}{bg=deptgreen!13!white}
\setbeamercolor{block title alerted}{bg=deptoran}
\setbeamercolor{block title alerted}{fg=white}
\setbeamercolor{block body alerted}{bg=deptoran!10!white}


% set logo at non-title pages
\logo{\includegraphics[height=0.8cm]{dept_logo.png}\vspace*{-.045\paperheight}\hspace*{.78\paperwidth}}

% set margin
\setbeamersize{text margin left=10mm,text margin right=10mm}

\begin{document}
{
\setbeamertemplate{logo}{}
\begin{frame}
    %\vspace{0.5in}
    \titlepage
    %\begin{textblock*}{10cm}(3.5cm,-7.5cm)
      %  \includegraphics[width=8cm]{dept_logo.png}
    %\end{textblock*}
\end{frame}
}

\begin{frame}{Outline}
    \begin{itemize}
    	\item Logic
        \item Review of Proof Techniques
        \item Examples
    \end{itemize}
\end{frame}


\begin{frame}{Propositional logic}{}
{\bf Propositions} are statements that could be true or false. They have a corresponding {\bf truth value}. \\

\vspace{1em}

ex. ``$n$ is odd'' and ``$n$ is divisible by 2'' are propositions . Let's call them $P$ and $Q$. \\

Whether they are true or not depends on what $n$ is. \\

\vspace{1em}
\pause

We can  negate statements: $\neg P$ is the statement ``$n$ is not odd''

\pause
\vspace{1em}
 We can combine statements: 
 \begin{itemize}
 \item $P \wedge Q$ is the statement ``$n$ is odd and $n$ is divisible by 2''.
 \item $P \vee Q$ is the statement ``$n$ is odd or $n$ is divisible by 2''. We always assume the inclusive or unless specifically stated otherwise.
\end{itemize}
\end{frame}

\begin{frame}\frametitle{Examples}
    \begin{columns}
        \column{.5\textwidth}
        \begin{center}
          \begin{tabular}{|c|c|}
\hline
    Symbol & Meaning  \\
    \hline
     Capital letters & propositions  \\
     $\implies$ & implies \\
     $\wedge$ & and \\
     $\vee$ & inclusive or \\
     $\neg$ & not \\
     \hline
\end{tabular}
\end{center}
        \column{.5\textwidth}
          \begin{itemize}
              \item If it's not raining, I won't bring my umbrella.
              \item I'm a banana or Toronto is in Canada.
              \item If I pass this exam, I'll be both happy and surprised.
          \end{itemize}
      \end{columns}
\end{frame}

\begin{frame}{Truth values}

\begin{exampleblock}{Example}
If it is snowing, then it is cold out. \\
It is snowing. \\
Therefore, it is cold out.  
\end{exampleblock}

Write this using propositional logic: \\

\pause
\begin{center}
$P \implies Q$ \\
$P$ \\
Conclusion: $Q$ \\
\end{center}

\vspace{1em}
How do we know if this statement is true or not?
\end{frame}



\begin{frame}{Truth table}
    \begin{columns}
        \column{.5\textwidth}
        \begin{center}
If it is snowing, then it is cold out. \\
It is snowing. \\
Therefore, it is cold out. 
\end{center}
\pause
        \column{.5\textwidth}
        \begin{center}
        $P \implies Q$ \\
        \vspace{1.5em}
        \begin{tabular}{|c|c| c|}
\hline
     $P$& $Q$ &  $P \implies Q$ \\ \hline
     T& T & T \\ \hline
     T & F & F \\ \hline
     F & T & T \\ \hline
     F & F & T \\ \hline
\end{tabular}
\end{center}
\end{columns}
\end{frame}


\begin{frame}{Logical equivalence}
    \begin{columns}
        \column{.5\textwidth}
        \begin{center}
        $P \implies Q$ \\
        \vspace{1.5em}
\begin{tabular}{|c|c| c|}
\hline
     $P$& $Q$ &  $P \implies Q$ \\ \hline
     T& T & T \\ \hline
     T & F & F \\ \hline
     F & T & T \\ \hline
     F & F & T \\ \hline
\end{tabular}
\end{center}
\pause
        \column{.5\textwidth}
        \begin{center}
        $\neg P \vee Q$ \\
        \vspace{1.5em}
        \begin{tabular}{|c | c | c | c|}
\hline
     $P$& $Q$ & $\neg P$ & $\neg P \vee Q$  \\ \hline
     T& T & F & T \\ \hline
     T & F & F & F \\ \hline
     F & T &  T &T \\ \hline
     F & F & T & T \\ \hline
\end{tabular}
\end{center}
\end{columns}
\vspace{2em}
\pause
\centering
What is $\neg (P \implies Q)$?
\end{frame}


\begin{frame}{Types of proof}
\begin{itemize}
	\item Direct
	\item Contradiction
	\item Contrapositive
	\item Induction
\end{itemize}
\end{frame}

\begin{frame}{Direct Proof}

{\bf Approach:} Use the definition and known results. \\
\vspace{1em}

\large{\bf Example}

\begin{exampleblock}{Claim}
The product of an even number with another integer is even.
\end{exampleblock}

\vspace{1em}
Approach: use the definition of even.


%We say that an integer $n$ is {\bf even} if there exists another integer $j$ such that $n=2j$.


\end{frame}

\begin{frame}{Direct Proof}

\begin{exampleblock}{Claim}
The product of an even number with another integer is even.
\end{exampleblock}


\begin{alertblock}{Definition}
We say that an integer $n$ is {\bf even} if there exists another integer $j$ such that $n=2j$. \\
We say that an integer $n$ is {\bf odd} if there exists another integer $j$ such that $n=2j+1$.
\end{alertblock}

\begin{proof}
Let $n, m \in \Z$, with $n$ even. By definition, there $\exists$ $j \in \Z$ such that $n = 2j$. Then 
$$ n m  =  (2 j) m = 2 (j m)$$
Therefore $n m$ is even by definition. 
\end{proof}

\end{frame}




\begin{frame}{}
\begin{exampleblock}{Claim}
If an integer squared is even, then the integer is itself even.
\end{exampleblock}

\vspace{1em}

How would you approach this proof?

\end{frame}



\begin{frame}{Proof by contrapositive}
    \begin{columns}
        \column{.5\textwidth}
        \begin{center}
        $P \implies Q$ \\
        \vspace{1.5em}
\begin{tabular}{|c|c| c|}
\hline
     $P$& $Q$ &  $P \implies Q$ \\ \hline
     T& T & T \\ \hline
     T & F & F \\ \hline
     F & T & T \\ \hline
     F & F & T \\ \hline
\end{tabular}
\end{center}
        \column{.5\textwidth}
        \begin{center}
        $\neg P \implies \neg Q$ \\
        \vspace{1.5em}
        \begin{tabular}{|c | c | c |  c | c |}
\hline
     $P$& $Q$ & $\neg P$ &  $\neg Q$ & $\neg Q \implies \neg P$ \\ \hline
     T& T & F & F &  \\ \hline
     T & F & F &  T & \\ \hline
     F & T &  T  & F & \\ \hline
     F & F & T & T &  \\ \hline
\end{tabular}
\end{center}
\end{columns}
\end{frame}

\begin{frame}{Proof by contrapositive}
    \begin{columns}
        \column{.5\textwidth}
        \begin{center}
        $P \implies Q$ \\
        \vspace{1.5em}
\begin{tabular}{|c|c| c|}
\hline
     $P$& $Q$ &  $P \implies Q$ \\ \hline
     T& T & T \\ \hline
     T & F & F \\ \hline
     F & T & T \\ \hline
     F & F & T \\ \hline
\end{tabular}
\end{center}
        \column{.5\textwidth}
        \begin{center}
        $\neg Q \implies \neg P$ \\
        \vspace{1.5em}
        \begin{tabular}{|c | c | c |  c | c |}
\hline
     $P$& $Q$ & $\neg P$ &  $\neg Q$ & $\neg Q \implies \neg P$ \\ \hline
     T& T & F & F & T \\ \hline
     T & F & F &  T & T \\ \hline
     F & T &  T  & F & F \\ \hline
     F & F & T & T & T \\ \hline
\end{tabular}
\end{center}
\end{columns}
\end{frame}

\begin{frame}{Proof by contrapositive}
\begin{exampleblock}{Claim}
If an integer squared is even, then the integer is itself even.
\end{exampleblock}

\vspace{1em}

\begin{proof}
We prove the contrapositive. Let $n$ be odd. Then there exists $k \in \Z$ such that $n = 2k + 1$. We compute
$$n^2 = (2k + 1)^2 = 4k^2 + 4k + 1 = 2(2k^2+2k) + 1.$$
Thus $n^2$ is odd.

\end{proof}

\end{frame}


\begin{frame}{Proof by contradiction}
\begin{exampleblock}{Claim}
The sum of a rational number and an irrational number is irrational.
\end{exampleblock}

\begin{proof}
Let $q \in \mathbb{Q}$ and $r \in \mathbb{R} \setminus \mathbb{Q}$.
Suppose in order to derive a contradiction that their sum is rational, i.e. $ r + q = s$ where $s \in \mathbb{Q}$.
But then $r = s - q \in \mathbb{Q}$. Contradiction.
\end{proof}
\end{frame}



\begin{frame}{Summary}

{\bf In sum, to prove $P \implies Q$:} \\

\vspace{1em}


\begin{tabular}{r l}
     Direct proof:  & assume $P$, prove $Q$ \\
     Proof by contrapositive:  & assume $\neg Q$, prove $\neg P$ \\ 
     Proof by contradiction: & assume $P \wedge \neg Q$ and derive something that is impossible \\ 
\end{tabular}

\end{frame}


\begin{frame}{Induction}

\begin{block}{Well-ordering principle for $\mathbb{N}$}
Every nonempty set of natural numbers has a least element.
\end{block}

\begin{block}{Principle of mathematical induction}
Let $n_0$ be a non-negative integer. Suppose $P$ is a property such that 
\begin{enumerate}
\item(base case) $P(n_0)$ is true 
\item (induction step) For every integer $k \geq n_0$, if $P(k)$ is true, then $P(k+1)$ is true.
\end{enumerate}
Then $P(n)$ is true for every integer $n \geq n_0$
\end{block}

Note: Principle of strong mathematical induction: For every integer $k \geq n_0$, if $P(n)$ is true for every $n = n_0, \ldots, k$, then $P(k+1)$ is true.
\end{frame}

\begin{frame}
\begin{exampleblock}{Claim}
$n! > 2^n$ if $n \geq 4$.
\end{exampleblock}


\begin{proof}
We prove this by induction on $n$. \\
{\it Base case:} Let $n = 4$. Then $n! = 4! = 24 > 16 = 2^4$. \\
\pause
{\it Inductive hypothesis:} Suppose for some $k \geq 4$, $k! > 2^k$. \\
\pause
Then
$$(k+1)! = (k+1) k! > (k+1) 2^k > 2 (2^k) = 2^{k+1}.$$
\end{proof}
\end{frame}



\begin{frame}
\begin{exampleblock}{Claim}
Every integer $n \geq 2$ can be written as the product of primes.
\end{exampleblock}

\begin{proof}
We prove this by induction on $n$. \\
{\it Base case:} $n = 2$ is prime. \\
\pause
{\it Inductive hypothesis:} Suppose for some $k \geq 2$ that one can write every integer $n$ such that $2 \leq n \leq k$ as a product of primes. \\
\pause
We must show that we can write $k+1$ as a product of primes. \\
First, if $k+1$ is prime then we are done.  \\
\pause
Otherwise, if $k+1$ is not prime, by definition it can be written as a product of some integers $a$, $b$ such that $1 < a,b < k+1$. 
By the induction hypothesis, $a$ and $b$ can both be written as products of primes, so we are done.
\end{proof}

\end{frame}


\begin{frame}{Exercises}
\begin{enumerate}
\item Prove De Morgan's Laws: $\neg (P \wedge Q) = \neg P \vee \neg Q$ and $\neg (P \vee Q) = \neg P \wedge \neg Q$ .
\item Prove the Fundamental Theorem of Arithmetic, that every integer $n \geq 2$ has a unique prime factorization (i.e. prove that the prime factorization from the last proof is unique).
\end{enumerate}


\end{frame}



\end{document}