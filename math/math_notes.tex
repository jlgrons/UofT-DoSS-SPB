\documentclass{article}
\usepackage[utf8]{inputenc}
\usepackage[margin=1in]{geometry}
\usepackage{hyperref}
\usepackage{setspace}
\pagenumbering{arabic}
\usepackage{graphicx}
\usepackage[dvipsnames]{xcolor}
\usepackage{fancyhdr} 
\usepackage{amsmath, amsfonts, amsthm}
\usepackage{bbm}
\usepackage{nth}
\usepackage{dsfont}



% commenting
\newcommand{\comment}[3]{\textcolor{#1}{\textbf{[#2: }\textit{#3}\textbf{]}}}
\newcommand{\emma}[1]{\comment{purple}{EK}{#1}}
\newcommand{\jesse}[1]{\comment{BurntOrange}{JG}{#1}}
\newcommand{\miaoshiqi}[1]{\comment{ForestGreen}{ML}{#1}}
\newcommand{\siyue}[1]{\comment{blue}{SY}{#1}}

% theorems
\newtheorem{theorem}{Theorem}[section]
\newtheorem{lemma}[theorem]{Lemma}
\newtheorem{definition}[theorem]{Definition}
\newtheorem{proposition}[theorem]{Proposition}
\newtheorem{example}[theorem]{Example}
%theoremstyle{plain} %boldface title, italicized body. Commonly used in theorems, lemmas, corollaries, propositions and conjectures.
%\theoremstyle{definition} %boldface title, Roman body. Commonly used in definitions, conditions, problems and examples
\theoremstyle{remark} %italicized title, Roman body. Commonly used in remarks, notes, annotations, claims, cases, acknowledgments and conclusions.
\newtheorem{exercise}[theorem]{Exercise}
% weird hack to get rid of dot:
\usepackage{xpatch}
\makeatletter
\AtBeginDocument{\xpatchcmd{\@thm}{\thm@headpunct{.}}{\thm@headpunct{}}{}{}}

% lin alg
\newcommand{\bu}{{\mathbf{u}}}
\newcommand{\bv}{{\mathbf{v}}}
\newcommand{\bw}{{\mathbf{w}}}
\newcommand{\zerovec}{{\mathbf{0}}}

% other useful stuff
\newcommand{\Id}{{\mathds{1}}}
\newcommand{\R}{{\mathds{R}}}
\newcommand{\C}{{\mathds{C}}}
\newcommand{\F}{{\mathds{F}}}



\hypersetup{
  colorlinks   = true, %Colours links instead of boxes
  urlcolor     = black, %Colour for external hyperlinks
  linkcolor    = black, %Colour of internal links
  citecolor   = black %Colour of citations
}

\allowdisplaybreaks % fixes align environment weird spacing on page
\setlength{\parindent}{0cm}


%\usepackage[natbib=true, style=vancouver]{biblatex}
 \usepackage[backend= biber, style=vancouver]{biblatex}
\bibliography{references.bib}

\title{Mathematics Bootcamp \\
\vspace{0.5em}
\large Department of Statistical Sciences, University of Toronto}
\author{Emma Kroell}
\date{Last updated: \today}

\begin{document}

\maketitle
\tableofcontents

\newpage
\section*{Preface}
\addcontentsline{toc}{section}{Preface}
These notes were prepared for the inaugaral Department of Statistical Sciences Graduate Student Bootcamp at the University of Toronto, which is to be held in July 2022. 

References are provided for each section. All references are freely available online, though some may require a University of Toronto library log-in to access. 

\newpage
\section{Review of proof techniques with examples from algebra and analysis}

\subsection{Axioms of a field}
\begin{enumerate}
\setlength\itemsep{0.1em}
    \item[(A1)] \textit{Commutativity in addition:} $x + y = y + x$
    \item[(A2)] \textit{Commutativity in multiplication:} $x \times y = y \times x$
    \item[(B1)] \textit{Associativity in addition:} $x + (y + z) = (x + y) + z$ 
    \item[(B2)] \textit{Associativity in multiplication:} $x \times (y\times z) = (x\times y) \times z$ 
    \item[(C)] \textit{Distributivity:} $x \times (y + z) = x \times y + x \times z$
    \item[(D1)]\textit{Existence of a neutral element, addition:} There exists a number 0 such that $x + 0 = x$ for every $x$.
    \item[(D2)] \textit{Existence of a neutral element, multiplication:} There exists a number 1 such that $x \times 1 = x$ for every $x$. 
    \item[(E1)]\textit{Existence of an inverse, addition:} For each number $x$, there exists a number $-x$ such that $x + (-x) = 0$.
     \item[(E2)]\textit{Existence of an inverse, multiplication:} For each number $x \neq 0$, there exists a number $1/x$ such that $x \times 1/x = 1$.
\end{enumerate}

%\vspace{1em}
%
%We also make the following assumptions about the order of the real numbers:
%\begin{enumerate}
%\setlength\itemsep{0.1em}
%    \item[(F)] \textit{Uniqueness of ordering:} For any $x,y \in \R$, only one of the following holds: $x <y$, $x=y$, or $x > y$.
%    \item[(G)] \textit{Transitivity:} If $x <y$ and $y < z$, then $x < z$.
%    \item[(H1)] \textit{Ordering with addition:} For any $x$, if $y < z$ then $x + y < x + z$.
%    \item[(H2)] \textit{Ordering with multiplication:} For any $x>0$, if $y < z$ then $x \times y < x \times z$.
%\end{enumerate}


\emma{This section to be worked on later}


\subsubsection{Exercises}
\begin{enumerate}
\item For any $a,b \neq 0$, $1/(ab) = 1/a \times 1/b$
\item For $a > 0$, $1/(-a) = -1/a$.
\item For $a, b \neq 0$, $1/(a/b) = b/a$
\end{enumerate}

\section{Linear Algebra}

\subsection{Vector spaces}
\subsubsection{Axioms of a vector space}
Let $V$ be a set and let $\mathbb{F}$ be a field.

\begin{definition}
\label{def:vec_space}
We call $V$ a \textbf{vector space} if the following hold: \\
Addition:
\begin{enumerate}
\setlength\itemsep{0.1em}
    \item[(A)] \textit{Commutativity in addition:} $\bu + \bv = \bv + \bu$ for all $\bu, \bv \in V$
    \item[(B)] \textit{Associativity in addition:} $\bu + (\bv + \bw) = (\bu + \bv) + \bw$ for all $\bu, \bv, \bw \in V$
    \item[(C)] \textit{Existence of a neutral element, addition:} There exists a vector $\zerovec$ such that for any $\bv \in V$, $\zerovec + \bv = \bv$
    \item[(D)] \textit{Additive inverse:} For every $\bv \in V$, there exists another vector, which we denote $-\bv$, such that $\bv + (-\bv) = \zerovec$.
\end{enumerate}

Multiplication by a scalar:

\begin{enumerate}
\setlength\itemsep{0.1em}
    \item[(E)] \textit{Existence of a neutral element, multiplication:} For any $\bv \in V$, $1\times \bv = \bv$
    \item[(F)] \textit{Associativity in multiplication:} Let $\alpha, \beta \in \mathbb{F}$. For any $\bv \in V$, $(\alpha \beta) \bv = \alpha (\beta \bv)$ 
\end{enumerate}

Associativity:
\begin{enumerate}
\setlength\itemsep{0.1em}
    \item[(G)] Let $\alpha \in \mathbb{F}, \bu, \bv \in V$. $\alpha (\bu + \bv) = \alpha \bu + \beta \bv$.
    \item[(H)] Let $\alpha, \beta \in \mathbb{F}, \bv \in V$. $(\alpha + \beta) \bv = \alpha \bv + \beta \bv$.
\end{enumerate}
\end{definition}

Elements of the vector space are called vectors.

Most often we will assume $\mathbb{F} = \mathbb{C}$ or $\R$.

Examples of vector spaces: $\R^n$. $\mathbb{C}^n$, $M_{m \times n}$ (matrices of size $m \times n$), $\mathbb{P}_n$ (polynomials of degree $n$, $p(x) = a_0 + a_1 x + \ldots + a_n x^n$).

\begin{lemma}
\label{lem:neg_vec}
For every $\bv \in V$, we have $-\bv = (-1) \times \bv$.
\end{lemma}
\begin{proof}
Our goal is to show that $(-1) \times \bv$ is the additive inverse of $\bv$.
We show this as follows:
\begin{align*}
    \bv + (-1) \times \bv = \bv \times (1 + (-1)) = \bv \times 0 = 0
\end{align*}
The last step uses exercise \ref{ex:zero}.
\emma{Done by hand in class}
\end{proof}

\subsubsection{Subspaces}

\begin{definition}
A subset $U$ of $V$ is called a \textbf{subspace} of of $V$ if $U$ is also a vector space (using the same addition and scalar multiplication as on $V$).
\end{definition}


\begin{proposition}
A subset $U$ of $V$ is a subspace of $V$ if
and only if $U$ satisfies the following three conditions:
\begin{enumerate}
\item  $\zerovec \in U$
\item Closed under addition: $u,w\in U$ implies $\bu+\bv \in U$
\item Closed under scalar multiplication: $\alpha \in\F$ and $u\in U$
implies $\alpha \bu \in U$
\end{enumerate}
\end{proposition}

\begin{proof}
$\Rightarrow$ If $U$ is a subspace of $V$, then $U$ satisfies these 3 properties by Definition \ref{def:vec_space}.

$\Leftarrow$ Suppose $U$ satisfies the given 3 conditions. Then for any $\bv \in U$, there must exist $-\bv \in U$ by property 3, since $-\bv = (-1) \times \bv$ by Lemma \ref{lem:neg_vec} (property D). Property 1 assures property C. Properties 2 and 3, and the fact that $U \subset V$, assure the remaining properties hold. 

\end{proof}

\emma{Add intersections and unions of subspaces}

\begin{definition}
Suppose $U_{1},...,U_{m}$ are subsets of $V$. The sum
of $U_{1},...,U_{m}$, denoted $U_{1}+...+U_{m}$, is the set of all
possible sums of elements of $U_{1},...,U_{m}.$ More precisely,
\[
U_{1}+...+U_{m}=\{\bu_{1}+...+\bu_{m}:\bu_{1}\in U_{1},...,\bu_{m}\in U_{m}\}
\]
\end{definition}


\begin{proposition}
Suppose $U_{1},...,U_{m}$ are subspaces of $V$. Then
$U_{1}+...+U_{m}$ is the smallest subspace of $V$ containing $U_{1},...,U_{m}$.
\end{proposition}



\subsubsection{Exercises}
\begin{exercise}[1.7 in \cite{linalgwrong}]
\label{ex:zero}
Show that $0 \bv= \zerovec$ for $\bv\in V$.
\end{exercise}
\begin{exercise}[1.B.1 in \cite{linalgright}]
Show that $-(-v)=v$ for $\bv\in V$.
\end{exercise}
\begin{exercise}[1.B.2 in \cite{linalgright}]
Suppose that $\alpha\in\F, \bv\in V$, and $\alpha \bv=0$. Prove that $a=0$
or $v=0$.
\end{exercise}
\begin{exercise}[1.B.4 in \cite{linalgright}]
Why is the empty space not a vector space?
\end{exercise}


Exercise:  Give an example of a nonempty subset $U$ of $\mathbb{R}^{2}$ such that $U$ is closed under scalar
multiplication, but $U$ is not a subspace of $\mathbb{R}$.

Exercise:  A function $f:\mathbb{R} \rightarrow \mathbb{R}$ is called periodic if there exists a positive number such that $f(x)=f(x+p)$ for 
all $x\in \mathbb{R}$.  Is the a set of periodic functions from $\mathbb{R}$ to $\mathbb{R}$ a subspace of $\mathbb{R}^{\mathbb{R}}$?



Exercise:  A function $f:\mathbb{R} \rightarrow \mathbb{R}$ is called odd if
\[
f(-x)=-f(x)
\]

for all $x\in\mathbb{R}$.  Let $U_{e}$ denote the set of real-valued even functions on $\mathbb{R}$ and let $U_{o}$ denote the set
of real-valued odd functions on $\mathbb{R}$. Show that $\mathbb{R}^\mathbb{R}=U_{e} \oplus U_{o}$.


\subsection{Linear (in)dependence and bases}

\begin{definition}
A linear combination of a list $\bv_{1},...,\bv_{n}$ of vectors
in $V$ is a vector of the form 
$$
\alpha_{1}\bv_{1}+...+\alpha_{n}\bv_{n} = \sum_{k=1}^n \alpha_k \bv_k
$$
 where $\alpha_{1},...,\alpha_{m}\in\F$.
\end{definition}

\begin{definition}
The set of all linear combinations of a list of vectors
$v_{1},...,v_{m}$ in $V$ is called the \textbf{span} of $v_{1},...,v_{m}$,
denoted span$\{\bv_{1},...,\bv_{n}\}$. In other words, 
$$
\text{span}\{\bv_{1},...,\bv_{n}\}=\{\alpha_{1}\bv_{1}+...+\alpha_{m}\bv_{n} :\alpha_{1},...,\alpha_{n}\in\F\}
$$
\end{definition}
The span of the empty list is defined to be $\{\zerovec\}$.

\begin{definition}
A system of vectors $\bv_1, \ldots, \bv_n$ is called a basis (for the vector space $V$ ) if any vector $\bv \in V$ admits a unique representation as a linear combination
$$
\bv = \alpha_1 \bv_1 + \ldots + \alpha_n \bv_n = \sum_{k=1}^n \alpha_k \bv_k
$$
\end{definition}

\begin{definition}
The linear combination $\alpha_{1}\bv_{1}+...+\alpha_{n}\bv_{n}$ is called trivial if $\alpha_k = 0$ for every $k$.

\end{definition}

\begin{proposition}
 A system of vectors $\bv_1, \ldots \bv_n \in V$ is a basis if and only if it is linearly independent and complete (generating).
\end{proposition}

\emma{Proof done by hand}

\subsection{Exercises}
From Harvard:
Exercise: Suppose $v_{1},v_{2},v_{3},v_{4}$ (a) spans $V$ and (b)
is linearly independent. Prove that the list 
\[
v_{1}-v_{2},v_{2}-v_{3},v_{3}-v_{4},v_{4}
\]
 also (a) spans $V$ and (b) is linearly independent. 

\vspace{7mm}

Exercise: Suppose $v_{1},...,v_{m}$ is linearly independent in $V$
and $w\in V$. Prove that if $v_{1}+w,...,v_{m}+w$ is linearly dependent,
then $w\in\text{span}(v_{1},...,v_{m})$. 

Exercise: Suppose that $v_{1},...,v_{m}$ is linearly independent
in $V$ and $w\in V$. Show that $v_{1},...,v_{m},w$ is linearly
independent if and only if 
\[
w\notin\text{span}(v_{1},...,v_{m})
\]
on{Exercises}
Exercise: Suppose $v_{1},v_{2},v_{3},v_{4}$ (a) spans $V$ and (b)
is linearly independent. Prove that the list 
\[
v_{1}-v_{2},v_{2}-v_{3},v_{3}-v_{4},v_{4}
\]
 also (a) spans $V$ and (b) is linearly independent. 

Exercise: Suppose $v_{1},...,v_{m}$ is linearly independent in $V$
and $w\in V$. Prove that if $v_{1}+w,...,v_{m}+w$ is linearly dependent,
then $w\in\text{span}(v_{1},...,v_{m})$. 

Exercise: Suppose that $v_{1},...,v_{m}$ is linearly independent
in $V$ and $w\in V$. Show that $v_{1},...,v_{m},w$ is linearly
independent if and only if 
\[
w\notin\text{span}(v_{1},...,v_{m})
\]

\emma{Add a few from books}

\subsection{Linear transformations}

\subsection{Solving linear equations}

\subsection{Determinants}

\subsection{Spectral theory}

\subsection{Inner product spaces}

\subsection{Matrix decomposition}


\subsection{References}
The following texts: \\
Linear Algebra Done Right \cite{linalgright} \\
Linear Algebra Done Wrong \cite{linalgwrong}

\section{Set theory}

\section{Metric spaces}

\section{Topology}

\section{Functions on $\R$}

\section{Multivariable calculus}

\section{Functional analysis}

\section{Complex analysis and Fourier analysis}

\newpage

\printbibliography


\end{document}
