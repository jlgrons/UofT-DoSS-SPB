\documentclass [aspectratio=169]{beamer}
\usetheme{Boadilla}
\usepackage{textpos} % package for the positioning
\usepackage[]{graphicx}
\usepackage{graphicx}
\usepackage{float}
\usepackage{hyperref}
\usepackage{caption}
\usepackage{subcaption}
\usepackage{algorithm,algpseudocode}
\usepackage{amsmath, amsfonts, amsthm, amssymb}
\usepackage{dsfont}
\usepackage[export]{adjustbox}
\usepackage{tikz}
\usetikzlibrary{positioning}
\usetikzlibrary{arrows, shapes, decorations, automata, backgrounds, fit, petri, calc}
\setbeamertemplate{itemize items}[circle]
\setbeamertemplate{enumerate items}[circle]
\usepackage{accents}

\newcommand*{\logofont}{\fontfamily{phv}\selectfont}
\definecolor{uoftblue}{RGB}{0,42,92} % official blue color for uoft
\definecolor{deptgreen}{RGB}{114,192,148} 
\definecolor{deptoran}{RGB}{252,103,63} 

\hypersetup{
  colorlinks   = true, %Colours links instead of boxes
  urlcolor     = uoftblue, %Colour for external hyperlinks
  linkcolor    = black, %Colour of internal links
  citecolor   = black %Colour of citations
}

% lin alg
\newcommand{\bu}{{\mathbf{u}}}
\newcommand{\bv}{{\mathbf{v}}}
\newcommand{\bw}{{\mathbf{w}}}
\newcommand{\bx}{\mathbf{x}}
\newcommand{\zerovec}{{\mathbf{0}}}

% other useful stuff
\newcommand{\Id}{{\mathds{1}}}
\newcommand{\R}{{\mathbb{R}}}
\newcommand{\C}{{\mathbb{C}}}
\newcommand{\Z}{{\mathbb{Z}}}
\newcommand{\N}{{\mathbb{N}}}
\newcommand{\Q}{{\mathbb{Q}}}
\newcommand{\F}{{\mathbb{F}}}
\newcommand{\cL}{{\mathcal{L}}}
\newcommand{\cP}{\mathcal{P}}
\newcommand{\cT}{\mathcal{T}}
\newcommand{\inv}{{-1}}
\newcommand{\interior}{\accentset{\circ}}

\beamertemplatenavigationsymbolsempty

% block
% example block
% alert block
\setbeamercolor{coloredboxstuff}{fg=yellow,bg=uoftblue!10!white}
\AtBeginSection[]{
  \begin{frame}
  \vfill
  \centering
  \begin{beamercolorbox}[sep=8pt,center,shadow=true,rounded=true]{coloredboxstuff}
    \usebeamerfont{title}\insertsectionhead\par%
  \end{beamercolorbox}
  \vfill
  \end{frame}
}



\title[]{Module 5: Metric spaces III \\ {\large Operational math bootcamp}\\ \includegraphics[width=8cm]{dept_logo.png}\vspace{-1em}}
\author[]{Ichiro Hashimoto}
\institute[]{University of Toronto}
\date{July 15, 2024}

% set color
\setbeamercolor{title in head/foot}{bg=white}
\setbeamercolor{author in head/foot}{bg=white}
\setbeamercolor{date in head/foot}{fg=uoftblue}
\setbeamercolor{date in head/foot}{bg=white}
\setbeamercolor{title}{fg=uoftblue}
\setbeamerfont{title}{series=\bfseries}
\setbeamercolor{frametitle}{fg=uoftblue}
\setbeamerfont{frametitle}{series=\bfseries}
\setbeamercolor*{item}{fg=uoftblue}
\setbeamercolor{block title}{bg=uoftblue}
\setbeamercolor{block title}{fg=white}
\setbeamercolor{block body}{bg=uoftblue!9!white}
\setbeamercolor{block title example}{bg=deptgreen}
\setbeamercolor{block title example}{fg=white}
\setbeamercolor{block body example}{bg=deptgreen!13!white}
\setbeamercolor{block title alerted}{bg=deptoran}
\setbeamercolor{block title alerted}{fg=white}
\setbeamercolor{block body alerted}{bg=deptoran!10!white}


% set logo at non-title pages
\logo{\includegraphics[height=0.8cm]{dept_logo.png}\vspace*{-.045\paperheight}\hspace*{.78\paperwidth}}

% set margin
\setbeamersize{text margin left=10mm,text margin right=10mm}

\begin{document}
{
\setbeamertemplate{logo}{}
\begin{frame}
    %\vspace{0.5in}
    \titlepage
    %\begin{textblock*}{10cm}(3.5cm,-7.5cm)
      %  \includegraphics[width=8cm]{dept_logo.png}
    %\end{textblock*}
\end{frame}
}


\begin{frame}{Last time}
Looked at open and closed sets.

\vspace{0.5em}

Discussed sequences, which includes Cauchy sequences and subsequences:
\vspace{0.5em}
	\begin{itemize}
	   \setlength\itemsep{0.6em}
	\item Convergent sequence: $x_n \to x$ $\Leftrightarrow$ $\forall \epsilon > 0 \; \exists \, n_\epsilon \in \N \text{ s.t. } d(x_n,x) < \epsilon \text{ for all } n \geq n_\epsilon$
	\item Cauchy sequence:  $ \forall \epsilon>0 \; \exists \, n_\epsilon \in \N \text{ s.t. } d(x_n,x_m) < \epsilon \text{ for all } n,m \geq n_\epsilon$
	\item Proved that a convergent sequence is Cauchy
	\item Discussed complete metric spaces, where all Cauchy sequences converge (like $\R$ with the usual metric, absolute value)
	\item Proved that a sequence converges to $x$ if and only if all subsequences converge to $x$
    \end{itemize}
\end{frame}

\begin{frame}{Outline for today}
    \begin{itemize}
      \setlength\itemsep{1em}
    	\item Continuity
	\item Equivalent metrics
	\item Density and separability
    \end{itemize}
\end{frame}


\begin{frame}{Continuity}
\begin{definition}
Let $(X,d_X)$ and $(Y,d_Y)$ be metric spaces, let $x_0 \in X$, and let $f:X\to Y$. $f$ is \emph{continuous} at $x_0$ if for every sequence $(x_n)_{n\in\N}$ in $X$ that converges to $x_0$, we have $\lim_{n\to\infty}f(x_n)=f(x_0)$.

\vspace{1em}

We say that $f$ is continuous if it is continuous at every point in $X$.
\end{definition}
\end{frame}


\begin{frame}
\begin{theorem}
\label{thm:cont_equiv}
Let $(X,d_X)$ and $(Y,d_Y)$ be metric spaces, let $x_0 \in X$, and let $f:X\to Y$. The following are equivalent:
\begin{enumerate}
    \item[(i)] $f$ is continuous at $x_0$
    \item[(ii)] for all $\epsilon>0$, there exists $\delta > 0$ such that $d_Y(f(x),f(x_0))) < \epsilon$ for all $x \in X$ with $d_X(x,x_0) < \delta$
    \item[(iii)] for each $\epsilon>0$, there is $\delta > 0$ such that $B_\delta(x_0) \subseteq f^{-1} (B_\epsilon(f(x_0)))$
\end{enumerate}
\end{theorem}
\end{frame}


\begin{frame}
\small
(i) $f$ is continuous at $x_0$ \\
(ii) for all $\epsilon>0$, there exists $\delta > 0$ such that $d_Y(f(x),f(x_0))) < \epsilon$ for all $x \in X$ with $d_X(x,x_0) < \delta$\\
(iii) for each $\epsilon>0$, there is $\delta > 0$ such that $B_\delta(x_0) \subseteq f^{-1} (B_\epsilon(f(x_0)))$

\normalsize
\vspace{0.5em}
\textit{Proof}.
(i) $\Rightarrow$ (ii) 
\vspace{5cm}



\end{frame}


\begin{frame}


(ii) $\Rightarrow$ (iii)

\vspace{1.75cm}
(iii) $\Rightarrow$ (i)

\vspace{4.75cm}

\end{frame}


\begin{frame}
\begin{exampleblock}{Corollary}
Let $(X,d_X)$ and $(Y,d_Y)$ be metric spaces and let $f:X\to Y$. The following are equivalent:
\begin{enumerate}
    \item[(i)] $f$ is continuous
    \item[(ii)] if $U \subseteq Y$ is open, then $f^\inv(U)$ is open
    \item[(iii)] if $F \subseteq Y$ is closed, then $f^\inv(F)$ is closed
\end{enumerate}
\end{exampleblock}

\end{frame}


\begin{frame}
We need the following results about sets and functions: \\
Let $X$ and $Y$ be sets and $f:X \to Y$. Let $A,B \subseteq Y$. Then 
\begin{enumerate}
\item $A \subseteq B$ $\implies$ $f^\inv(A) \subseteq f^\inv(B)$
\item $f^\inv(Y \setminus A) = X \setminus f^\inv(A)$
\end{enumerate}

\vspace{1em}

\textit{Proof}.
Let $(X,d_X)$ and $(Y,d_Y)$ be metric spaces and let $f:X\to Y$.

(i) $\Rightarrow$ (ii): 

\vspace{3cm}




\end{frame}


\begin{frame}

\vspace{2cm}

(ii) $\Rightarrow$ (i)

\vspace{5cm}

\end{frame}

\begin{frame}


(ii) $\Rightarrow$ (iii)

\vspace{4cm}
(iii) $\Rightarrow$ (ii) 
\vspace{2cm}

\end{frame}



\begin{frame}
\begin{definition}
Let $(X,d_X)$ and $(Y,d_Y)$ be metric spaces and let $f:X\to Y$. 
\begin{itemize}
    \item $f$ is \emph{uniformly continuous} if for all $\epsilon>0$, there exists $\delta > 0$ such that for every $x_1,x_2\in X$ with $d_X(x_1,x_2) < \delta$, we have  $d_Y(f(x_1),f(x_2))) < \epsilon$ 
    \item $f$ is \emph{Lipschitz continuous} if there exists a $K > 0$ such that for every $x_1,x_2\in X$ we have  $d_Y(f(x_1),f(x_2))) \leq K d_X(x_1,x_2)$
\end{itemize}
\end{definition}

\begin{exampleblock}{Proposition}
Let $(X,d_X)$ and $(Y,d_Y)$ be metric spaces and let $f:X\to Y$. 
$$f \text{ is Lipschitz continuous } \Rightarrow \text{ f is uniformly continuous } \Rightarrow \text{ f is continuous}$$
\end{exampleblock}
Proof is one of your exercises.
\end{frame}


\begin{frame}{Contraction Mapping Theorem}
\begin{definition}
Let $(X,d)$ be a metric space and let $f:X \to X$. We say that $x^* \in X$ is a \emph{fixed point} of $f$ if $f(x^*) = x^*$.
\end{definition}

\begin{definition}
Let $(X,d)$ be a metric space and let $f:X \to X$. $f$ is a \emph{contraction} if there exists a constant $k \in [0,1)$ such that for all $x,y \in X$, $d(f(x),f(y))) \leq k d(x,y)$.
\end{definition}

Observe that a function is a contraction if and only if it is Lipschitz continuous with constant $K < 1$.



\begin{theorem}[Contraction Mapping Theorem]
Suppose that $f : X \to X$ is a contraction and the metric space $X$ is complete. Then $f$ has a unique fixed point $x^*$.
\end{theorem}
\end{frame}

\begin{frame}

\begin{example}
Let $f:\left[-\frac{1}{3},\frac{1}{3}\right] \to \left[-\frac{1}{3},\frac{1}{3}\right]$ be defined by the mapping $x \mapsto x^2$. Assume we use the standard Euclidean metric, $d(x,y) = |x-y|$. $f$ has a unique fixed point because

\vspace{5cm}
\end{example}

\end{frame}


\begin{frame}{Equivalent metrics}
\begin{definition}[Equivalent metrics]
Two metrics $d_1$ and $d_2$ on a set $X$ are \emph{equivalent} if the identity maps from $(X,d_1)$ to $(X,d_2)$ and from $(X,d_2)$ to $(X,d_1)$ are continuous. 
\end{definition}

\vspace{1em}

\begin{exampleblock}{Proposition}
Two metrics $d_1$, $d_2$ on a set $X$ are equivalent if and only if they have the same open sets or the same closed sets.
\end{exampleblock}
\end{frame}


\begin{frame}

\begin{definition}
Two metrics $d_1$ and $d_2$ on a set $X$ are \emph{strongly equivalent} if for every $x,y\in X$, there exists constants $\alpha>0$ and $\beta>0$ such
\begin{equation*}
    \alpha d_1(x,y) \leq d_2(x,y) \leq \beta d_1(x,y).
\end{equation*}
\end{definition}

If two metrics are strongly equivalent then they are equivalent. The proof of this is one of the exercises. 
\end{frame}

\begin{frame}
\begin{example}
We show that the Euclidean distance (induced by 2-norm) and the metric induced by the $\infty$-norm are equivalent on $\R^n$. 
\vspace{5cm}
\end{example}
Can you think of an example that we've seen of a metric that isn't equivalent to the Euclidean metric?
\end{frame}


\begin{frame}{Density}
\begin{definition}
Let $(X,d)$ be a metric space. A subset $A\subseteq X$ is called \emph{dense} if $\overline{A} = X$.
\end{definition}
\vspace{2.5cm}
Why are dense sets important?
\end{frame}

\begin{frame}{Examples}
1. $(\R,\vert \cdot \vert)$
\vspace{2.5cm}

2. Let $X$ be a set and define $d\colon X \times X \to \mathbb{R}$ by 
    \begin{align*}
        d(x,y) = \begin{cases}
            0, & \; x=y, \\
            1, & \; x\neq y.
        \end{cases}
    \end{align*}
\vspace{2.5cm}
\end{frame}

\begin{frame}
\begin{definition}
A metric space $(X,d)$ is \emph{separable} if it contains a countable dense subset.
\end{definition}

\vspace{1em}
\textbf{Example:}

\end{frame}


\begin{frame}
\begin{example}
Define $\ell_\infty=\{ (x_n)_{n\in \N} \, : x_n \in \R , \; \sup_{n\in \N} \vert x_n \vert <\infty \}$, the space of bounded real valued sequences. Endow $\ell_\infty$ with a metric induced by the supremum norm, namely $d((x_n)_{n\in \N}, (y_n)_{n\in \N}) = \sup_{n\in \N} \vert x_n-y_n\vert$. 
Then $\ell_\infty$ is \textbf{not separable} with respect to the topology induced by this metric. 
\end{example}

\vspace{0.5em}
Note: this proof relies on content from the cardinality section. Specifically, Cantor's theorem gives us that $|A| < |\cP(A)|$ for any set $A$.

\vspace{0.5em}
\textit{Proof}.
\vspace{2.5cm}

\end{frame}


\begin{frame}
\textit{Proof continued}.
\vspace{7cm}

\end{frame}







\begin{frame}{References}

Ji\v{r}\UTF{00ED} Lebl (2022). \textit{Basic Analysis I}. Vol. 1. Introduction to Real Analysis.  \href{ https://www.jirka.org/ra/realanal.pdf}{https://www.jirka.org/ra/realanal.pdf} 

\vspace{1em}


Runde, Volker (2005). \textit{A Taste of Topology}. Universitext.  url:  \href{https://link.springer.com/book/10.1007/0-387-28387-0}{https://link.springer.com/book/10.1007/0-387-28387-0} 


\end{frame}




\end{document}