\documentclass [aspectratio=169]{beamer}
\usetheme{Boadilla}
\usepackage{textpos} % package for the positioning
\usepackage[]{graphicx}
\usepackage{graphicx}
\usepackage{float}
\usepackage{hyperref}
\usepackage{caption}
\usepackage{subcaption}
\usepackage{algorithm,algpseudocode}
\usepackage{amsmath, amsfonts, amsthm, amssymb}
\usepackage{dsfont}
\usepackage[export]{adjustbox}
\usepackage{tikz}
\usetikzlibrary{positioning}
\usetikzlibrary{arrows, shapes, decorations, automata, backgrounds, fit, petri, calc}
\setbeamertemplate{itemize items}[circle]
\setbeamertemplate{enumerate items}[circle]
\usepackage{accents}


\newcommand*{\logofont}{\fontfamily{phv}\selectfont}
\definecolor{uoftblue}{RGB}{0,42,92} % official blue color for uoft
\definecolor{deptgreen}{RGB}{114,192,148} 
\definecolor{deptoran}{RGB}{252,103,63} 

\hypersetup{
  colorlinks   = true, %Colours links instead of boxes
  urlcolor     = uoftblue, %Colour for external hyperlinks
  linkcolor    = black, %Colour of internal links
  citecolor   = black %Colour of citations
}



\beamertemplatenavigationsymbolsempty

\setbeamercolor{coloredboxstuff}{fg=yellow,bg=uoftblue!10!white}
\AtBeginSection[]{
  \begin{frame}
  \vfill
  \centering
  \begin{beamercolorbox}[sep=8pt,center,shadow=true,rounded=true]{coloredboxstuff}
    \usebeamerfont{title}\insertsectionhead\par%
  \end{beamercolorbox}
  \vfill
  \end{frame}
}


% block
% example block
% alert block

% lin alg
\newcommand{\bu}{{\mathbf{u}}}
\newcommand{\bv}{{\mathbf{v}}}
\newcommand{\bw}{{\mathbf{w}}}
\newcommand{\bx}{\mathbf{x}}
\newcommand{\zerovec}{{\mathbf{0}}}

% other useful stuff
\newcommand{\Id}{{\mathds{1}}}
\newcommand{\R}{{\mathbb{R}}}
\newcommand{\C}{{\mathbb{C}}}
\newcommand{\Z}{{\mathbb{Z}}}
\newcommand{\N}{{\mathbb{N}}}
\newcommand{\Q}{{\mathbb{Q}}}
\newcommand{\F}{{\mathbb{F}}}
\newcommand{\cL}{{\mathcal{L}}}
\newcommand{\cP}{\mathcal{P}}
\newcommand{\cT}{\mathcal{T}}
\newcommand{\inv}{{-1}}
\newcommand{\interior}{\accentset{\circ}}



\title[]{Module 6: End of Topology, Start of Linear Algebra \\ {\large Operational math bootcamp}\\ \includegraphics[width=7cm]{dept_logo.png}\vspace{-1em}}
\author[]{Emma Kroell}
\institute[]{University of Toronto}
\date{July 20, 2022}

% set color
\setbeamercolor{title in head/foot}{bg=white}
\setbeamercolor{author in head/foot}{bg=white}
\setbeamercolor{date in head/foot}{fg=uoftblue}
\setbeamercolor{date in head/foot}{bg=white}
\setbeamercolor{title}{fg=uoftblue}
\setbeamerfont{title}{series=\bfseries}
\setbeamercolor{frametitle}{fg=uoftblue}
\setbeamerfont{frametitle}{series=\bfseries}
\setbeamercolor*{item}{fg=uoftblue}
\setbeamercolor{block title}{bg=uoftblue}
\setbeamercolor{block title}{fg=white}
\setbeamercolor{block body}{bg=uoftblue!9!white}
\setbeamercolor{block title example}{bg=deptgreen}
\setbeamercolor{block title example}{fg=white}
\setbeamercolor{block body example}{bg=deptgreen!13!white}
\setbeamercolor{block title alerted}{bg=deptoran}
\setbeamercolor{block title alerted}{fg=white}
\setbeamercolor{block body alerted}{bg=deptoran!10!white}


% set logo at non-title pages
\logo{\includegraphics[height=0.8cm]{dept_logo.png}\vspace*{-.045\paperheight}\hspace*{.78\paperwidth}}

% set margin
\setbeamersize{text margin left=10mm,text margin right=10mm}

\begin{document}
{
\setbeamertemplate{logo}{}
\begin{frame}
    %\vspace{0.5in}
    \titlepage
    %\begin{textblock*}{10cm}(3.5cm,-7.5cm)
      %  \includegraphics[width=8cm]{dept_logo.png}
    %\end{textblock*}
\end{frame}
}

\begin{frame}{Outline}
    \begin{itemize}
      \setlength\itemsep{1em}
    	\item Finish topology
	\begin{itemize}
	\setlength\itemsep{0.3em}
	\item Dense subsets
	\item Compactness
	\item Continuity
	\end{itemize}
	\item Start linear algebra
	\begin{itemize}
	\setlength\itemsep{0.3em}
	\item Vector spaces and subspaces
	\end{itemize}
    \end{itemize}
\end{frame}

\begin{frame}{Last time}

\begin{definition}[Topology]
\footnotesize
Let $\cT\subseteq \cP(X)$. We call $\cT$ a \emph{topology} on $X$ if the following holds:
\begin{itemize}
    \item[(i)] $\emptyset, X \in \cT$
    \item[(ii)] Let $A$ be an arbitrary index set. If $U_\alpha \in \cT$ for $\alpha\in A$, then $\bigcup_{\alpha \in A} U_\alpha \in \cT$ ($\cT$ is closed under taking arbitrary unions)
    \item[(iii)] Let $n \in \N$. If $U_1, \ldots, U_n \in \cT$, then $\bigcap_{i=1}^n U_i \in \cT$ ($\cT$ is closed under taking finite intersections)
\end{itemize}
If $U\in \cT$, we call $U$ \emph{open}. We call $U\subseteq X$ \emph{closed}, if $U^c\in \cT$. We call $(X,\cT)$ a \emph{topological space}.
\end{definition}

\begin{definition}
\footnotesize
Let $(X,\cT)$ be a topological space and let $A\subseteq X$ be any subset.
\begin{itemize}
    \item The \emph{interior} of $A$ is $\interior A := \{a \in A \colon \exists U \in  \cT \text{ s.t. } U\subseteq A \text{ and } a\in U\} $.
    \item The \emph{closure} of $A$ is $\overline{A} := \{x \in X \colon \forall U \in \cT \text{ with } x \in U, U\cap A \neq \emptyset\}$.
    \item The \emph{boundary} of $A$ is $\partial A:= \{x \in X \colon \forall U\in \cT \text{ with } x\in U, \, U \cap A \neq \emptyset \text{ and } U \cap A^c \neq \emptyset \}$.
\end{itemize}
\end{definition}
\end{frame}


\begin{frame}{Density}
\begin{definition}
Let $(X,\cT)$ be a topological space. A subset $A\subseteq X$ is called \emph{dense} if $\overline{A} = X$.
\end{definition}

\vspace{1em}

Using the definition of closure, we see that $A\subseteq X$ is dense if and only if for all non-empty $U\in \cT$, $U\cap A \neq \emptyset$.

\vspace{1em}

\begin{example}
\begin{itemize}
    \item The rationals $\Q$ are dense in the reals $\R$. 
    \item The only dense subset in $(X,\cP(X))$ is $X$ itself.
    \item Any non-empty subset is dense in $(X,\{\emptyset,X\})$.
\end{itemize}
\end{example}

\end{frame}


\begin{frame}{Separability}
\begin{definition}
A topological space $(X,\cT)$ is \emph{separable} if it contains a countable dense subset.
\end{definition}

\vspace{1em}

\begin{example}
\vspace{2cm}
\end{example}
\end{frame}


\begin{frame}{Hausdorff space}
\begin{definition}
A topological space $(X,\cT)$ is called \emph{Hausdorff} if for all $x\neq y\in X$ there exist open sets $U_x,U_y$ with $x\in U_x$ and $y\in U_y$ such that $U_x \cap U_y = \emptyset$.
\end{definition}

So in a Hausdorff space, we can separate any two elements using open sets.
\end{frame}


\begin{frame}

\begin{example}
Let $(X,d)$ be a metric space. Then $(X,\cT_d)$ is Hausdorff, where $\cT_d$ is the topology induced by the metric $d$.
\vspace{4cm}
\end{example}
\end{frame}


\begin{frame}
\begin{example}
Let $X$ be an infinite set and $\cT = \{U\subseteq X \colon U^c \text{ is finite} \}\cup \emptyset$. Then  $(X,\cT)$ is not Hausdorff. 
\vspace{4cm}
\end{example}
\end{frame}


\begin{frame}{Compactness}
\begin{definition}
Let $(X,\cT)$ be a topological space and $K\subseteq X$. 

\vspace{0.5em}

A collection $\{U_i\}_{i\in I} $ of open sets is called \emph{open cover} of $K$ if $K\subseteq \cup_{i\in I} U_i$. 

\vspace{0.5em}

The set $K$ is called \emph{compact} if for all open covers $\{U_i\}_{i\in I}$ there exists a finite subcover, meaning there exists an $n\in \N$ and $\{U_1,\ldots,U_n\}\subseteq \{U_i\}_{i\in I}$ such that $K\subseteq \cup_{i=1}^n U_i$.
\end{definition}
\end{frame}


\begin{frame}

\begin{example}
Let $S\subseteq X$ where $(X,\cT)$ is a topological space. If $S$ is finite, then it is compact. 
\vspace{4cm}
\end{example}
\end{frame}


\begin{frame}
\begin{example}
$(0,1)$ is not compact. 
\vspace{5cm}
\end{example}
\end{frame}


\begin{frame}
\begin{exampleblock}{Proposition}
Let $(X,\cT)$ be a topological space and take a non-empty subset $K\subseteq X$. The following holds:
\begin{enumerate}
    \item If $X$ is compact and $K$ is closed, then $K$ is compact (i.e. closed subsets of compact sets are compact).
    \item If $(X,\cT)$ is Hausdorff, then $K$ being compact implies that $K$ is closed.
\end{enumerate}
\end{exampleblock}
\end{frame}


\begin{frame}
\begin{proof}
(1) If $X$ is compact and $K \subseteq X$ is closed, then $K$ is compact
\vspace{6cm}

\end{proof}
\end{frame}


\begin{frame}
\begin{proof}
(2) If $(X,\cT)$ is Hausdorff, then $K \subseteq X$ compact $\Leftrightarrow$ $K$ is closed.
\vspace{6cm}

\end{proof}
\end{frame}

\begin{frame}
\begin{block}{Proof continued}
\vspace{5cm}

\end{block}
\vspace{1cm}
\end{frame}

\begin{frame}{Compactness on $\R^n$}
\begin{theorem}[Heine-Borel Theorem]
Let $K\subseteq \R^n$. Then $K$ is compact with respect to the topology induced by the Euclidean distance if and only if it is closed and bounded.
\end{theorem}
\end{frame}

\begin{frame}
Just as we had a sequential characterization of the closure of a set in metric spaces, we similarly have a sequential characterization of compactness. 

\begin{theorem}
Let $(X,d)$ be a metric space. Then $K\subset X$ is compact with respect to the metric induced by $d$ if and only if every sequence in $K$ admits a subsequence converging to some point in $K$.
\end{theorem}

\vspace{1em}

A corollary of this statement together with Heine-Borel is the Bolzano-Weierstrass theorem.

\begin{corollary}[Bolzano-Weierstrass]
Any bounded sequence in $\R^n$ has a convergent subsequence.
\end{corollary}

\end{frame}


\begin{frame}{Continuity on a topological space}
\begin{definition}
Let $(X,\cT_X)$ and $(Y,\cT_Y)$ be topological spaces. A map $f\colon X \to Y$ is called \emph{continuous} if for all $U\in \cT_Y$, $f^{-1}(U)\in \cT_X$, i.e. the pre-image of open sets is open. 
\end{definition}

We can also specify continuity at a point $x_0\in X$.

\begin{definition}
Let $(X,\cT_X)$ and $(Y,\cT_Y)$ be topological spaces. A map $f\colon X \to Y$ is called \emph{continuous at $x_0\in X$} if for all $U\in \cT_Y$ with $f(x_0)\in U$, $f^{-1}(U)\in \cT_X$, i.e. the preimage of open sets containing $f(x_0)$ is open (and contains $x_0$).
\end{definition}


\end{frame}


\begin{frame}
\vspace{1em}
\begin{exampleblock}{Proposition}
Let $(X,\cT_X)$ and $(Y, \cT_Y)$ be topological spaces. Suppose $K\subset X$ is compact and let $f\colon K \to Y$ be continuous. Then $f(K)$ is compact. 
\end{exampleblock}

\vspace{1em}

Recall from the set theory section: \\
If $f:X \to Y$:
\begin{enumerate}
\item $A \subseteq B \subseteq Y$ $\Rightarrow$ $f^{-1}(A) \subseteq f^{-1}(B)$ and  $A \subseteq B \subseteq X$ $\Rightarrow$ $f(A) \subseteq f(B)$
\item $f^\inv(\cup_{i \in I}A_i) = \cup_{i \in I}f^\inv(A_i)$, where $A_i \subseteq Y \, \forall i \in I$
\item $f(\cup_{i \in I}A_i) = \cup_{i \in I}f(A_i)$, where $A_i \subseteq X \, \forall i \in I$
\item $A \subseteq X$ $\Rightarrow$  $A \subseteq f^{-1}(f(A))$
\item $B \subseteq Y$ $\Rightarrow$   $f(f^{-1}(B)) \subseteq B$
\end{enumerate}

\end{frame}

\begin{frame}

\begin{block}{Proof.}
\vspace{6cm}
\end{block}
\end{frame}


\begin{frame}
\begin{block}{Proof continued}
\vspace{4.5cm}
\end{block}
\end{frame}

\begin{frame}{Right and left continuous}
Recall: $f\colon \R \to \R$ is continuous at $x_0\in \R$ if for all $\epsilon>0$ there exists a $\delta>0$ such that $\vert x_0-y\vert <\delta$ implies $\vert f(x_0)-f(y)\vert<\epsilon$.

\vspace{1em}

\begin{definition} Let $f\colon \R \to \R$.
\begin{itemize}
    \item $f$ is \emph{left continuous} at $x_0\in \R$ if for all $\epsilon >0$ there exists a $\delta>0$, such $\vert f(x_0)-f(x)\vert<\epsilon$ whenever $ x_0-\delta <x<x_0$.
    \item $f$ is \emph{right continuous} at $x_0\in \R$ if for all $\epsilon >0$ there exists a $\delta>0$, such $\vert f(x_0)-f(x)\vert<\epsilon$ whenever $x_0<x<x_0+\delta$.
\end{itemize}
We say that $f$ is left continuous if it is left continuous at all points in the domain, and similar for right continuous.
\end{definition}




\end{frame}


\begin{frame}
\begin{exampleblock}{Proposition}
 A function $f\colon \R \to \R$ is continuous if and only if it is left and right continuous. 
\end{exampleblock}
\begin{proof}
\vspace{5cm}
\end{proof}
\end{frame}


\begin{frame}
\begin{block}{Proof continued}
\vspace{4.5cm}
\end{block}
\end{frame}


\begin{frame}{Bounded sequences and monotone convergence}
\begin{definition}
Let $(x_n)_{n\in \N}$ be a sequence in $\R$. We call $(x_n)_{n\in \N}$ \emph{bounded} if there exists an $M > 0$ such that $\vert x_n\vert<M$ for all $n\in \N$.
\end{definition}

\vspace{0.5em}

\begin{theorem}[Monotone convergence theorem]
\begin{itemize}
\item[(i)] Suppose $(x_n)_{n\in \N}$ is an increasing sequence, i.e. $x_n\leq x_{n+1}$ for all $n\in \N$, and that it is bounded (above). Then the sequence converges. Furthermore, $\lim_{n\to \infty} x_n = \sup_{n\in \N} x_n$, where $\sup_{n\in \N} x_n := \sup\{x_n \; \colon \; n\in \N\}$.
\item[(ii)] Suppose $(x_n)_{n\in \N}$ is a decreasing sequence, i.e. $x_n\geq x_{n+1}$ for all $n\in \N$, which is bounded (below). Then the sequence converges and $\lim_{n\to \infty} x_n = \inf_{n\in \N} x_n := \inf\{x_n \; \colon \; n\in \N\}$. 
\end{itemize}
\end{theorem}

\end{frame}


\begin{frame}
Convention: $\sup A = \infty$ if $A\subseteq \R$ is not bounded above and $\inf A= -\infty$ if $A$ is not bounded below.

\begin{lemma}
If $A \subseteq B \subseteq \R$ is non-empty, then $\inf A \leq \sup A$, $\sup A \leq \sup B$, and $\inf A \geq \inf B$.
\end{lemma}
The proof of this follows from the definition of greatest lower and least upper bound. 

\end{frame}

\begin{frame}
\begin{definition}
Let $(x_n)_{n\in \N}$ be a sequence in $\R$. We define the \emph{limit superior} of $(x_n)_{n\in \N}$ as 
$$\limsup_{n\to \infty} x_n: = \lim_{n\to \infty} \sup_{k\geq n} x_k.$$ 
Similarly we define  the \emph{limit inferior} of $(x_n)_{n\in \N}$ as 
$$\liminf_{n\to \infty} x_n: = \lim_{n\to \infty} \inf_{k\geq n} x_k.$$
\end{definition}
\vspace{1em}
If the sequence $(x_n)_{n\in \N}$ is not bounded above, then $\limsup_{n\to \infty} x_n = \infty$. Similarly, if the sequence $(x_n)_{n\in \N}$ is not bounded below, then $\liminf_{n\to \infty} x_n = -\infty$. 
\end{frame}

\begin{frame}
\begin{exampleblock}{Proposition}
Let $(x_n)_{n \in \N}$ be a sequence in $\R$.
\begin{itemize} 
\item The sequence of suprema, $s_n = \sup_{k\geq n} x_k$, is decreasing and the sequence of infima, $i_n = \inf_{k\geq n} x_k$, is increasing.
\item The limit superior and the limit inferior of a bounded sequence always exist and are finite.
\end{itemize}
\end{exampleblock}
\begin{proof}
The first part is true by the previous Lemma. The second follows by the Monotone Convergence Theorem.
\end{proof}

\end{frame}

\begin{frame}
\begin{theorem}
Let $(x_n)_{n\in \N}$ be a sequence in $\R$. Then the sequence converges to $x\in \R$ if and only if $\limsup_{n\to \infty} x_n= x =\liminf_{n\to \infty} x_n$.
\end{theorem}
\begin{block}{Proof.}
\vspace{4.5cm}
\end{block}
\end{frame}


\begin{frame}
\begin{block}{Proof continued}
\vspace{6.5cm}
\end{block}
\end{frame}

\begin{frame}
\vspace{2em}
We can extend this easily to a sequence of functions $f_n \colon X \to \R$ as follows:
\vspace{0.5em}

Define $f = \limsup_{n\to \infty } f_n$ to be the function defined pointwise by $f(x) = \limsup_{n\to \infty } (f_n (x)) $ and similar for the limit inferior. 

\vspace{1em}

There also exists a set theoretic version in terms of unions and intersections which you will encounter in probability.

\end{frame}


\section{Linear Algebra}

\begin{frame}
\begin{definition}
We call $V$ a \textbf{vector space} if the following hold: \\

\begin{enumerate}
\setlength\itemsep{0.1em}
    \item[(A)] \textit{Commutativity in addition:} $\bu + \bv = \bv + \bu$ for all $\bu, \bv \in V$
    \item[(B)] \textit{Associativity in addition:} $\bu + (\bv + \bw) = (\bu + \bv) + \bw$ for all $\bu, \bv, \bw \in V$
    \item[(C)] \textit{Existence of a neutral element, addition:} There exists a vector $\zerovec$ such that for any $\bv \in V$, $\zerovec + \bv = \bv$
    \item[(D)] \textit{Additive inverse:} For every $\bv \in V$, there exists another vector, which we denote $-\bv$, such that $\bv + (-\bv) = \zerovec$.
\end{enumerate}


\begin{enumerate}
\setlength\itemsep{0.1em}
    \item[(E)] \textit{Existence of a neutral element, multiplication:} For any $\bv \in V$, $1\times \bv = \bv$
    \item[(F)] \textit{Associativity in multiplication:} Let $\alpha, \beta \in \mathbb{F}$. For any $\bv \in V$, $(\alpha \beta) \bv = \alpha (\beta \bv)$ 
\end{enumerate}

\begin{enumerate}
\setlength\itemsep{0.1em}
    \item[(G)] Let $\alpha \in \mathbb{F}, \bu, \bv \in V$. $\alpha (\bu + \bv) = \alpha \bu + \alpha \bv$.
    \item[(H)] Let $\alpha, \beta \in \mathbb{F}, \bv \in V$. $(\alpha + \beta) \bv = \alpha \bv + \beta \bv$.
\end{enumerate}
\end{definition}

\end{frame}

\begin{frame}
Elements of the vector space are called vectors.

Most often we will assume $\mathbb{F} = \mathbb{C}$ or $\R$.

\vspace{1em}

\begin{example}
The following are vector spaces:
\begin{itemize}
\item $\R^n$
\item $\C^n$
\item $C(\R;\R)$, continuous functions from $\R$ to $\R$
\item $M_{n \times m}$, matrices of size $n \times m$
\item $\mathbb{P}_n$ (polynomials of degree $n$, $p(x) = a_0 + a_1 x + \ldots + a_n x^n$).
\end{itemize}
\end{example}

\end{frame}


\begin{frame}
\begin{alertblock}{Lemma}
For every $\bv \in V$, $0 \bv= \zerovec$.
\end{alertblock}

\vspace{0.5em}

\begin{proof}
\vspace{3cm}
\end{proof}

\end{frame}

\begin{frame}
\begin{alertblock}{Lemma}
For every $\bv \in V$, we have $-\bv = (-1) \times \bv$.
\end{alertblock}

\vspace{0.5em}

\begin{proof}
\vspace{3cm}
\end{proof}

\end{frame}


\begin{frame}

\begin{definition}
A subset $U$ of $V$ is called a \textbf{subspace} of of $V$ if $U$ is also a vector space (using the same addition and scalar multiplication as on $V$).
\end{definition}


\begin{exampleblock}{Proposition}
A subset $U$ of $V$ is a subspace of $V$ if
and only if $U$ satisfies the following three conditions:
\begin{enumerate}
\item  $\zerovec \in U$
\item Closed under addition: $\bu, \bv \in U$ implies $\bu+\bv \in U$
\item Closed under scalar multiplication: $\alpha \in\F$ and $\bu\in U$
implies $\alpha \bu \in U$
\end{enumerate}
\end{exampleblock}
\end{frame}

\begin{frame}
\begin{proof}
($\Rightarrow$)

\vspace{1cm}

($\Leftarrow$)

\vspace{3cm}
\end{proof}
\end{frame}


\begin{frame}{References}

Axler S. \textit{Linear Algebra Done Right}. 3rd ed. Undergraduate Texts in Mathematics. Springer, 2015.
Available from: \href{https://link.springer.com/book/10.1007/978-3-319-11080-6}{https://link.springer.com/book/10.1007/978-3-319-11080-6} 

\vspace{1em}


Runde, Volker (2005). \textit{A Taste of Topology}. Universitext.  url:  \href{https://link.springer.com/book/10.1007/0-387-28387-0}{https://link.springer.com/book/10.1007/0-387-28387-0} 

\vspace{1em}


\indent Treil S. \textit{Linear Algebra Done Wrong}. 2017. Available from: \href{https://www.math.brown.edu/streil/papers/LADW/LADW.html}{https://www.math.brown.edu/streil/papers/LADW/LADW.html}
\end{frame}




\end{document}