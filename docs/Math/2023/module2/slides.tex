\documentclass [aspectratio=169, handout]{beamer}
\usetheme{Boadilla}
\usepackage{textpos} % package for the positioning
\usepackage[]{graphicx}
\usepackage{graphicx}
\usepackage{float}
\usepackage{hyperref}
\usepackage{caption}
\usepackage{subcaption}
\usepackage{algorithm,algpseudocode}
\usepackage{amsmath, amsfonts, amsthm, amssymb}
\usepackage{dsfont}
\usepackage[export]{adjustbox}
\usepackage{tikz}
\usetikzlibrary{positioning}
\usetikzlibrary{arrows, shapes, decorations, automata, backgrounds, fit, petri, calc}
\setbeamertemplate{itemize items}[circle]
\setbeamertemplate{enumerate items}[circle]

\newcommand*{\logofont}{\fontfamily{phv}\selectfont}
\definecolor{uoftblue}{RGB}{0,42,92} % official blue color for uoft
\definecolor{deptgreen}{RGB}{114,192,148} 
\definecolor{deptoran}{RGB}{252,103,63} 

\hypersetup{
  colorlinks   = true, %Colours links instead of boxes
  urlcolor     = uoftblue, %Colour for external hyperlinks
  linkcolor    = black, %Colour of internal links
  citecolor   = black %Colour of citations
}

% lin alg
\newcommand{\bu}{{\mathbf{u}}}
\newcommand{\bv}{{\mathbf{v}}}
\newcommand{\bw}{{\mathbf{w}}}
\newcommand{\bx}{\mathbf{x}}
\newcommand{\zerovec}{{\mathbf{0}}}

% other useful stuff
\newcommand{\Id}{{\mathds{1}}}
\newcommand{\R}{{\mathbb{R}}}
\newcommand{\C}{{\mathbb{C}}}
\newcommand{\F}{{\mathbb{F}}}
\newcommand{\Z}{{\mathbb{Z}}}
\newcommand{\N}{{\mathbb{N}}}
\newcommand{\Q}{{\mathbb{Q}}}
\newcommand{\cP}{\mathcal{P}}

\newtheorem{proposition}[theorem]{Proposition}


\beamertemplatenavigationsymbolsempty

% block
% example block
% alert block


\title[]{Module 2: Set Theory \\ {\large Operational math bootcamp}\\ \includegraphics[width=8cm]{dept_logo.png}\vspace{-1em}}
\author[]{Emma Kroell}
\institute[]{University of Toronto}
\date{July 12, 2023}

% set color
\setbeamercolor{title in head/foot}{bg=white}
\setbeamercolor{author in head/foot}{bg=white}
\setbeamercolor{date in head/foot}{fg=uoftblue}
\setbeamercolor{date in head/foot}{bg=white}
\setbeamercolor{title}{fg=uoftblue}
\setbeamerfont{title}{series=\bfseries}
\setbeamercolor{frametitle}{fg=uoftblue}
\setbeamerfont{frametitle}{series=\bfseries}
\setbeamercolor*{item}{fg=uoftblue}
\setbeamercolor{block title}{bg=uoftblue}
\setbeamercolor{block title}{fg=white}
\setbeamercolor{block body}{bg=uoftblue!9!white}
\setbeamercolor{block title example}{bg=deptgreen}
\setbeamercolor{block title example}{fg=white}
\setbeamercolor{block body example}{bg=deptgreen!13!white}
\setbeamercolor{block title alerted}{bg=deptoran}
\setbeamercolor{block title alerted}{fg=white}
\setbeamercolor{block body alerted}{bg=deptoran!10!white}


% set logo at non-title pages
\logo{\includegraphics[height=0.8cm]{dept_logo.png}\vspace*{-.045\paperheight}\hspace*{.78\paperwidth}}

% set margin
\setbeamersize{text margin left=10mm,text margin right=10mm}

\begin{document}
{
\setbeamertemplate{logo}{}
\begin{frame}
    %\vspace{0.5in}
    \titlepage
    %\begin{textblock*}{10cm}(3.5cm,-7.5cm)
      %  \includegraphics[width=8cm]{dept_logo.png}
    %\end{textblock*}
\end{frame}
}

\begin{frame}{Outline}
    \begin{itemize}
    \setlength \itemsep{1em}
    	\item Review of basic set theory
	\item Ordered Sets
        \item Functions
    \end{itemize}
\end{frame}

\begin{frame}{Introduction to Set Theory}
\begin{itemize}
  \setlength\itemsep{1em}
\item We define a \emph{set} to be a collection of mathematical objects.
\item  If $S$ is a set and $x$ is one of the objects in the set, we say $x$ is an element of $S$ and denote it by $x\in S$.
\item The set of no elements is called empty set and is denoted by $\emptyset$.
\end{itemize}
\end{frame}

\begin{frame}
\begin{definition}[Subsets, Union, Intersection]
Let $S, T$ be sets. 
\begin{itemize}
    \item We say that $S$ is a \emph{subset} of $T$, denoted $S\subseteq T$, if $s\in S$ implies $s\in T$. 
    \item We say that $S=T$ if $S\subseteq T$ and $T\subseteq S$.
    \item We define the \emph{union} of $S$ and $T$, denoted $S \cup T$, as all the elements that are in \emph{either} $S$ or $T$.
    \item We define the \emph{intersection} of $S$ and $T$, denoted $S \cap T$, as all the elements that are in \emph{both} $S$ and $T$.
    \item We say that $S$ and $T$ are \emph{disjoint} if $S \cap T = \emptyset$.
\end{itemize}
\end{definition}
\end{frame}

\begin{frame}{Some examples}

\begin{example}
$\mathbb{N} \subseteq \mathbb{N}_0 \subseteq \mathbb{Z} \subseteq \mathbb{Q} \subseteq \mathbb{R} \subseteq \mathbb{C}$
\end{example}

\vspace{1em}

\begin{example}  Let $a, b \in \R$ such that $a < b$. \\
Open interval: $(a,b) := \{x \in \R : a < x < b \}$  ($a,b$ may be $- \infty$ or $+ \infty$)\\
Closed interval: $[a,b] := \{x \in \R : a \leq x \leq b \}$ \\
We can also define half-open intervals. 
\end{example}

\end{frame}


\begin{frame}
\begin{example}
Let $A = \{x \in \N: 3 | x \}$ and $B = \{x \in \N: 6 | x \}$
Show that $B \subseteq A$. 
\end{example}
\textit{Proof.}
\vspace{7em}
\end{frame}

\begin{frame}{Difference of sets}
\begin{definition}
Let $A,B \subseteq X$. We define the \emph{set-theoretic difference} of $A$ and $B$, denoted $A \setminus B$ (sometimes $A-B$) as the elements of $X$ that are in $A$ but \emph{not} in $B$. 

The complement of a set $A \subseteq X$ is the set $A^c := X \setminus A$.
\end{definition}

\vspace{1em}

\begin{example}
Let $X\subseteq\R$ be defined as $X = \{x \in \R: 0 < x \leq 40 \} = (0,40]$. Then 

\vspace{0.5em}
$X^c=$% \{x \in \R: x \leq 0 \text{ or } x > 40 \} = (-\infty, 0] \, \cup \, (40, \infty) $.
\vspace{3em}
\end{example}

\end{frame}


\begin{frame}

Recall that for sets $S, T$:
    \begin{itemize}
    	\item  the \emph{union} of $S$ and $T$, denoted $S \cup T$, is all the elements that are in \emph{either} $S$ and $T$
	\item  and the \emph{intersection} of $S$ and $T$, denoted $S \cap T$, is all the elements that are in \emph{both} $S$ and $T$.
    \end{itemize}

We extend the definition of union and intersection to an arbitrary family of sets as follows:

\begin{definition}
Let $S_\alpha$, $\alpha \in A$, be a family of sets. $A$ is called the \emph{index set}. We define
\begin{equation*}
    \bigcup_{\alpha \in A} S_\alpha := \{ x: \exists \alpha \text{ such that } x \in S_\alpha \},
\end{equation*}
\begin{equation*}
    \bigcap_{\alpha \in A} S_\alpha := \{ x: x \in S_\alpha \text{ for all } \alpha \in A \}.
\end{equation*}
\end{definition}

\end{frame}

\begin{frame}

\begin{example}
$$\bigcup_{n=1}^\infty [-n,n] = $$
$$\bigcap_{n=1}^\infty \left( -\frac{1}{n},\frac{1}{n} \right) = $$
\end{example}

\end{frame}


\begin{frame}
\begin{theorem}[De Morgan's Laws]
Let $\{S_\alpha\}_{\alpha \in A}$ be an arbitrary collection of sets. Then 
\begin{equation*}
    \left( \bigcup_{\alpha \in A} S_\alpha \right)^c = \bigcap_{\alpha \in A}  S_\alpha^c \quad \text{and} \quad \left( \bigcap_{\alpha \in A} S_\alpha \right)^c = \bigcup_{\alpha \in A}  S_\alpha^c
\end{equation*}
\end{theorem}
\textit{Proof.}
\vspace{8em}

\end{frame}

\begin{frame}
Since a set is itself a mathematical object, a set can itself contain sets.
\begin{definition}
The power set $\cP(S)$ of a set $S$ is the set of all subsets of $S$.
\end{definition}

\vspace{1em}

\begin{example}
Let $S = \{a,b,c\}$.  \\
Then $\cP(S) = $ 
\vspace{5em}
\end{example}
\end{frame}

\begin{frame}
Another way of building a new set from two old ones is the Cartesian product of two sets.

\begin{definition}\label{def:cartes_prod}
Let $S,T$ be sets. The \emph{Cartesian product} $S\times T$ is defined as the set of tuples with elements from $S,T$, i.e 
\begin{equation*}
    S\times T = \{ (s,t) \; \colon \; s \in S \; \text{ and } \; t \in T\}.
\end{equation*}
\end{definition}
\vspace{1em}

This can also be extended inductively to a finite family of sets. 

\end{frame}

\begin{frame}{Ordered set}
\begin{definition}
A \emph{relation} $R$ on a set $X$ is a subset of $X \times X$. A relation $\leq$ is called a \emph{partial order} on $X$ if it satisfies
\begin{enumerate}
\setlength\itemsep{1em}
\item reflexivity: 
\item transitivity: 
\item anti-symmetry: 
\end{enumerate}
\vspace{1em}
The pair $(X, \leq)$ is called a \emph{partially ordered set}.

\vspace{1em}

A \emph{chain} or \emph{totally ordered set} $C \subseteq X$ is a subset with the property $x \leq y$ or $y \leq x$ for any $x,y \in C$.
\end{definition}

\end{frame}

\begin{frame}
\begin{example}
The real numbers with the usual ordering, $(\R, \leq)$ are totally ordered. 
\end{example}

\begin{example}
The power set of a set $X$ with the ordering given by $\subseteq$, $(\cP(X), \subseteq)$ is a partially ordered set. 
\end{example}
\end{frame}

\begin{frame}
\begin{example}
Let $X = \{a,b,c,d\}$. What is $\cP(X)$? Find a chain in $\cP(X)$.

\vspace{1em}
\pause
$\cP(X) = \{\emptyset,\{a\},\{b\},\{c\},\{d\},\{a,b\},\{b,c\},\{c,d\},\{b,d\},\{a,c\},\{a,d\},\{a,b,c\}, $
 $ \{b,c,d\},\{a,b,d\},\{a,c,d\},X\}$


\vspace{3cm}
\end{example}
\end{frame}

\begin{frame}
\begin{example}
Consider the set $C([0,1],\R):= \{f:[0,1] \to \R : f \text{ is continuous}\}$.

\vspace{1em}

For two functions $f,g \in C([0,1],\R)$, we define the ordering as $f \leq g$ if $f(x) \leq g(x)$ for $x \in [0,1]$. Then $(C([0,1],\R),\leq)$ is a partially ordered set. 

\vspace{1em}

Can you think of a chain that is a subset of $(C([0,1],\R)$?
\end{example}

\end{frame}



\begin{frame}
\begin{definition}
A non-empty partially ordered set $(X,\leq)$ is \emph{well-ordered} if every non-empty subset $A \subseteq X$ has a mimimum element.
\end{definition}

\vspace{2em}

Example: 
\vspace{0.5em}

$(\N, \leq)$ is... 
\vspace{1em}

$(\R,\leq)$ is...
\end{frame}



\begin{frame}
\begin{definition}%Marcoux Definition 2.12
Let $(X,\leq)$ be a partially ordered set and $S\subseteq X$. 

 \vspace{1em}

Then $x\in X$ is an \emph{upper bound} for $S$ if for all $s \in S$ we have $s\leq x$.\\
 Similarly, $y\in X$ is a \emph{lower bound} for $S$ if for all $s\in S$, $y\leq s$. 
 
 \vspace{1em}
 
 If there exists an upper bound for $S$, we call $S$ \emph{bounded above} and if there exists a lower bound for $S$, we call $S$ \emph{bounded below}. If $S$ is bounded above and bounded below, we say $S$ is \emph{bounded}. 
\end{definition}

\end{frame}

\begin{frame}

We can also ask if there exists a least upper bound or a greatest lower bound. 

\begin{definition}%Marcoux Definition 2.12
Let $(X,\leq)$ be a partially ordered set and $S\subseteq X$. 

\vspace{1em}

We call $x\in X$ \emph{least upper bound} or \emph{supremum}, denoted $x= \sup S$, if $x$ is an upper bound and for any other upper bound $y\in X$ of $S$ we have $x\leq y$. 

\vspace{1em}

Likewise, $x\in X$ is the \emph{greatest lower bound} or \emph{infimum} for $S$, denoted $x= \inf S$, if it is a lower bound and for any other lower bound $y\in X$, $y\leq x$.
\end{definition}

\vspace{1em}
Note that the supremum and infimum of a bounded set do not necessarily need to exist. However, if they do exists they are unique, which justifies the article \emph{the} (exercise). Nevertheless, the reals have a remarkable property, which we will take as an axiom.

\end{frame}

\begin{frame}


\begin{alertblock}{Completeness Axiom}
Let $S\subseteq \R$ be bounded above. Then there exists $r\in \R$ such that $r= \sup S$, i.e. $S$ has a least upper bound. 
\end{alertblock}

\vspace{1em}

By setting $S^\prime = -S:= \{ -s \; \colon \; s\in S\}$ and noting $\inf S = - \sup S^\prime$, we obtain a similar statement for infima if $S$ is bounded below. As mentioned above, this property is fairly special, for example it fails for the rationals.

\vspace{1em}

\begin{example}
Let $S= \{q \in \Q \; \colon \; x^2 < 7 \}$. Then $S$ is bounded above in $\Q$, but there exists no least upper bound in $\Q$.
\end{example}



\end{frame}

\begin{frame}
There is a nice alternative characterization for suprema in the real numbers. 

\begin{proposition}
 Let $S\subseteq \R$ be bounded above. Then $r= \sup S$ if and only if $r$ is an upper bound and for all $\epsilon>0$ there exists an $s\in S$ such that $r-\epsilon <s$. 
\end{proposition}

\textit{Proof.}
($\Rightarrow$) 
\vspace{3cm}

\end{frame}

\begin{frame}

\textit{Proof.}
($\Leftarrow$) 
\vspace{3cm}


Using the same trick, we may obtain a similar result for infima.

\vspace{1em}

\begin{example}
Consider $S = \{1/n \; \colon \; n\in \N\}$. Then $\sup S = 1$ and $\inf S = 0$.
\end{example}
\end{frame}

\begin{frame}{Functions}
\begin{definition}
A function $f$ from a set $X$ to a set $Y$ is a subset of $X \times Y$ with the properties:
\begin{enumerate}
    \item For every $x \in X$, there exists a $y \in Y$ such that $(x,y) \in f$
    \item If $(x,y) \in f$ and $(x,z) \in f$, then $y = z$.
\end{enumerate}
$X$ is called the \emph{domain} of $f$.
\end{definition}
How does this connect to other descriptions of functions you may have seen?
\vspace{2em}

\begin{example}
For a set $X$, the identity function is:
$$ 1_X: X \to X, \quad x \mapsto x $$
\end{example}
\end{frame}



\begin{frame}
\begin{definition}[Image and pre-image]
Let $f:X \to Y$ and $A \subseteq X$ and $B \subseteq Y$. 
\begin{itemize}
\item The \emph{image} of $f$ is the set $f(A) := \{f(x): x \in A \}$.
\item The \emph{pre-image} of $f$ is the set $f^{-1}(B) := \{x: f(x) \in B \}$.
\end{itemize}
\end{definition}

\vspace{1em}

Helpful way to think about it for proofs:

\vspace{0.5em}
\textbf{Image:} If $y \in f(A)$, then $y \in Y$, and there exists an $x \in A$ such that $y = f(x)$.

\vspace{0.5em}
\textbf{Pre-image:} If $x \in f^{-1}(B)$, then $x \in X$ and $f(x) \in B$.


\end{frame}


\begin{frame}
\begin{definition}[Surjective, injective and bijective]
Let $f:X \to Y$, where $X$ and $Y$ are sets. Then
\begin{itemize}
    \item $f$ is \emph{injective} if $x_1 \neq x_2$ implies $f(x_1) \neq f(x_2)$
    \item $f$ is \emph{surjective} if for every $y \in Y$, there exists an $x \in X$ such that $y = f(x)$
    \item $f$ is \emph{bijective} if it is both injective and bijective
\end{itemize}
\end{definition}

\begin{example}
Let $f:X \to Y$, $ x \mapsto x^2$. \\
$f$ is surjective if \\
$f$ is injective if \\
$f$ is bijective if \\
$f$ is neither surjective nor injective if
\end{example}
\end{frame}


\begin{frame}{References}

Marcoux, Laurent W. (2019). \textit{PMATH 351 Notes}. url: \href{https://www.math.uwaterloo.ca/~lwmarcou/notes/pmath351.pdf}{https://www.math.uwaterloo.ca/~lwmarcou/notes/pmath351.pdf}

\vspace{1em}


Runde ,Volker (2005). \textit{A Taste of Topology}. Universitext.  url:  \href{https://link.springer.com/book/10.1007/0-387-28387-0}{https://link.springer.com/book/10.1007/0-387-28387-0} 

\vspace{1em}

Zwiernik, Piotr (2022). \textit{Lecture notes in Mathematics for Economics and Statistics}. url: \href{http://84.89.132.1/~piotr/docs/RealAnalysisNotes.pdf}{http://84.89.132.1/~piotr/docs/RealAnalysisNotes.pdf} \\




\end{frame}



\end{document}