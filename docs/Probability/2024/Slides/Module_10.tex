\documentclass [aspectratio=169]{beamer}
\usetheme{Boadilla}
\usepackage{textpos} % package for the positioning
\usepackage[]{graphicx}
\usepackage{graphicx}
\usepackage{float}
\usepackage{hyperref}
\usepackage{caption}
\usepackage{subcaption}
\usepackage{comment}
\usepackage{algorithm,algpseudocode}
\usepackage[export]{adjustbox}
\usepackage{tikz}
\usetikzlibrary{positioning}
\usetikzlibrary{matrix}
\usetikzlibrary{positioning}
\usetikzlibrary{arrows, shapes, decorations, automata, backgrounds, fit, petri, calc}
\newcommand{\tikzmark}[1]{\tikz[overlay,remember picture] \node (#1) {};}
\usepackage{xcolor}
\usepackage{pgfplots}

%\pgfplotsset{compat=1.10}
\usepgfplotslibrary{fillbetween}
\usepackage{filecontents}


\pgfplotsset{compat=1.7}
\usetikzlibrary{positioning}
\usetikzlibrary{arrows, shapes, decorations, automata, backgrounds, fit, petri, calc}
\setbeamertemplate{itemize items}[circle]
\setbeamertemplate{enumerate items}[circle]
\setbeamertemplate{itemize subitem}{$\triangleright$}

\newcommand{\notimplies}{\;\not\!\!\!\implies}
\newcommand*{\logofont}{\fontfamily{phv}\selectfont}
\definecolor{uoftblue}{RGB}{6,41,88} % official blue color for uoft
\definecolor{yamabuki}{RGB}{255,177,27}
\definecolor{sakuranezumi}{RGB}{177,150,147}
\definecolor{enji}{RGB}{159,53,58}
\definecolor{torinoko}{RGB}{218, 201, 166}
\definecolor{baikocha}{RGB}{137,145,107}
\definecolor{ginnezumi}{RGB}{145,152,159}


\vspace{1in}
\title[]{DoSS Summer Bootcamp Probability \\ Module 10}
\author[]{Ichiro Hashimoto}
\institute[]{University of Toronto}
\date{July 28, 2023}

% set color
\setbeamercolor{title in head/foot}{bg=white}
\setbeamercolor{author in head/foot}{bg=white}
\setbeamercolor{date in head/foot}{fg=uoftblue}
\setbeamercolor{date in head/foot}{bg=white}
\setbeamercolor{title}{fg=uoftblue}
\setbeamerfont{title}{series=\bfseries}
\setbeamercolor{frametitle}{fg=uoftblue}
\setbeamerfont{frametitle}{series=\bfseries}
\setbeamercolor*{item}{fg=uoftblue}
\setbeamercolor{block title}{bg=uoftblue}
\setbeamercolor{block title}{fg=white}
\setbeamercolor{block body}{bg=uoftblue!5!white}

% set logo at non-title pages
\logo{\includegraphics[height=0.8cm]{logo_uoft.png}\vspace*{-.055\paperheight}\hspace*{.85\paperwidth}}

% set margin
\setbeamersize{text margin left=10mm,text margin right=10mm}

\newcommand{\mc}{\mathcal}


\begin{filecontents*}{LOFT.txt}
        n   an   a    
        1   1    0   
        2   0.5  0
        5   0.2    0   
        10   0.1    0   
        20   0.05    0   
        40   0.025    0   
        50   0.02    0   
        100   0.01    0   
\end{filecontents*}


\begin{document}
{
\setbeamertemplate{logo}{}
\begin{frame}
    \vspace{0.5in}
    \titlepage
    \begin{textblock*}{4cm}(0.5cm,-7.5cm)
        \includegraphics[width=4cm]{logo_uoft.png}
    \end{textblock*}
    \begin{textblock*}{8cm}(5.0cm,-7cm)
        \huge \color{uoftblue}{$\Bigr\rvert$ \hspace{0.15cm} \textbf{\logofont Statistical Sciences}}
    \end{textblock*}
\end{frame}
}

\begin{frame}{Recap}
Learnt in last module:\\
\vspace{0.1in}
\begin{itemize}
    \item Convergence of functions of random variables
    \begin{itemize}
        \item Slutsky's theorem
        \item Continuous mapping theorem
    \end{itemize}
    \item Laws of large numbers
    \begin{itemize}
        \item WLLN
        \item SLLN
        \item Glivenko-Cantelli theorem
    \end{itemize}
    \item Central limit theorem
\end{itemize}
\end{frame}

\begin{frame}{Outline}
\begin{itemize}
    \item Limit Theorems and Counterexamples
        \begin{itemize}
        \item Law of Large Numbers
        \item Monotone Convergence Theorem
        \item Dominated Convergence Theorem
        \item More about CLT
     \end{itemize}
\end{itemize}
\end{frame}

\begin{frame}{Limit Theorems and Counterexamples}
    \textbf{Recall:} For the law of large numbers to hold, the assumption $E|X|<\infty$ is crucial. \\ 
    \begin{block}{Law of Large Numbers fail for infinite mean i.i.d. random variables}
    If $X_1 X_2, \dots$ are i.i.d. to $X$ with $E|X_i| = \infty$, then for $S_n = X_1 + \cdots + X_n$, $P(\lim_{n\to \infty}S_n/n \in (-\infty, \infty))=0$.
    \end{block}
    \textbf{Proof: Omitted}
    \vspace{1.8in}
\end{frame}

\begin{frame}{Limit Theorems and Counterexamples}
    \begin{block}{Monotone Convergence Theorem}
    If $X_n \geq c$ and $X_n \nearrow X$, then $EX_n \nearrow EX$ 
    \end{block}
    \textbf{Usage: }
    \vspace{1.8in}
\end{frame}

\begin{frame}{Limit Theorems and Counterexamples}
    \begin{block}{Monotone Convergence Theorem}
    If $X_n \geq 0$ and $X_n \nearrow X$, then $EX_n \nearrow EX$ 
    \end{block}
    \textbf{Counterexample when $X_n$ is not lower bounded: }
    \vspace{1.8in}
\end{frame}

\begin{frame}{Limit Theorems and Counterexamples}
    \begin{block}{Dominated Convergence Theorem}
    If $X_n \to X$ a.s. and $|X_n| \leq Y$ a.s. for all $n$ and $Y$ is integrable, then $EX_n \to EX$ 
    \end{block}
    \textbf{Usage: }
    \vspace{1.8in}
\end{frame}

\begin{frame}{Limit Theorems and Counterexamples}
    \begin{block}{Dominated Convergence Theorem}
    If $X_n \to X$ a.s. and $|X_n| \leq Y$ a.s. for all $n$ and $Y$ is integrable, then $EX_n \to EX$ 
    \end{block}
    \textbf{Counterexample when $X_n$ is not dominated by an integrable random variable: }
    \vspace{1.8in}
\end{frame}

\begin{frame}{Limit Theorems and Counterexamples}
    \begin{block}{More about CLT: Delta method}
    Suppose $X_n$ are i.i.d. random variables with $EX_n=0, VAR(X_n) = \sigma^2 >0$. Let $g$ be a measurable function that is differentiable at $0$ with $g^\prime(0) \neq 0$. Then
    \[
    \sqrt{n}\left(g\left(\frac{\sum_{k=1}^nX_k}{n} - g(0) \right)\right) \to N(0, \sigma^2g^\prime(0)^2) \quad \text{weakly.}
    \] 
    \end{block}
    \textbf{Proof under stronger assumption: } Here, we suppose $g$ is continuously differentiable on $\mathbb{R}$. If you are interested in a general proof refer to Robert Keener's \textit{Theoretical Statistics}.
    \vspace{1.8in}
\end{frame}


\end{document}
