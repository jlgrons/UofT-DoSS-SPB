\documentclass [aspectratio=169]{beamer}
\usetheme{Boadilla}
\usepackage{textpos} % package for the positioning
\usepackage[]{graphicx}
\usepackage{graphicx}
\usepackage{float}
\usepackage{hyperref}
\usepackage{caption}
\usepackage{subcaption}
\usepackage{comment}
\usepackage{algorithm,algpseudocode}
\usepackage[export]{adjustbox}
\usepackage{tikz}
\usetikzlibrary{positioning}
\usetikzlibrary{matrix}
\usetikzlibrary{positioning}
\usetikzlibrary{arrows, shapes, decorations, automata, backgrounds, fit, petri, calc}
\newcommand{\tikzmark}[1]{\tikz[overlay,remember picture] \node (#1) {};}
\usepackage{xcolor}
\usepackage{pgfplots}

%\pgfplotsset{compat=1.10}
\usepgfplotslibrary{fillbetween}
\usepackage{filecontents}


\pgfplotsset{compat=1.7}
\usetikzlibrary{positioning}
\usetikzlibrary{arrows, shapes, decorations, automata, backgrounds, fit, petri, calc}
\setbeamertemplate{itemize items}[circle]
\setbeamertemplate{enumerate items}[circle]
\setbeamertemplate{itemize subitem}{$\triangleright$}

\newcommand{\notimplies}{\;\not\!\!\!\implies}
\newcommand*{\logofont}{\fontfamily{phv}\selectfont}
\definecolor{uoftblue}{RGB}{6,41,88} % official blue color for uoft
\definecolor{yamabuki}{RGB}{255,177,27}
\definecolor{sakuranezumi}{RGB}{177,150,147}
\definecolor{enji}{RGB}{159,53,58}
\definecolor{torinoko}{RGB}{218, 201, 166}
\definecolor{baikocha}{RGB}{137,145,107}
\definecolor{ginnezumi}{RGB}{145,152,159}


\vspace{1in}
\title[]{DoSS Summer Bootcamp Probability \\ Module 8}
\author[]{Ichiro Hashimoto}
\institute[]{University of Toronto}
\date{July 22, 2025}

% set color
\setbeamercolor{title in head/foot}{bg=white}
\setbeamercolor{author in head/foot}{bg=white}
\setbeamercolor{date in head/foot}{fg=uoftblue}
\setbeamercolor{date in head/foot}{bg=white}
\setbeamercolor{title}{fg=uoftblue}
\setbeamerfont{title}{series=\bfseries}
\setbeamercolor{frametitle}{fg=uoftblue}
\setbeamerfont{frametitle}{series=\bfseries}
\setbeamercolor*{item}{fg=uoftblue}
\setbeamercolor{block title}{bg=uoftblue}
\setbeamercolor{block title}{fg=white}
\setbeamercolor{block body}{bg=uoftblue!5!white}

% set logo at non-title pages
\logo{\includegraphics[height=0.8cm]{logo_uoft.png}\vspace*{-.055\paperheight}\hspace*{.85\paperwidth}}

% set margin
\setbeamersize{text margin left=10mm,text margin right=10mm}

\newcommand{\mc}{\mathcal}


\begin{filecontents*}{LOFT.txt}
        n   an   a    
        1   1    0   
        2   0.5  0
        5   0.2    0   
        10   0.1    0   
        20   0.05    0   
        40   0.025    0   
        50   0.02    0   
        100   0.01    0   
\end{filecontents*}


\begin{document}
{
\setbeamertemplate{logo}{}
\begin{frame}
    \vspace{0.5in}
    \titlepage
    \begin{textblock*}{4cm}(0.5cm,-7.5cm)
        \includegraphics[width=4cm]{logo_uoft.png}
    \end{textblock*}
    \begin{textblock*}{8cm}(5.0cm,-7cm)
        \huge \color{uoftblue}{$\Bigr\rvert$ \hspace{0.15cm} \textbf{\logofont Statistical Sciences}}
    \end{textblock*}
\end{frame}
}

\begin{frame}{Recap}
Learnt in last module:\\
\vspace{0.1in}
\begin{itemize}
    \item Stochastic convergence
    \begin{itemize}
        \item Convergence in distribution
        \item Convergence in probability
        \item Convergence almost surely
        \item Convergence in $L^p$
        \item Relationship between convergences
    \end{itemize}
\end{itemize}
\end{frame}

\begin{frame}{Outline}
\begin{itemize}
    \item Convergence of functions of random variables
    \begin{itemize}
        \item Slutsky's theorem
        \item Continuous mapping theorem
    \end{itemize}
    \item Laws of large numbers
    \begin{itemize}
        \item WLLN
        \item SLLN
        \item Glivenko-Cantelli theorem
    \end{itemize}
    \item Central limit theorem
\end{itemize}
\end{frame}


\begin{frame}{Convergence of functions of random variables}
    \textbf{Recall: Stochastic convergence}
    If $X_n \to X$, $Y_n \to Y$ in some sense, how is the limiting property of $f(X_n, Y_n)$?
    \vspace{0.1in}
    \uncover<2->{
    \begin{block}{Convergence of functions of random variables (a.s.)}
    Suppose the probability space is complete, if $X_n  \xrightarrow[]{a.s.} X, Y_n  \xrightarrow[]{a.s.} Y$, then for any real numbers $a, b$, \begin{itemize}
        \item $aX_n + bY_n  \xrightarrow[]{a.s.} aX + bY$;
        \item $X_n Y_n  \xrightarrow[]{a.s.} XY$.
    \end{itemize}
    \end{block}
    \vspace{0.1in}
    \textbf{Remark:}
    \begin{itemize}
        \item Still require all the random variables to be defined on the same probability space
        
    \end{itemize}}
\vspace{0.2in}
\end{frame}


\begin{frame}{Convergence of functions of random variables}
    \begin{block}{Convergence of functions of random variables (probability)}
     Suppose the probability space is complete, if $X_n  \xrightarrow[]{P} X, Y_n  \xrightarrow[]{P} Y$, then for any real numbers $a, b$, \begin{itemize}
        \item $aX_n + bY_n  \xrightarrow[]{P} aX + bY$;
        \item $X_n Y_n  \xrightarrow[]{P} XY$.
    \end{itemize}
    \end{block}
    \vspace{0.1in}
    \textbf{Remark:}
    \begin{itemize}
        \item Still require all the random variables to be defined on the same probability space
        
    \end{itemize}
\end{frame}

\begin{frame}{Convergence of functions of random variables}
    \begin{block}{Convergence of functions of random variables ($L^p$)}
     Suppose the probability space is complete, if $X_n  \xrightarrow[]{L^p} X, Y_n  \xrightarrow[]{L^p} Y$, then for any real numbers $a, b$, \begin{itemize}
        \item $aX_n + bY_n  \xrightarrow[]{L^p} aX + bY$;
    \end{itemize}
    \end{block}
    \vspace{0.1in}
    \textbf{Remark:}
    \begin{itemize}
        \item Still require all the random variables to be defined on the same probability space
        
    \end{itemize}
    \vspace{0.5in}
\end{frame}



\begin{frame}{Convergence of functions of random variables}
\textbf{Remark}: Convergence in distribution is different.
\vspace{0.1in}
        \begin{block}{Slutsky's theorem}
   If $X_n \xrightarrow[]{d} X$ and $Y_n \xrightarrow[]{P} c$ ($c$ is a constant), then
\begin{itemize}
    \item $X_{n}+Y_{n}\ {\xrightarrow {d}}\ X+c$;
    \item $X_{n}Y_{n}\ {\xrightarrow {d}}\ cX$;
    \item $X_{n}/Y_{n}\ {\xrightarrow {d}}\ X/c$, where $c \neq 0$.
\end{itemize}
    \end{block}
    \vspace{0.1in}
    \uncover<2->{
    \textbf{Remark:}
    \begin{itemize}
        \item The theorem remains valid if we replace all the convergence in distribution with convergence in probability. 
    \end{itemize}}
\end{frame}



\begin{frame}{Convergence of functions of random variables}
\textbf{Remark}: The requirement that $Y_n \xrightarrow[]{P} c$ ($c$ is a constant) is necessary.\\
\vspace{0.1in}
\uncover<2->{
        \textbf{Examples:}\\
        $X_n \sim \mc{N}(0,1), Y_n = -Xn$, then
        \begin{itemize}
            \item $X_n \xrightarrow[]{d} Z \sim \mc{N}(0,1)$, $Y_n \xrightarrow[]{d} Z \sim \mc{N}(0,1)$;
            \item $X_n + Y_n \xrightarrow[]{d} 0$;
            \item $X_nY_n = -X_n^2 \xrightarrow[]{d} -\chi^2(1)$;
            \item $X_n/Y_n = -1$.
        \end{itemize}}
\end{frame}


\begin{frame}{Convergence of functions of random variables}
    \begin{block}{Continuous mapping theorem}
    Let $X_n$, $X$ be random variables, if $g(\cdot): \mathbb{R} \to \mathbb{R}$ satisfies $\mathbb{P}(X \in D_g) = 0$, then
    \begin{itemize}
        \item $X_n  \xrightarrow[]{a.s.} X \quad \Rightarrow \quad g(X_n)  \xrightarrow[]{a.s.} g(X)$ ;
        \item $X_n  \xrightarrow[]{P} X \quad \Rightarrow \quad g(X_n)  \xrightarrow[]{P} g(X)$ ;
        \item $X_n  \xrightarrow[]{d} X \quad \Rightarrow \quad g(X_n)  \xrightarrow[]{d} g(X)$ ;
    \end{itemize}
    where $D_g$ is the set of discontinuity points of $g(\cdot)$. 
    \end{block}
    \vspace{0.1in}
    \uncover<2->{
    \textbf{Remark:}
\begin{itemize}
    \item If $g(\cdot)$ is continuous, then ...
    \item If $X$ is a continuous random variable, and $D_g$ only include countably many points, then ... 
\end{itemize}}
\end{frame}

\begin{frame}{Law of large numbers}
    \begin{block}{Weak Law of Large Numbers (WLLN)}
    If $X_1, X_2, \cdots, X_n$ are i.i.d. random variables, $\mu = \mathbb{E}(|X_i|) < \infty$, then
    $$\bar{X} = \frac{\sum_{i = 1}^n X_i}{n} \quad \xrightarrow[]{P} \quad \mu.$$
    \end{block}
    \vspace{0.1in}
    \textbf{Remark:}\\
    A more easy-to-prove version is the $L^2$ weak law, where an additional assumption $Var(X_i) < \infty$ is required.\\
    \vspace{0.1in}
    \textbf{Sketch of the proof:}\\
    \vspace{0.8in}
\end{frame}

\begin{frame}{Law of large numbers}
\textbf{A generalization of the theorem: triangular array}
\begin{block}{Triangular array}
A triangular array of random variables is a collection $\{X_{n,k}\}_{1 \le k \le n}$. 
\end{block}
\begin{equation*}
    \begin{aligned}
    & X_{1,1} \\
    & X_{2,1}, X_{2,2}\\
    & X_{3,1}, X_{3,2}, X_{3,3}\\
    & \quad \vdots \\
    & X_{n, 1}, X_{n,2}, \cdots, X_{n,n}
    \end{aligned}
\end{equation*}
\vspace{0.1in}
\textbf{Remark:} We can consider the limiting property of the row sum $S_n$.
\end{frame}

\begin{frame}{Law of Large Numbers}
     \begin{block}{$L^2$ weak law for triangular array}
    Suppose $\{X_{n,k}\}$ is a triangular array, $n = 1, 2, \cdots, k =1, 2, \cdots, n$. Let $S_n = \sum_{k = 1}^n X_{n,k}$, $\mu_n = \mathbb{E}(S_n)$, if $\sigma_n^2/b_n^2 \to 0$, where $\sigma_n^2 = Var(S_n)$ and $b_n$ is a sequence of positive real numbers, then
    $$ \frac{S_n - \mu_n}{b_n} \quad \xrightarrow[]{P} \quad 0.$$
    \end{block}
    \vspace{0.1in}
    \textbf{Remark:}\\
    The $L^2$ weak law for i.i.d. random variables is a special case of that for triangular array.
    \vspace{0.7in}
\end{frame}

\begin{frame}{Law of large numbers}
\textbf{Proof:}\\
\vspace{1.5in}
% actually very easy, markov inequality
\uncover<2->{
\textbf{Remark:}\\
A more generalized version incorporates truncation, then the second-moment constraint is relieved.}
\end{frame}


\begin{frame}{Law of large numbers}
    \begin{block}{Strong Law of Large Numbers (SLLN)}
    Let $X_1, X_2, \cdots$ be an i.i.d. sequence satisfying $\mathbb{E}(X_i) = \mu$ and $\mathbb{E}(|X_i|) < \infty$, then 
    $\bar{X} = \frac{\sum_{i = 1}^n X_i}{n} \quad \xrightarrow[]{a.s.} \quad \mu $.
    \end{block}
    \vspace{0.1in}
    \textbf{Remark:} The proof needs Borel-Cantelli lemma.
    \vspace{0.1in}
    \uncover<2->{
    \begin{block}{Glivenko-Cantelli theorem}
    Let $X_i, i = 1, \cdots,n$ i.i.d. with distribution function $F(\cdot)$, and let $F_n(x) = \frac{1}{n} \sum_{i=1}^n I(X_i \le x)$, then as $n \to \infty$,
    $$
    \sup_{x \in \mathbb{R}}|F(x) - F_n(x)| \quad \to \quad  0, \quad a.s.
    $$
    \end{block}}
\end{frame}

\begin{frame}{Law of large numbers}
    \textbf{Proof:}\\
    % https://www.math.stonybrook.edu/~rdhough/mat639-spring17/lectures/lecture3.pdf  P60
    \vspace{2.5in}
\end{frame}

\begin{frame}{Limit Theorems and Counterexamples}
    \textbf{Recall:} For the law of large numbers to hold, the assumption $E|X|<\infty$ is crucial. \\ 
    \begin{block}{Law of Large Numbers fail for infinite mean i.i.d. random variables}
    If $X_1 X_2, \dots$ are i.i.d. to $X$ with $E|X_i| = \infty$, then for $S_n = X_1 + \cdots + X_n$, $P(\lim_{n\to \infty}S_n/n \in (-\infty, \infty))=0$.
    \end{block}
    \textbf{Proof: Omitted}
    \vspace{1.8in}
\end{frame}


\begin{frame}{Central Limit Theorem}
\textbf{What is the limiting distribution of the sample mean?}\\
\begin{block}{Classic CLT}
Suppose $X_1, \cdots X_n$ is a sequence of i.i.d. random variables with $\mathbb{E}(X_i) = \mu$, $Var(X_i) = \sigma^2 < \infty$, then
$$
\frac{\sqrt{n}(\bar{X}_n - \mu)}{\sigma} \quad \xrightarrow[]{d} \quad \mc{N}(0,1).
$$
\end{block}
\vspace{0.1in}
\textbf{Remark:}\\
\begin{itemize}
    \item The proof involves characteristic function.
    \item A more generalized CLT is referred to as ``Lindeberg CLT". 
\end{itemize}
\end{frame}


\begin{frame}{Central Limit Theorem}
\textbf{Example:}\\
Suppose $X_i \sim Bernoulli(p)$, i.i.d.,  consider $Z_n = \frac{\sum_{i=1}^n {X_i} - np}{\sqrt{np(1-p)}}$, then by CLT, $Z_n \sim \mc{N}(0,1)$ asymptotically. 
\vspace{1in}

\end{frame}

\begin{frame}{Monotone Convergence Theorem}
    \begin{block}{Monotone Convergence Theorem}
    If $X_n \geq c$ and $X_n \nearrow X$, then $EX_n \nearrow EX$ 
    \end{block}
    \textbf{Usage: }
    \vspace{1.8in}
\end{frame}

\begin{frame}{Dominate Convergence Theorem}
    \begin{block}{Dominated Convergence Theorem}
    If $X_n \to X$ a.s. and $|X_n| \leq Y$ a.s. for all $n$ and $Y$ is integrable, then $EX_n \to EX$ 
    \end{block}
    \textbf{Usage: }
    \vspace{1.8in}
\end{frame}

\begin{frame}{Delta Method}
    \begin{block}{More about CLT: Delta method}
    Suppose $X_n$ are i.i.d. random variables with $EX_n=0, VAR(X_n) = \sigma^2 >0$. Let $g$ be a measurable function that is differentiable at $0$ with $g^\prime(0) \neq 0$. Then
    \[
    \sqrt{n}\left(g\left(\frac{\sum_{k=1}^nX_k}{n} - g(0) \right)\right) \to N(0, \sigma^2g^\prime(0)^2) \quad \text{weakly.}
    \] 
    \end{block}
    \textbf{Proof under stronger assumption: } Here, we suppose $g$ is continuously differentiable on $\mathbb{R}$. If you are interested in a general proof refer to Robert Keener's \textit{Theoretical Statistics}.
    \vspace{1.8in}
\end{frame}


\begin{frame}{Problem Set}
    \textbf{Problem 1:}  Prove that on a complete probability space, if $X_n  \xrightarrow[]{a.s.} X, Y_n  \xrightarrow[]{a.s.} Y$, then $X_n + Y_n \xrightarrow[]{a.s.} X+Y$.\\
    \vspace{0.1in}
    
     \textbf{Problem 2:} Prove that on a complete probability space, if $X_n  \xrightarrow[]{P} X, Y_n  \xrightarrow[]{P} Y$, then $X_n + Y_n \xrightarrow[]{P} X+Y$.
    \vspace{0.1in}\\
    
    \textbf{Problem 3:} A bank teller serves customers standing in the queue one by one. Suppose that the service time $X_i$ for customer $i$ has mean $\mathbb{E}(X_i) = 2$ (minutes) and $Var(X_i) = 1$. We assume that service times for different bank customers are independent. Let $Y$ be the total time the bank teller spends serving 50 customers. Find $\mathbb{P}(90<Y<110)$.
    \vspace{0.1in}\\
   
\end{frame}



\end{document}
