\documentclass [aspectratio=169]{beamer}
\usetheme{Boadilla}
\usepackage{textpos} % package for the positioning
\usepackage[]{graphicx}
\usepackage{graphicx}
\usepackage{float}
\usepackage{hyperref}
\usepackage{caption}
\usepackage{subcaption}
\usepackage{comment}
\usepackage{algorithm,algpseudocode}
\usepackage[export]{adjustbox}
\usepackage{tikz}
\usetikzlibrary{positioning}
\usetikzlibrary{matrix}
\usetikzlibrary{positioning}
\usetikzlibrary{arrows, shapes, decorations, automata, backgrounds, fit, petri, calc}
\newcommand{\tikzmark}[1]{\tikz[overlay,remember picture] \node (#1) {};}
\usepackage{xcolor}
\usepackage{pgfplots}

%\pgfplotsset{compat=1.10}
\usepgfplotslibrary{fillbetween}
\usepackage{filecontents}


\pgfplotsset{compat=1.7}
\usetikzlibrary{positioning}
\usetikzlibrary{arrows, shapes, decorations, automata, backgrounds, fit, petri, calc}
\setbeamertemplate{itemize items}[circle]
\setbeamertemplate{enumerate items}[circle]
\setbeamertemplate{itemize subitem}{$\triangleright$}

\newcommand{\notimplies}{\;\not\!\!\!\implies}
\newcommand*{\logofont}{\fontfamily{phv}\selectfont}
\definecolor{uoftblue}{RGB}{6,41,88} % official blue color for uoft
\definecolor{yamabuki}{RGB}{255,177,27}
\definecolor{sakuranezumi}{RGB}{177,150,147}
\definecolor{enji}{RGB}{159,53,58}
\definecolor{torinoko}{RGB}{218, 201, 166}
\definecolor{baikocha}{RGB}{137,145,107}
\definecolor{ginnezumi}{RGB}{145,152,159}


\vspace{1in}
\title[]{DoSS Summer Bootcamp Probability \\ Module 8}
\author[]{Ichiro Hashimoto}
\institute[]{University of Toronto}
\date{July 25, 2023}

% set color
\setbeamercolor{title in head/foot}{bg=white}
\setbeamercolor{author in head/foot}{bg=white}
\setbeamercolor{date in head/foot}{fg=uoftblue}
\setbeamercolor{date in head/foot}{bg=white}
\setbeamercolor{title}{fg=uoftblue}
\setbeamerfont{title}{series=\bfseries}
\setbeamercolor{frametitle}{fg=uoftblue}
\setbeamerfont{frametitle}{series=\bfseries}
\setbeamercolor*{item}{fg=uoftblue}
\setbeamercolor{block title}{bg=uoftblue}
\setbeamercolor{block title}{fg=white}
\setbeamercolor{block body}{bg=uoftblue!5!white}

% set logo at non-title pages
\logo{\includegraphics[height=0.8cm]{logo_uoft.png}\vspace*{-.055\paperheight}\hspace*{.85\paperwidth}}

% set margin
\setbeamersize{text margin left=10mm,text margin right=10mm}

\newcommand{\mc}{\mathcal}


\begin{filecontents*}{LOFT.txt}
        n   an   a    
        1   1    0   
        2   0.5  0
        5   0.2    0   
        10   0.1    0   
        20   0.05    0   
        40   0.025    0   
        50   0.02    0   
        100   0.01    0   
\end{filecontents*}


\begin{document}
{
\setbeamertemplate{logo}{}
\begin{frame}
    \vspace{0.5in}
    \titlepage
    \begin{textblock*}{4cm}(0.5cm,-7.5cm)
        \includegraphics[width=4cm]{logo_uoft.png}
    \end{textblock*}
    \begin{textblock*}{8cm}(5.0cm,-7cm)
        \huge \color{uoftblue}{$\Bigr\rvert$ \hspace{0.15cm} \textbf{\logofont Statistical Sciences}}
    \end{textblock*}
\end{frame}
}

\begin{frame}{Recap}
Learnt in last module:\\
\vspace{0.1in}
\begin{itemize}
    \item Stochastic convergence
    \begin{itemize}
        \item Convergence in distribution
        \item Convergence in probability
        \item Convergence almost surely
        \item Convergence in $L^p$
        \item Relationship between convergences
    \end{itemize}
\end{itemize}
\end{frame}

\begin{frame}{Outline}
\begin{itemize}
    \item Relationship between convergences and counterexamples
    \begin{itemize}
        \item Monotonicity of $L^p$ Convergence
        \item $L^p$ convergence implies Convergence in Probability
        \item a.s. Convergence implies Convergence in Probability
        \item Convergence in Probability implies Convergence in distribution
     \end{itemize}
\end{itemize}
\end{frame}

\begin{frame}{Relationship between convergences and counterexamples}
    \textbf{Recall:} A random variable $X \in L^p$ if $\|X\|_{L^p} = (E |X|^p)^{1/p} < \infty$.\\ 
    \quad \qquad $X_n\to X$ in $L^p$ if $\lim_{n\to \infty}\|X_n - X\|_{L^p}=0$ \\
    \begin{block}{Monotonicity of $L^p$ Convergence}
    If $q>p>0$, $L^q$ convergence implies $L^p$ convergence 
    \end{block}
    \textbf{Proof: }
    \vspace{1.8in}
\end{frame}

\begin{frame}{Relationship between convergences and counterexamples}
    \textbf{Recall:} A random variable $X \in L^p$ if $\|X\|_{L^p} = (E |X|^p)^{1/p} < \infty$.\\ 
    \quad \qquad $X_n\to X$ in $L^p$ if $\lim_{n\to \infty}\|X_n - X\|_{L^p}=0$ \\
    \begin{block}{Monotonicity of $L^p$ Convergence}
    If $q>p>0$, $L^q$ convergence implies $L^p$ convergence 
    \end{block}
    \textbf{Counterexample to the Converse: }
    \vspace{1.8in}
\end{frame}

\begin{frame}{Relationship between convergences and counterexamples}
    \textbf{Recall:} $X_n$ converges to $X$ in probability if for any $\epsilon>0$ $\lim_{n\to \infty}P(|X_n - X|>\epsilon)=0$.\\
    \begin{block}{$L^p$ convergence implies Convergence in Probability}
    If $X_n \to X$ in $L^p$, then $X_n \to X$ in probability. 
    \end{block}
    \textbf{Proof: }
    \vspace{1.8in}
\end{frame}

\begin{frame}{Relationship between convergences and counterexamples}
    \textbf{Recall:} $X_n$ converges to $X$ in probability if for any $\epsilon>0$ $\lim_{n\to \infty}P(|X_n - X|>\epsilon)=0$.\\
    \begin{block}{$L^p$ convergence implies Convergence in Probability}
    If $X_n \to X$ in $L^p$, then $X_n \to X$ in probability. 
    \end{block}
    \textbf{Counterexample to the Converse: }
    \vspace{1.8in}
\end{frame}

\begin{frame}{Relationship between convergences and counterexamples}
    \textbf{Recall:} $X_n$ converges to $X$ in probability if for any $\epsilon>0$ $\lim_{n\to \infty}P(|X_n - X|>\epsilon)=0$.\\
    \begin{block}{a.s. Convergence implies Convergence in Probability}
    If $X_n \to X$ almost surely, then $X_n \to X$ in probability. 
    \end{block}
    \textbf{Proof: }
    \vspace{1.8in}
\end{frame}

\begin{frame}{Relationship between convergences and counterexamples}
    \textbf{Recall:} $X_n$ converges to $X$ in probability if for any $\epsilon>0$ $\lim_{n\to \infty}P(|X_n - X|>\epsilon)=0$.\\
    \begin{block}{a.s. Convergence implies Convergence in Probability}
    If $X_n \to X$ almost surely, then $X_n \to X$ in probability. 
    \end{block}
    \textbf{Counterexample to the Converse: }
    \vspace{1.8in}
\end{frame}

\begin{frame}{Relationship between convergences and counterexamples}
    \textbf{Recall:} $X_n$ converges to $X$ in distribution if for any continuity point $x$ of $P(X\leq x)$, $\lim_{n\to \infty}P(X_n \leq x) = P(X\leq x)$ holds.\\
    \begin{block}{Convergence in Probability implies Convergence in Distribution}
    If $X_n \to X$ in probability, then $X_n \to X$ in distribution. 
    \end{block}
    \textbf{Proof: }
    \vspace{1.8in}
\end{frame}

\begin{frame}{Relationship between convergences and counterexamples}
    \textbf{Recall:} $X_n$ converges to $X$ in distribution if for any continuity point $x$ of $P(X\leq x)$, $\lim_{n\to \infty}P(X_n \leq x) = P(X\leq x)$ holds.\\
    \begin{block}{Convergence in Probability implies Convergence in Distribution}
    If $X_n \to X$ in probability, then $X_n \to X$ in distribution. 
    \end{block}
    \textbf{Counterexample to the Converse: }
    \vspace{1.8in}
\end{frame}

\begin{frame}{Relationship between convergences and counterexamples}
    \textbf{Recall:} $X_n$ converges to $X$ in distribution if for any continuity point $x$ of $P(X\leq x)$, $\lim_{n\to \infty}P(X_n \leq x) = P(X\leq x)$ holds.\\
    \begin{block}{Convergence in Probability implies Convergence in Distribution}
    If $X_n \to X$ in probability, then $X_n \to X$ in distribution. 
    \end{block}
    \textbf{Special case when the Converse holds: }
    \vspace{1.8in}
\end{frame}

\end{document}
