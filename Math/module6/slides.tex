\documentclass [aspectratio=169]{beamer}
\usetheme{Boadilla}
\usepackage{textpos} % package for the positioning
\usepackage[]{graphicx}
\usepackage{graphicx}
\usepackage{float}
\usepackage{hyperref}
\usepackage{caption}
\usepackage{subcaption}
\usepackage{algorithm,algpseudocode}
\usepackage{amsmath, amsfonts, amsthm, amssymb}
\usepackage{dsfont}
\usepackage[export]{adjustbox}
\usepackage{tikz}
\usetikzlibrary{positioning}
\usetikzlibrary{arrows, shapes, decorations, automata, backgrounds, fit, petri, calc}
\setbeamertemplate{itemize items}[circle]
\setbeamertemplate{enumerate items}[circle]


\newcommand*{\logofont}{\fontfamily{phv}\selectfont}
\definecolor{uoftblue}{RGB}{0,42,92} % official blue color for uoft
\definecolor{deptgreen}{RGB}{114,192,148} 
\definecolor{deptoran}{RGB}{252,103,63} 

\hypersetup{
  colorlinks   = true, %Colours links instead of boxes
  urlcolor     = uoftblue, %Colour for external hyperlinks
  linkcolor    = black, %Colour of internal links
  citecolor   = black %Colour of citations
}


% lin alg
\newcommand{\bu}{{\mathbf{u}}}
\newcommand{\bv}{{\mathbf{v}}}
\newcommand{\bw}{{\mathbf{w}}}
\newcommand{\bx}{\mathbf{x}}
\newcommand{\zerovec}{{\mathbf{0}}}

% other useful stuff
\newcommand{\Id}{{\mathds{1}}}
\newcommand{\R}{{\mathds{R}}}
\newcommand{\C}{{\mathds{C}}}
\newcommand{\F}{{\mathds{F}}}
\newcommand{\Z}{{\mathds{Z}}}



\beamertemplatenavigationsymbolsempty

% block
% example block
% alert block


\title[]{Module 6: Linear Algebra I \\ {\large Operational math bootcamp}\\ \includegraphics[width=7cm]{dept_logo.png}\vspace{-1em}}
\author[]{Emma Kroell}
\institute[]{University of Toronto}
\date{\today}

% set color
\setbeamercolor{title in head/foot}{bg=white}
\setbeamercolor{author in head/foot}{bg=white}
\setbeamercolor{date in head/foot}{fg=uoftblue}
\setbeamercolor{date in head/foot}{bg=white}
\setbeamercolor{title}{fg=uoftblue}
\setbeamerfont{title}{series=\bfseries}
\setbeamercolor{frametitle}{fg=uoftblue}
\setbeamerfont{frametitle}{series=\bfseries}
\setbeamercolor*{item}{fg=uoftblue}
\setbeamercolor{block title}{bg=uoftblue}
\setbeamercolor{block title}{fg=white}
\setbeamercolor{block body}{bg=uoftblue!9!white}
\setbeamercolor{block title example}{bg=deptgreen}
\setbeamercolor{block title example}{fg=white}
\setbeamercolor{block body example}{bg=deptgreen!13!white}
\setbeamercolor{block title alerted}{bg=deptoran}
\setbeamercolor{block title alerted}{fg=white}
\setbeamercolor{block body alerted}{bg=deptoran!10!white}


% set logo at non-title pages
\logo{\includegraphics[height=0.8cm]{dept_logo.png}\vspace*{-.045\paperheight}\hspace*{.78\paperwidth}}

% set margin
\setbeamersize{text margin left=10mm,text margin right=10mm}

\begin{document}
{
\setbeamertemplate{logo}{}
\begin{frame}
    %\vspace{0.5in}
    \titlepage
    %\begin{textblock*}{10cm}(3.5cm,-7.5cm)
      %  \includegraphics[width=8cm]{dept_logo.png}
    %\end{textblock*}
\end{frame}
}

\begin{frame}{Outline}
    \begin{itemize}
    	\item Vector spaces and subspaces
	\item Linear combinations and bases
	\item Linear transformations
    \end{itemize}
\end{frame}

\begin{frame}
\begin{definition}
We call $V$ a \textbf{vector space} if the following hold: \\

\begin{enumerate}
\setlength\itemsep{0.1em}
    \item[(A)] \textit{Commutativity in addition:} $\bu + \bv = \bv + \bu$ for all $\bu, \bv \in V$
    \item[(B)] \textit{Associativity in addition:} $\bu + (\bv + \bw) = (\bu + \bv) + \bw$ for all $\bu, \bv, \bw \in V$
    \item[(C)] \textit{Existence of a neutral element, addition:} There exists a vector $\zerovec$ such that for any $\bv \in V$, $\zerovec + \bv = \bv$
    \item[(D)] \textit{Additive inverse:} For every $\bv \in V$, there exists another vector, which we denote $-\bv$, such that $\bv + (-\bv) = \zerovec$.
\end{enumerate}


\begin{enumerate}
\setlength\itemsep{0.1em}
    \item[(E)] \textit{Existence of a neutral element, multiplication:} For any $\bv \in V$, $1\times \bv = \bv$
    \item[(F)] \textit{Associativity in multiplication:} Let $\alpha, \beta \in \mathbb{F}$. For any $\bv \in V$, $(\alpha \beta) \bv = \alpha (\beta \bv)$ 
\end{enumerate}

\begin{enumerate}
\setlength\itemsep{0.1em}
    \item[(G)] Let $\alpha \in \mathbb{F}, \bu, \bv \in V$. $\alpha (\bu + \bv) = \alpha \bu + \beta \bv$.
    \item[(H)] Let $\alpha, \beta \in \mathbb{F}, \bv \in V$. $(\alpha + \beta) \bv = \alpha \bv + \beta \bv$.
\end{enumerate}
\end{definition}

\end{frame}


\begin{frame}

\begin{definition}
A subset $U$ of $V$ is called a \textbf{subspace} of of $V$ if $U$ is also a vector space (using the same addition and scalar multiplication as on $V$).
\end{definition}


\begin{exampleblock}{Proposition}
A subset $U$ of $V$ is a subspace of $V$ if
and only if $U$ satisfies the following three conditions:
\begin{enumerate}
\item  $\zerovec \in U$
\item Closed under addition: $u,w\in U$ implies $\bu+\bv \in U$
\item Closed under scalar multiplication: $\alpha \in\F$ and $u\in U$
implies $\alpha \bu \in U$
\end{enumerate}
\end{exampleblock}
\end{frame}

\begin{frame}
\begin{proof}
$\Rightarrow$ If $U$ is a subspace of $V$, then $U$ satisfies these 3 properties by the definition of a vector space. \\
$\Leftarrow$ Suppose $U$ satisfies the given 3 conditions. \\
Then for any $\bv \in U$, there must exist $-\bv \in U$ by property 3, since $-\bv = (-1) \times \bv$ (exercise). Property 1 assures property C. Properties 2 and 3, and the fact that $U \subset V$, assure the remaining properties hold. 
\end{proof}
\end{frame}

\begin{frame}{Linear combinations}
\begin{definition}
A linear combination of vectors $\bv_{1},...,\bv_{n}$ of vectors in $V$ is a vector of the form 
$$
\alpha_{1}\bv_{1}+...+\alpha_{n}\bv_{n} = \sum_{k=1}^n \alpha_k \bv_k
$$
 where $\alpha_{1},...,\alpha_{m} \in \F$.
\end{definition}
\end{frame}

\begin{frame}{Span}
\begin{definition}
The set of all linear combinations of a list of vectors
$v_{1},...,v_{m}$ in $V$ is called the \textbf{span} of $v_{1},...,v_{m}$,
denoted span$\{\bv_{1},...,\bv_{n}\}$. In other words, 
$$
\text{span}\{\bv_{1},...,\bv_{n}\}=\{\alpha_{1}\bv_{1}+...+\alpha_{m}\bv_{n} :\alpha_{1},...,\alpha_{n}\in\F\}
$$
\end{definition}
The span of the empty list is defined to be $\{\zerovec\}$.

\vspace{1em}
We say a vector space is \emph{finite dimensional} if it can be spanned by a finite list of vectors; otherwise it is \emph{infinite dimensional}.

\end{frame}


\begin{frame}{Linear independence}
\begin{definition}
A list of vectors $\bv_1, \ldots, \bv_n \in \F^n$ is said to be \emph{linearly independent} if 
$$ 0 = \alpha_i \bv_i + \ldots + \alpha_n \bv_n, $$
where the $\alpha_i$, $i=1,\ldots,n$ are scalars, admits only the solution $\alpha_1 = \cdots = \alpha_n = 0$.
\end{definition}

\vspace{1em} Otherwise we say the vectors are \emph{linearly dependent}.

\end{frame}


\begin{frame}{Basis}
\begin{definition}
A system of vectors $\bv_1, \ldots, \bv_n$ is called a basis (for the vector space $V$ ) if any vector $\bv \in V$ admits a unique representation as a linear combination
$$
\bv = \alpha_1 \bv_1 + \ldots + \alpha_n \bv_n = \sum_{k=1}^n \alpha_k \bv_k
$$
\end{definition}

In undergrad, you likely thought about this as: the equation $\bv = \alpha_1 \bv_1 + \ldots + \alpha_n \bv_n$, where the $\alpha_i$ are unknown, has a unique solution.
\end{frame}

\begin{frame}


\begin{exampleblock}{Claim}
All bases of a vector space $V$ have the same length.
\end{exampleblock}

Proof.

\vspace{4cm}

\begin{definition}
The \emph{dimension} of a vector space $V$, denoted $\text{dim} V$, is the length of any basis of $V$.
\end{definition}


\end{frame}

\begin{frame}{Bases}
Example of bases: \\
For $\R^n$: $e_1 = (1,0,\ldots, 0), \; e_2 = (0,1,0,\ldots,0), \; \ldots, \; e_n = (0, \ldots, 0, 1)$ \\
For $\mathbb{P}^n: \; 1, x, x^2, \ldots, x^n$

\begin{definition}
The linear combination $\alpha_{1}\bv_{1}+...+\alpha_{n}\bv_{n}$ is called trivial if $\alpha_k = 0$ for every $k$.
\end{definition}

\begin{block}{proposition}
 A system of vectors $\bv_1, \ldots \bv_n \in V$ is a basis if and only if it is linearly independent and complete (generating).
\end{block}

\end{frame}


\begin{frame}{Linear transformations}
\begin{definition}
A \textbf{transformation} $T$ from domain $X$ to codomain $Y$ is a rule that assigns an output $y = T(x) \in Y$ to each input $x \in X$
\end{definition}

\begin{definition}
A transformation from a vector space $U$ to a vector space $V$ is \textbf{linear} if
\begin{equation*}
    T(\alpha \bu + \beta \bv) = \alpha T(\bu) + \beta T(\bv) \quad \text{for any } \bu, \bv \in V, \; \alpha, \beta \in \F
\end{equation*}
\end{definition}

\end{frame}


\begin{frame}{Examples}

\begin{itemize}
\item Differentiation
\item Integration
\item Rotation of vectors
\item Reflection of vectors
\end{itemize}


\end{frame}


\begin{frame}{Linear transformations}
\begin{definition}
Let $T:U \to V$ be a linear transformation. We define the following important subspaces:
\begin{itemize}
\item \emph{Kernel} or \emph{Null space}: 
$$\text{Ker} T = \{\bu \in U : T\bu = 0 \}$$
\item \emph{Range} 
$$\text{Range} T = \{\bv \in V : \exists \bu \in U \text{ such that } \bv = T \bu \}$$
\end{itemize}
The dimensions of these spaces are often called the following:
\begin{itemize}
\item \emph{Nullity}
$$\text{Nullity}(T) = \text{dim}(\text{Ker}(T))$$
\item \emph{Rank}
$$\text{Rank}(T) = \text{dim}(\text{Range}(T))$$
\end{itemize}
\end{definition}

\end{frame}


\begin{frame}{Linear transformations}
\begin{exampleblock}{Rank Theorem}
For a matrix $A$ or equivalently a linear transformation $A: \F^n \to \F^m$:
\begin{equation*}
\text{Rank } A = \text{Rank }  A^T 
\end{equation*}
\end{exampleblock}

\begin{exampleblock}{Rank Nullity Theorem}
Let $T:U \to V$ be a linear transformation, where $U$ and $V$ are finite-dimensional vector spaces. Then  
\begin{equation*}
\text{Rank} T + \text{Nullity}  T = \text{dim } U.
\end{equation*}
\end{exampleblock}

\end{frame}



\begin{frame}{References}

Axler S. \textit{Linear Algebra Done Right}. 3rd ed. Undergraduate Texts in Mathematics. Springer, 2015.
Available from: \href{https://link.springer.com/book/10.1007/978-3-319-11080-6}{https://link.springer.com/book/10.1007/978-3-319-11080-6} 
%U of T login:  \href{https://link-springer-com.myaccess.library.utoronto.ca/book/10.1007/978-3-319-11080-6}{https://link-springer-com.myaccess.library.utoronto.ca/book/10.1007/978-3-319-11080-6}

\vspace{1em}


\indent Treil S. \textit{Linear Algebra Done Wrong}. 2017. Available from: \href{https://www.math.brown.edu/streil/papers/LADW/LADW.html}{https://www.math.brown.edu/streil/papers/LADW/LADW.html}
\end{frame}







\end{document}