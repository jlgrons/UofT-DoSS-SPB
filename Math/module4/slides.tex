\documentclass [aspectratio=169]{beamer}
\usetheme{Boadilla}
\usepackage{textpos} % package for the positioning
\usepackage[]{graphicx}
\usepackage{graphicx}
\usepackage{float}
\usepackage{hyperref}
\usepackage{caption}
\usepackage{subcaption}
\usepackage{algorithm,algpseudocode}
\usepackage{amsmath, amsfonts, amsthm, amssymb}
\usepackage{dsfont}
\usepackage[export]{adjustbox}
\usepackage{tikz}
\usetikzlibrary{positioning}
\usetikzlibrary{arrows, shapes, decorations, automata, backgrounds, fit, petri, calc}
\setbeamertemplate{itemize items}[circle]
\setbeamertemplate{enumerate items}[circle]
\usepackage{tikz}
\usepackage{accents}

\newcommand*{\logofont}{\fontfamily{phv}\selectfont}
\definecolor{uoftblue}{RGB}{0,42,92} % official blue color for uoft
\definecolor{deptgreen}{RGB}{114,192,148} 
\definecolor{deptoran}{RGB}{252,103,63} 

\hypersetup{
  colorlinks   = true, %Colours links instead of boxes
  urlcolor     = uoftblue, %Colour for external hyperlinks
  linkcolor    = black, %Colour of internal links
  citecolor   = black %Colour of citations
}

% lin alg
\newcommand{\bu}{{\mathbf{u}}}
\newcommand{\bv}{{\mathbf{v}}}
\newcommand{\bw}{{\mathbf{w}}}
\newcommand{\bx}{\mathbf{x}}
\newcommand{\zerovec}{{\mathbf{0}}}
\newcommand{\inv}{{-1}}


% other useful stuff
\newcommand{\Id}{{\mathds{1}}}
\newcommand{\R}{{\mathbb{R}}}
\newcommand{\C}{{\mathbb{C}}}
\newcommand{\F}{{\mathbb{F}}}
\newcommand{\Z}{{\mathbb{Z}}}
\newcommand{\N}{{\mathbb{N}}}
\newcommand{\Q}{{\mathbb{Q}}}
\newcommand{\cP}{\mathcal{P}}

% set theory
\newcommand{\interior}{\accentset{\circ}}

\newtheorem{proposition}[theorem]{Proposition}


\beamertemplatenavigationsymbolsempty

% block
% example block
% alert block


\title[]{Module 4: Metric Spaces and Sequences II\\ {\large Operational math bootcamp}\\ \includegraphics[width=8cm]{dept_logo.png}\vspace{-1em}}
\author[]{Emma Kroell}
\institute[]{University of Toronto}
\date{July 15, 2022}

% set color
\setbeamercolor{title in head/foot}{bg=white}
\setbeamercolor{author in head/foot}{bg=white}
\setbeamercolor{date in head/foot}{fg=uoftblue}
\setbeamercolor{date in head/foot}{bg=white}
\setbeamercolor{title}{fg=uoftblue}
\setbeamerfont{title}{series=\bfseries}
\setbeamercolor{frametitle}{fg=uoftblue}
\setbeamerfont{frametitle}{series=\bfseries}
\setbeamercolor*{item}{fg=uoftblue}
\setbeamercolor{block title}{bg=uoftblue}
\setbeamercolor{block title}{fg=white}
\setbeamercolor{block body}{bg=uoftblue!9!white}
\setbeamercolor{block title example}{bg=deptgreen}
\setbeamercolor{block title example}{fg=white}
\setbeamercolor{block body example}{bg=deptgreen!13!white}
\setbeamercolor{block title alerted}{bg=deptoran}
\setbeamercolor{block title alerted}{fg=white}
\setbeamercolor{block body alerted}{bg=deptoran!10!white}


% set logo at non-title pages
\logo{\includegraphics[height=0.8cm]{dept_logo.png}\vspace*{-.045\paperheight}\hspace*{.78\paperwidth}}

% set margin
\setbeamersize{text margin left=10mm,text margin right=10mm}

\begin{document}
{
\setbeamertemplate{logo}{}
\begin{frame}
    %\vspace{0.5in}
    \titlepage
    %\begin{textblock*}{10cm}(3.5cm,-7.5cm)
      %  \includegraphics[width=8cm]{dept_logo.png}
    %\end{textblock*}
\end{frame}
}

\begin{frame}{Outline}
    \begin{itemize}
      \setlength\itemsep{1em}
    	\item Sequences
	\begin{itemize}
	\item Cauchy sequences
	\item subsequences
	\end{itemize}
	\item Continuous functions
	\begin{itemize}
	\item Contractions
	\end{itemize}
	\item Equivalence of metrics
    \end{itemize}
\end{frame}


\begin{frame}{Sequences}
\begin{definition}[Sequence]
Let $(X,d)$ be a metric space. A \emph{sequence} is an ordered list of points $x_n$, $n\in\N$, in $X$, denoted $(x_n)_{n \in \N}$. We say that a sequence $(x_n)_{n \in \N}$ \emph{converges} to a point $x \in X$ if 
\vspace{1cm}
%\begin{equation*}
%    \forall \epsilon > 0 \, \exists \, n_\epsilon \in \N \text{ s.t. } d(x_n,x) < \epsilon \text{ for all } n \geq n_\epsilon .
%\end{equation*}

\end{definition}
\end{frame}


\begin{frame}
\begin{exampleblock}{Proposition}
Let $(X, d)$ be a metric space, and let $A \subseteq X$. Then $\overline A$ is equal to the set of points in $X$ which are limits of a sequence in $A$.
\end{exampleblock}

\begin{proof}
\vspace{4cm}
\end{proof}
\end{frame}


\begin{frame}
\begin{block}{Proof continued}
\vspace{3cm}
\end{block}
\vspace{0.75cm}
\begin{alertblock}{Corollary}
\label{cor:closed_converge}
A set $F \subseteq X$, where $(X, d)$ is a metric space, is closed if and only if every sequence in $F$ which converges in $X$ converges to a point in $F$.
\end{alertblock}
\end{frame}


\begin{frame}{Cluster points of a set}
\begin{definition}
Let $(X,d)$ be a metric space and $A \subseteq X$. A point $x \in X$ is a \emph{cluster point} of $A$ (also called accumulation point) if for every $\epsilon >0$, $B_\epsilon(x)$ contains uncountably many points in $A$.
\end{definition}
\end{frame}

\begin{frame}

\begin{exampleblock}{Proposition}
 $x \in X$ is a cluster point of $A \subseteq X$ where $(X,d)$ is a metric space if and only if there exists a sequence of points $x_n \in A$, $n \in \N$, such that $x_n \to x$.
\end{exampleblock}
\begin{proof}
\vspace{4.5cm}
\end{proof}
\end{frame}


\begin{frame}
Combining the previous result with the limit characterization of closure gives the following: 

\begin{alertblock}{Corollary}
For $A \subseteq X$, $(X,d)$ a metric space, we have $$\overline{A} = A \cup \{x \in X : x \text{ is a cluster point of }A \}.$$
\end{alertblock}

\end{frame}


\begin{frame}{Cauchy sequences}
\begin{definition}[Cauchy sequence]
Let $(X,d)$ be a metric space. A sequence denoted $(x_n)_{n \in \N} \in X$ is called a \emph{Cauchy sequence} if
\vspace{1cm}
%\begin{equation*}
%    \forall \epsilon \, \exists \, n_\epsilon \in \N \text{ s.t. } d(x_n,x_m) < \epsilon \text{ for all } n,m \geq n_\epsilon .
%\end{equation*}

\end{definition}

\end{frame}


\begin{frame}
\begin{exampleblock}{Proposition}
\label{prop:converge_means_Cauchy}
Let $(X, d)$ be a metric space, and let $(x_n)_{n\in\N}$ be a convergent sequence in $X$. Then  $(x_n)_{n\in\N}$ is Cauchy.
\end{exampleblock}

\begin{proof}
\vspace{4cm}
\end{proof}

\end{frame}


\begin{frame}
\begin{definition}
A metric space where every Cauchy sequence converges (to a point in the space) is called \emph{complete}.
\end{definition}
\vspace{1cm}

\begin{exampleblock}{Proposition}
Let $(X, d)$ be a metric space, and let $Y\subseteq X$.
\begin{enumerate}
    \item[(i)] If $X$ is complete and if $Y$ is closed in $X$, then $Y$ is complete.
    \item[(ii)] If $Y$ is complete, then it is closed in $X$.
\end{enumerate}
\end{exampleblock}

\end{frame}


\begin{frame}
\begin{proof}
\vspace{6.5cm}
\end{proof}
\end{frame}


\begin{frame}{Subsequences}
\begin{definition}
Let $(x_n)_{n \in \N}$ be a sequence in a metric space $(X,d)$. Let $(n_k)_{k \in \N}$ be a sequence of natural numbers with $n_1 < n_2 < \cdots$. The sequence $(x_{n_k})_{k \in \N}$ is called a \emph{subsequence} of $(x_n)_{n \in \N}$. If $(x_{n_k})_{k \in \N}$ converges to $x \in X$, we call $x$ a \emph{subsequential limit}.
\end{definition}

\vspace{0.5cm}

\begin{example}
$\left((-1)^n\right)_{n \in \N}$ 

\vspace{2cm}

%The sequence $\left((-1)^n\right)_{n \in \N}$ diverges but the subsequences $\left((-1)^{2n}\right)_{n \in \N}$ and $\left((-1)^{2n-1}\right)_{n\in \N}$ converge to subsequential limits 1 and $-1$, respectively.
\end{example}

\end{frame}


\begin{frame}
\begin{exampleblock}{Proposition}
A sequence $(x_n)_{n \in \N}$ in a metric space $(X,d)$ converges to $x \in X$ if and only if every subsequence of $(x_n)_{n \in \N}$  also converges to $x$.
\end{exampleblock}
\begin{proof}
\vspace{3cm}
\end{proof}
\end{frame}


\begin{frame}
\begin{block}{Proof continued}
\vspace{5cm}
\end{block}

\end{frame}


\begin{frame}{Continuity}
\begin{definition}
Let $(X,d_X)$ and $(Y,d_Y)$ be metric spaces, let $x_0 \in X$, and let $f:X\to Y$. $f$ is \emph{continuous} at $x_0$ if for every sequence $(x_n)_{n\in\N}$ in $X$ that converges to $x_0$, we have $\lim_{n\to\infty}f(x_n)=f(x_0)$.

\vspace{1em}

We say that $f$ is continuous if it is continuous at every point in $X$.
\end{definition}
\end{frame}


\begin{frame}
\begin{theorem}
\label{thm:cont_equiv}
Let $(X,d_X)$ and $(Y,d_Y)$ be metric spaces, let $x_0 \in X$, and let $f:X\to Y$. The following are equivalent:
\begin{enumerate}
    \item[(i)] $f$ is continuous at $x_0$
    \item[(ii)] for all $\epsilon>0$, there exists $\delta > 0$ such that $d_Y(f(x),f(x_0))) < \epsilon$ for all $x \in X$ with $d_X(x,x_0) < \delta$
    \item[(iii)] for each $\epsilon>0$, there is $\delta > 0$ such that $B_\delta(x_0) \subseteq f^{-1} (B_\epsilon(f(x_0)))$
\end{enumerate}
\end{theorem}
\end{frame}


\begin{frame}
\small
(i) $f$ is continuous at $x_0$ \\
(ii) for all $\epsilon>0$, there exists $\delta > 0$ such that $d_Y(f(x),f(x_0))) < \epsilon$ for all $x \in X$ with $d_X(x,x_0) < \delta$

\normalsize
\begin{proof}
(i) $\Rightarrow$ (ii) 
\vspace{4cm}

\end{proof}

\end{frame}


\begin{frame}
\small
(i) $f$ is continuous at $x_0$ \\
(ii) for all $\epsilon>0$, there exists $\delta > 0$ such that $d_Y(f(x),f(x_0))) < \epsilon$ for all $x \in X$ with $d_X(x,x_0) < \delta$ \\
(iii) for each $\epsilon>0$, there is $\delta > 0$ such that $B_\delta(x_0) \subseteq f^{-1} (B_\epsilon(f(x_0)))$
\normalsize
\begin{block}{Proof continued}
(ii) $\Rightarrow$ (iii)

\vspace{0.5cm}
(iii) $\Rightarrow$ (i)

\vspace{3cm}
\end{block}
\end{frame}


\begin{frame}
\begin{exampleblock}{Corollary}
Let $(X,d_X)$ and $(Y,d_Y)$ be metric spaces and let $f:X\to Y$. The following are equivalent:
\begin{enumerate}
    \item[(i)] $f$ is continuous
    \item[(ii)] if $U \subseteq Y$ is open, then $f^\inv(U)$ is open
    \item[(iii)] if $F \subseteq Y$ is closed, then $f^\inv(F)$ is closed
\end{enumerate}
\end{exampleblock}

\end{frame}


\begin{frame}
We need the following results about sets and functions: \\
Let $X$ and $Y$ be sets and $f:X \to Y$. Let $A,B \subseteq Y$. Then 
\begin{enumerate}
\item $f^\inv(A) \subseteq f^\inv(B)$
\item $f^\inv(Y \setminus A) = X \setminus f^\inv(A)$
\end{enumerate}

\begin{proof}
Let $(X,d_X)$ and $(Y,d_Y)$ be metric spaces and let $f:X\to Y$.

(i) $\Rightarrow$ (ii): 

\vspace{3cm}


\end{proof}

\end{frame}


\begin{frame}

\begin{block}{Proof continued}
(ii) $\Rightarrow$ (i)

\vspace{3cm}
(ii) $\Rightarrow$ (iii)

\vspace{1cm}
(iii) $\Rightarrow$ (ii) 
\vspace{0.5cm}
\end{block}
\end{frame}



\begin{frame}
\begin{definition}
Let $(X,d_X)$ and $(Y,d_Y)$ be metric spaces and let $f:X\to Y$. 
\begin{itemize}
    \item $f$ is \emph{uniformly continuous} if for all $\epsilon>0$, there exists $\delta > 0$ such that for every $x_1,x_2\in X$ with $d_X(x_1,x_2) < \delta$, we have  $d_Y(f(x_1),f(x_2))) < \epsilon$ 
    \item $f$ is \emph{Lipschitz continuous} if there exists a $K > 0$ such that for every $x_1,x_2\in X$ we have  $d_Y(f(x_1),f(x_2))) \leq K d_X(x_1,x_2)$
\end{itemize}
\end{definition}

\begin{exampleblock}{Proposition}
Let $(X,d_X)$ and $(Y,d_Y)$ be metric spaces and let $f:X\to Y$. 
$$f \text{ is Lipschitz continuous } \Rightarrow \text{ f is uniformly continuous } \Rightarrow \text{ f is continuous}$$
\end{exampleblock}
Proof is one of your exercises.
\end{frame}


\begin{frame}{Contraction Mapping Theorem}
\begin{definition}
Let $(X,d)$ be a metric space and let $f:X \to X$. We say that $x^* \in X$ is a \emph{fixed point} of $f$ if $f(x^*) = x^*$.
\end{definition}

\begin{definition}
Let $(X,d)$ be a metric space and let $f:X \to X$. $f$ is a \emph{contraction} if there exists a constant $k \in [0,1)$ such that for all $x,y \in X$, $d(f(x),f(y))) \leq k d(x,y)$.
\end{definition}

Observe that a function is a contraction if and only if it is Lipschitz continuous with constant $K < 1$.

\begin{theorem}[Contraction Mapping Theorem]
Suppose that $f : X \to X$ is a contraction and the metric space $X$ is complete. Then $f$ has a unique fixed point $x^*$.
\end{theorem}
\end{frame}

\begin{frame}

\begin{example}
Let $f:\left[-\frac{1}{3},\frac{1}{3}\right] \to \left[-\frac{1}{3},\frac{1}{3}\right]$ be defined by the mapping $x \mapsto x^2$. Assume we use the standard Euclidean metric, $d(x,y) = |x-y|$. $f$ has a unique fixed point because

\vspace{3cm}
\end{example}

\end{frame}


\begin{frame}{Equivalent metrics}
\begin{definition}[Equivalent metrics]
Two metrics $d_1$ and $d_2$ on a set $X$ are \emph{equivalent} if the identity maps from $(X,d_1)$ to $(X,d_2)$ and from $(X,d_2)$ to $(X,d_1)$ are continuous. 
\end{definition}


\begin{exampleblock}{Proposition}
Two metrics $d_1$, $d_2$ on a set $X$ are equivalent if and only if they have the same open sets or the same closed sets.
\end{exampleblock}
\end{frame}


\begin{frame}

\begin{definition}
Two metrics $d_1$ and $d_2$ on a set $X$ are \emph{strongly equivalent} if for every $x,y\in X$, there exists constants $\alpha>0$ and $\beta>0$ such
\begin{equation*}
    \alpha d_1(x,y) \leq d_2(x,y) \leq \beta d_1(x,y).
\end{equation*}
\end{definition}

If two metrics are strongly equivalent then they are equivalent. The proof of this is one of the exercises. 
\end{frame}

\begin{frame}
\begin{example}
We show that the Euclidean distance (induced by 2-norm) and the metric induced by the $\infty$-norm are equivalent on $\R^n$. 
\vspace{5.5cm}
\end{example}
\end{frame}


\begin{frame}{References}
 
Charles C. Pugh (2015). \textit{Real Mathematical Analysis}. Undergraduate Texts in Mathematics. \href{https://link-springer-com.myaccess.library.utoronto.ca/book/10.1007/978-3-319-17771-7}{https://link-springer-com.myaccess.library.utoronto.ca/book/10.1007/978-3-319-17771-7}

\vspace{1em}

Runde ,Volker (2005). \textit{A Taste of Topology}. Universitext.  url:  \href{https://link.springer.com/book/10.1007/0-387-28387-0}{https://link.springer.com/book/10.1007/0-387-28387-0} 

\vspace{1em}

Zwiernik, Piotr (2022). \textit{Lecture notes in Mathematics for Economics and Statistics}. url: \href{http://84.89.132.1/~piotr/docs/RealAnalysisNotes.pdf}{http://84.89.132.1/~piotr/docs/RealAnalysisNotes.pdf} \\

\end{frame}




\end{document}