\documentclass{article}
\usepackage[utf8]{inputenc}
\usepackage[margin=1in]{geometry}
\usepackage{hyperref}
\usepackage{setspace}
\pagenumbering{arabic}
\usepackage{graphicx}
\usepackage[dvipsnames]{xcolor}
\usepackage{fancyhdr} 
\usepackage{amsmath, amsfonts, amsthm, amssymb}
\usepackage{bbm}
\usepackage{nth}
\usepackage{dsfont}

\hypersetup{
  colorlinks   = true, %Colours links instead of boxes
  urlcolor     = black, %Colour for external hyperlinks
  linkcolor    = black, %Colour of internal links
  citecolor   = black %Colour of citations
}

\allowdisplaybreaks % fixes align environment weird spacing on page
\setlength{\parindent}{0cm}


%For references using Lemma, Theorem, etc, use \cref
\usepackage[nameinlink,capitalize,sort]{cleveref}

% theorems
\newtheorem{theorem}{Theorem}[section]
\newtheorem{lemma}[theorem]{Lemma}
\newtheorem{definition}[theorem]{Definition}
\newtheorem{proposition}[theorem]{Proposition}
\newtheorem{example}[theorem]{Example}
\newtheorem{corollary}[theorem]{Corollary}
%theoremstyle{plain} %boldface title, italicized body. Commonly used in theorems, lemmas, corollaries, propositions and conjectures.
%\theoremstyle{definition} %boldface title, Roman body. Commonly used in definitions, conditions, problems and examples
\theoremstyle{remark} %italicized title, Roman body. Commonly used in remarks, notes, annotations, claims, cases, acknowledgments and conclusions.
\newtheorem{exercise}[theorem]{Exercise}

\newenvironment{solution}
  {\renewcommand\qedsymbol{$\blacksquare$}\begin{proof}[Solution]}
  {\end{proof}}

% weird hack to get rid of dot:
\usepackage{xpatch}
\makeatletter
\AtBeginDocument{\xpatchcmd{\@thm}{\thm@headpunct{.}}{\thm@headpunct{}}{}{}}



% lin alg
\newcommand{\bu}{{\mathbf{u}}}
\newcommand{\bv}{{\mathbf{v}}}
\newcommand{\bw}{{\mathbf{w}}}
\newcommand{\bx}{\mathbf{x}}
\newcommand{\zerovec}{{\mathbf{0}}}

% other useful stuff
\newcommand{\Id}{{\mathds{1}}}
\newcommand{\R}{{\mathbb{R}}}
\newcommand{\C}{{\mathbb{C}}}
\newcommand{\Z}{{\mathbb{Z}}}
\newcommand{\N}{{\mathbb{N}}}
\newcommand{\Q}{{\mathbb{Q}}}
\newcommand{\F}{{\mathbb{F}}}
\newcommand{\cL}{{\mathcal{L}}}
\newcommand{\cP}{\mathcal{P}}
\newcommand{\cT}{\mathcal{T}}
\newcommand{\inv}{{-1}}

%\DeclareMathOperator{\dim}{dim}
\DeclareMathOperator{\range}{range}
\DeclareMathOperator{\rank}{rank}
\DeclareMathOperator{\nullity}{nullity}


\begin{document}
\begin{center}
\Large{Exercises for Module 2: Set Theory}
\end{center}

1 Let $A = \{x\in \R : x <100\}$, $B = \{x\in \Z : |x| \geq 20\}$, and $C = \{y \in \N : y \text{ is prime}\}$ ($A,B,C \subseteq \R$). Find $A \cap B$, $B^c \cap C$, $B \cup C$, and $(A \cup B )^c$.


\vspace{11cm} % delete this

% \begin{proof}  
% ADD your content 
% \end{proof}  


2. Is $\R \times \R$ with the ordering $(x_1,y_1) \preceq (x_2,y_2)$ if $x_1 \leq x_2$ a partially ordered set? 

\vspace{10cm} % delete this

% \begin{proof}  
% ADD your content 
% \end{proof}  


3. Let $S$ be a non-empty set. A relation $R$ on $S$ is called an equivalence relation if it is
    \begin{enumerate}
        \item[(i)] Reflexive: $(x,x) \in R$ for all $x \in S$
        \item[(ii)] Symmetric: if $(x,y) \in R$  then $(y,x) \in R$ for all $x,y \in S$
        \item[(iii)] Transitive: if $(x,y), (y,z) \in R$ then $(x,z) \in R$ for all $x,y,z \in S$
    \end{enumerate}
Given $x \in S$, the equivalence class of $x$ (with respect to a given equivalence relation $R$) is defined to consist of those $y \in S$ for which $(x,y) \in R$. Show that two equivalence classes are either disjoint or identical.

\vspace{13cm} % delete this

% \begin{proof}  
% ADD your content 
% \end{proof}  

\newpage
4. Let $(X, \leq)$ be a partially ordered set and $S\subseteq X$ be bounded. Show that the infimum and supremum of $S$ are unique (if they exist).

\vspace{11cm} % delete this

% \begin{proof}  
% ADD your content 
% \end{proof}  

\newpage
5. Let $S,T \subseteq \R$ and suppose both are bounded above. Define $S+T = \{s + t \; \colon \;  \, s\in S,t\in T\}$. Show that $S+T$ is bounded above and $\sup(S+T) = \sup S + \sup T$. 

\vspace{11cm} % delete this

% \begin{proof}  
% ADD your content 
% \end{proof}  

6. Let $f: X \to Y$, $X,Y \subseteq \R$, be defined by the map $x \mapsto \sin(x)$. For what choices of $X$ and $Y$ is $f$ injective, surjective, bijective, or neither?

\vspace{12cm} % delete this

% \begin{proof}  
% ADD your content 
% \end{proof}  



\end{document}


