\documentclass{article}
\usepackage[utf8]{inputenc}
\usepackage[margin=1in]{geometry}
\usepackage{hyperref}
\usepackage{setspace}
\pagenumbering{arabic}
\usepackage{graphicx}
\usepackage[dvipsnames]{xcolor}
\usepackage{fancyhdr} 
\usepackage{amsmath, amsfonts, amsthm, amssymb}
\usepackage{bbm}
\usepackage{nth}
\usepackage{dsfont}
\usepackage{subfig}
\usepackage{tikz}
\usepackage{accents}


% commenting
\newcommand{\comment}[3]{\textcolor{#1}{\textbf{[#2: }\textit{#3}\textbf{]}}}
\newcommand{\emma}[1]{\comment{purple}{EK}{#1}}
\newcommand{\jesse}[1]{\comment{BurntOrange}{JG}{#1}}
\newcommand{\miaoshiqi}[1]{\comment{ForestGreen}{ML}{#1}}
\newcommand{\siyue}[1]{\comment{blue}{SY}{#1}}

% theorems
\newtheorem{theorem}{Theorem}[section]
\newtheorem{lemma}[theorem]{Lemma}
\newtheorem{definition}[theorem]{Definition}
\newtheorem{proposition}[theorem]{Proposition}
\newtheorem{example}[theorem]{Example}
%theoremstyle{plain} %boldface title, italicized body. Commonly used in theorems, lemmas, corollaries, propositions and conjectures.
%\theoremstyle{definition} %boldface title, Roman body. Commonly used in definitions, conditions, problems and examples
\theoremstyle{remark} %italicized title, Roman body. Commonly used in remarks, notes, annotations, claims, cases, acknowledgments and conclusions.
\newtheorem{exercise}[theorem]{Exercise}
% weird hack to get rid of dot:
\usepackage{xpatch}
\makeatletter
\AtBeginDocument{\xpatchcmd{\@thm}{\thm@headpunct{.}}{\thm@headpunct{}}{}{}}

% lin alg
\newcommand{\bu}{{\mathbf{u}}}
\newcommand{\bv}{{\mathbf{v}}}
\newcommand{\bw}{{\mathbf{w}}}
\newcommand{\zerovec}{{\mathbf{0}}}

% other useful stuff
\newcommand{\Id}{{\mathds{1}}}
\newcommand{\R}{{\mathds{R}}}
\newcommand{\C}{{\mathds{C}}}
\newcommand{\F}{{\mathds{F}}}



\hypersetup{
  colorlinks   = true, %Colours links instead of boxes
  urlcolor     = blue, %Colour for external hyperlinks
  linkcolor    = black, %Colour of internal links
  citecolor   = black %Colour of citations
}

\allowdisplaybreaks % fixes align environment weird spacing on page
\setlength{\parindent}{0cm}

% get rid of dot after theorem 
\usepackage{xpatch}
\makeatletter
% Patch to accommodate for \begin{theorem}[...]
\AtBeginDocument{\xpatchcmd{\cref@thmoptarg}{\thm@headpunct{.}}{\thm@headpunct{}}{}{}}
% Patch to accommodate \begin{theorem} (without an optional argument)
\AtBeginDocument{\xpatchcmd{\cref@thmnoarg}{\thm@headpunct{.}}{\thm@headpunct{}}{}{}}


% \usepackage[natbib=true, style=vancouver]{biblatex}
 \usepackage[backend= biber, style=alphabetic]{biblatex}
\bibliography{references.bib}

\title{Mathematics Bootcamp Lecture Notes \\
\vspace{0.5em}
\large Department of Statistical Sciences, University of Toronto}
\author{Emma Kroell}
\date{Last updated: \today}

\begin{document}

\maketitle


\newpage
\section*{Preface}

These lecture notes were prepared for the Mathematics course at the inaugural Department of Statistical Sciences Graduate Student Bootcamp at the University of Toronto. The course taught an overview of necessary mathematics prerequisites to incoming statistics graduate students, with an emphasis on proofs.

\vspace{1em}

These lectures are based on the following books or lecture notes:

\vspace{1em}

1. \href{https://link-springer-com.myaccess.library.utoronto.ca/book/10.1007/978-1-4614-4265-3}{{\emph{An Introduction to Mathematical Structures and Proofs}}} by Larry J. Gerstein

2. \href{https://link-springer-com.myaccess.library.utoronto.ca/book/10.1007/0-387-28387-0}{\emph{A Taste of Topology}} by Volker Runde

3. \href{https://link-springer-com.myaccess.library.utoronto.ca/book/10.1007/978-3-319-11080-6}{{\emph{Linear Algebra Done Right}}} by Sheldon Axler

4. \href{https://www.math.brown.edu/streil/papers/LADW/LADW.html}{{\emph{Linear Algebra Done Wrong}}} by Sergei Treil

5. \href{https://digitalcommons.trinity.edu/mono/7/}{{\emph{Introduction to Real Analysis}}} by William F. Trench

6. \href{https://link-springer-com.myaccess.library.utoronto.ca/book/10.1007/978-3-319-17771-7}{\emph{Real Mathematical Analysis}} by Charles C. Pugh

7. \href{http://84.89.132.1/~piotr/docs/RealAnalysisNotes.pdf}{\emph{Lecture notes in Mathematics for Economics and Statistics}} by Piotr Zwiernik

8. \href{http://www.math.uwaterloo.ca/~lwmarcou/notes/pmath351.pdf}{{\emph{Real Analysis Lecture Notes}}} by Laurent Marcoux

\vspace{1em}

Chapter 1 of Gerstein (2012) is used as reference for the proof technique section. Runde (2005) is the main text for the sections on set theory, metric spaces, and topology, which follow chapters 1, 2, and 3 of his book, respectively. The linear algebra content comes mostly from Axler (2015), with Treil (2017) used in some sections for an alternate perspective.

\vspace{1em}

Most of the material in these notes belongs to these texts. All of these texts are available online to University of Toronto users (some to everyone).

\vspace{1em}

I would like to acknowledge the assistance of Jesse Gronsbell, Stanislav Volgushev, Piotr Zwiernik, and Robert Zimmerman in developing the list of topics for the course. 

\newpage
\tableofcontents


\newpage
\section{Review of proof techniques}
\subsection{Propositional logic}

{\bf Propositions} are statements that could be true or false. They have a corresponding {\bf truth value}.We will use capital letters to denote propositions. 

\vspace{1em}

ex. ``$n$ is odd'' and ``$n$ is divisible by 2'' are propositions . 

Let's call them $P$ and $Q$. Whether they are true or not (i.e. their truth value) depends on what $n$ is. 

\vspace{1em}

We can  negate statements: $\neg P$ is the statement ``$n$ is not odd''

\vspace{1em}
 We can combine statements: 
 \begin{itemize}
 \item $P \wedge Q$ is the statement ``$n$ is odd and $n$ is divisible by 2''.
 \item $P \vee Q$ is the statement ``$n$ is odd or $n$ is divisible by 2''. We always assume the inclusive or unless specifically stated otherwise.
\end{itemize}

Examples:
\begin{itemize}
              \item If it's not raining, I won't bring my umbrella.
              \item I'm a banana or Toronto is in Canada.
              \item If I pass this exam, I'll be both happy and surprised.
\end{itemize}


\subsubsection{Truth values}

\begin{example} Write the following using propositional logic:
If it is snowing, then it is cold out. \\
It is snowing. \\
Therefore, it is cold out.  
\end{example}

\begin{solution}
$P \implies Q$ \\
$P$ \\
Conclusion: $Q$ \\
\end{solution}


To examine if statement is true or not, we use a truth table


\begin{tabular}{|c|c| c|}
\hline
     $P$& $Q$ &  $P \implies Q$ \\ \hline
     T& T & T \\ \hline
     T & F & F \\ \hline
     F & T & T \\ \hline
     F & F & T \\ \hline
\end{tabular}



\subsubsection{Logical equivalence}

\begin{tabular}{|c|c| c|}
\hline
     $P$& $Q$ &  $P \implies Q$ \\ \hline
     T& T & T \\ \hline
     T & F & F \\ \hline
     F & T & T \\ \hline
     F & F & T \\ \hline
\end{tabular} \hspace{2cm} \begin{tabular}{|c | c | c | c|}
\hline
     $P$& $Q$ & $\neg P$ & $\neg P \vee Q$  \\ \hline
     T& T & F & T \\ \hline
     T & F & F & F \\ \hline
     F & T &  T &T \\ \hline
     F & F & T & T \\ \hline
\end{tabular}


What is $\neg (P \implies Q)$?

\subsubsection{Quantifiers}
There are two important logical operators that we have not yet discussed. They are the following symbols: $\forall$, read as ``for all'' or ``for each'', and $\exists$, read as ``there exists''. We will explore their meanings, how they can help us simplify statements we need to prove, and how we prove such statements.

\vspace{1em}

\textbf{For all}

``for all'', $\forall$, is also called the universal quantifier. If $P(x)$ is some property that applies to $x$ from some domain, then $\forall x P(x)$ means that the property $P$ holds for every $x$ in the domain. An example is the statement ``Every real number has a non-negative square.'' We write this as $\forall x \in \R, \, x^2 \geq 0$. In logic, people use brackets to separate parts of the logical expression, ex. $(\forall x \in \R) (x^2 \geq 0)$.

\vspace{1em}

How do we prove a for all statement? We need to take an arbitrary element of the set, and show the property holds for that element.

\vspace{1em}

\textbf{There exists}

``there exists'', $\exists$, is also called the existential quantifier. If $P(x)$ is some property that applies to $x$ from some domain, then $\exists x P(x)$ means that the property $P$ holds for some $x$ in the domain. An example is the statement that 4 has a square root in the reals. We write this as $\exists x \in \R$ such that $x^2 = 4$ or in proper logic notation as $(\exists x \in \R$) ($x^2 = 4)$.

\vspace{1em}

How do we prove a there exists statement? We need to find an element of the set for which the property holds (find an example).

\vspace{1em}

There is also a special way of writing when there exists a unique element. We use $\exists!$ for this case. For example, the statement ``there exists a unique positive integer square root of 64'' is written $\exists!  z \in \N$ such that $z^2 = 64$.


\vspace{1em}

\textbf{Combining quantifiers}

Often we will need to prove statements where we combine quantifiers.

Here are some examples:
\vspace{1em}


\begin{tabular}{p{0.45\textwidth} p{0.45\textwidth}}
     Statement & Logical expression \\
     \hline
     Every non-zero rational number has a multiplicative inverse & $\forall q \in \Q \setminus \{0\}, \, \exists s \in \Q$ such that $qs=1$ \\
     Each integer has a unique additive inverse & $\forall x \in \Z \, , \exists ! y \in \Z$ such that $x+y = 0$ \\
     $f:\R \to \R$ is continuous at $x_0\in\R$ \ &  $\forall \epsilon >0 \; \exists \delta > 0$ such that whenever $|x - x_0| < \delta$, $|f(x)-f(x_0)| < \epsilon$
\end{tabular}

\vspace{1em}


The order of quantifiers is important! Changing the order changes the meaning. Consider the following example. Which are true? Which are false?

\begin{center}

$\forall x \in \R \, \forall y \in \R$  $x + y = 2$ \\
$\forall x \in \R \, \exists y \in \R$  $x + y = 2$ \\
$\exists x \in \R \, \forall y \in \R$  $x + y = 2$ \\
$\exists x \in \R \, \exists y \in \R$ $x + y = 2$ 
\end{center}
\vspace{1em}

It's also important to know how to negate logical statements that include quantifiers, as it will often help us prove or disprove the statements. The results are intuitive, but things can get complicated when we have more complex statements. The negation of a statement being true for all $x$ is that is isn't true for at least one $x$. The negation of a statement being true for at least one $x$ is that is isn't true for any $x$. 

In summary,

\begin{center}
    $\neg \forall x P(x)$ = $\exists \neg P(X)$ \\
$\neg \exists x P(x)$ = $\forall \neg P(X)$
\end{center}

\vspace{1em}

The negations of the statements above are (note that we use De Morgan's laws, as well as the negation of an if, then statement). 

\begin{tabular}{p{0.45\textwidth} p{0.45\textwidth}}
     Logical expression & Negation \\
     \hline
     $\forall q \in \Q \setminus \{0\}, \, \exists s \in \Q$ such that $qs=1$ & $\exists q \in \Q \setminus \{0\}$ such that  $\forall s \in \Q, \, qs \neq 1$\\
     $\forall x \in \Z \, , \exists ! y \in \Z$ such that $x+y = 0$ & $\exists x \in \Z$ such  that $(\forall y \in \Z, x+y \neq 0)$ $\vee$ $(\exists y_1, y_2 \in \Z$ such that $y_1 \neq y_2 \wedge x+y_1 = 0 \wedge x+y_2 = 0$  ) \\
     $\forall \epsilon >0 \; \exists \delta > 0$ such that whenever $|x - x_0| < \delta$, $|f(x)-f(x_0)| < \epsilon$ & $\exists \epsilon >0$ such that $\forall \delta > 0$,  $|x - x_0| < \delta$ and  $|f(x)-f(x_0)| \geq \epsilon$
\end{tabular}



\vspace{1em}

What do these mean in English?



\subsection{Types of proof}

\subsubsection{Direct Proof}

{\bf Approach:} Use the definition and known results.
\vspace{1em}

\begin{example}
The product of an even number with another integer is even.
\end{example}

Approach: use the definition of even.

\begin{definition}
We say that an integer $n$ is {\bf even} if there exists another integer $j$ such that $n=2j$. \\
We say that an integer $n$ is {\bf odd} if there exists another integer $j$ such that $n=2j+1$.
\end{definition}

\begin{proof}
Let $n, m \in \Z$, with $n$ even. By definition, there $\exists$ $j \in \Z$ such that $n = 2j$. Then 
$$ n m  =  (2 j) m = 2 (j m)$$
Therefore $n m$ is even by definition. 
\end{proof}


\begin{definition}
Let $a,b \in \Z$. We say that ``a divides b'', written $a | b$, if the remainder is zero when $b$ is divided by $a$, i.e. $\exists j \in \Z$ such that $b = a j$.
\end{definition}

% https://hsm.stackexchange.com/questions/5656/who-invented-the-divisibility-symbol-and-why-is-it-backwards

\begin{example}
Let $a,b,c \in \Z$ with $a \neq 0$. Prove that if $a | b$ and $b | c$, then $a | c$.
\end{example}
\begin{proof}
Suppose $a | b$ and $b | c$. Then by definition, there exists $j,k \in \Z$ such that $b = aj$ and $c = kb$. Combining these two equations gives $c = k (aj) = a (kj)$. Thus $a | c$ by definition.
\end{proof}


\subsubsection{Proof by contrapositive}
\begin{example}
If an integer squared is even, then the integer is itself even.
\end{example}


How would you approach this proof?


$P \implies Q$  \hspace{5cm}  $\neg Q \implies \neg P$

        \vspace{1.5em}
\begin{tabular}{|c|c| c|}
\hline
     $P$& $Q$ &  $P \implies Q$ \\ \hline
     T& T & T \\ \hline
     T & F & F \\ \hline
     F & T & T \\ \hline
     F & F & T \\ \hline
\end{tabular}   \hspace{2cm}  \begin{tabular}{|c | c | c |  c | c |}
\hline
     $P$& $Q$ & $\neg P$ &  $\neg Q$ & $\neg Q \implies \neg P$ \\ \hline
     T& T & F & F & T \\ \hline
     T & F & F &  T & T \\ \hline
     F & T &  T  & F & F \\ \hline
     F & F & T & T & T \\ \hline
\end{tabular}
\vspace{1.5em}


\begin{proof}
We prove the contrapositive. Let $n$ be odd. Then there exists $k \in \Z$ such that $n = 2k + 1$. We compute
$$n^2 = (2k + 1)^2 = 4k^2 + 4k + 1 = 2(2k^2+2k) + 1.$$
Thus $n^2$ is odd.

\end{proof}


\subsubsection{Proof by contradiction}
\begin{example}
The sum of a rational number and an irrational number is irrational.
\end{example}

\begin{proof}
Let $q \in \mathbb{Q}$ and $r \in \mathbb{R} \setminus \mathbb{Q}$.
Suppose in order to derive a contradiction that their sum is rational, i.e. $ r + q = s$ where $s \in \mathbb{Q}$.
But then $r = s - q \in \mathbb{Q}$. Contradiction.
\end{proof}


\subsubsection{Summary}

{\bf In sum, to prove $P \implies Q$:} 

\vspace{1em}

\begin{tabular}{r l}
     Direct proof:  & assume $P$, prove $Q$ \\
     Proof by contrapositive:  & assume $\neg Q$, prove $\neg P$ \\ 
     Proof by contradiction: & assume $P \wedge \neg Q$ and derive something that is impossible \\ 
\end{tabular}

\subsubsection{Induction}

\begin{theorem}[Well-ordering principle for $\mathbb{N}$]
Every nonempty set of natural numbers has a least element.
\end{theorem}

\begin{theorem}[Principle of mathematical induction]
Let $n_0$ be a non-negative integer. Suppose $P$ is a property such that 
\begin{enumerate}
\item(base case) $P(n_0)$ is true 
\item (induction step) For every integer $k \geq n_0$, if $P(k)$ is true, then $P(k+1)$ is true.
\end{enumerate}
Then $P(n)$ is true for every integer $n \geq n_0$
\end{theorem}

Note: Principle of strong mathematical induction: For every integer $k \geq n_0$, if $P(n)$ is true for every $n = n_0, \ldots, k$, then $P(k+1)$ is true.


\begin{example}
$n! > 2^n$ if $n \geq 4$.
\end{example}


\begin{proof}
We prove this by induction on $n$. \\
{\it Base case:} Let $n = 4$. Then $n! = 4! = 24 > 16 = 2^4$. \\
{\it Inductive hypothesis:} Suppose for some $k \geq 4$, $k! > 2^k$. \\
Then
$$(k+1)! = (k+1) k! > (k+1) 2^k > 2 (2^k) = 2^{k+1}.$$
\end{proof}

\begin{example}
Every integer $n \geq 2$ can be written as the product of primes.
\end{example}

\begin{proof}
We prove this by induction on $n$. \\

{\it Base case:} $n = 2$ is prime. \\

{\it Inductive hypothesis:} Suppose for some $k \geq 2$ that one can write every integer $n$ such that $2 \leq n \leq k$ as a product of primes. \\

We must show that we can write $k+1$ as a product of primes. \\
First, if $k+1$ is prime then we are done.  \\

Otherwise, if $k+1$ is not prime, by definition it can be written as a product of some integers $a$, $b$ such that $1 < a,b < k+1$. 
By the induction hypothesis, $a$ and $b$ can both be written as products of primes, so we are done.
\end{proof}


\subsection{Exercises}
\begin{enumerate}
\item Prove De Morgan's Laws for propositions: $\neg (P \wedge Q) = \neg P \vee \neg Q$ and $\neg (P \vee Q) = \neg P \wedge \neg Q$ (Hint: use truth tables).
\item Write the following statements and their negations using logical quantifier notation and then prove or disprove them:
\begin{enumerate}
    \item[(i)] Every odd integer is divisible by three.
    \item [(ii)] For any real number, twice its square plus twice itself plus 6 is greater than or equal to five. \textit{(You may assume knowledge of calculus.)}
    \item[(iii)] Every integer can be written as a unique difference  of two natural numbers.
\end{enumerate}
\item Prove the following statements:
\begin{enumerate}
    \item[(i)] If $a | b$ and $a,n \in \Z_{>0}$ (positive integers), then $a \leq b$.
    \item[(ii)] If $a | b$ and $a | c$, then $a | (x b + y c)$, where $x,y \in \Z$.
    \item[(iii)] Let $a,b,n \in \Z$. If $n$ does not divide the product $ab$, then $n$ does not divide $a$ and $n$ does not divide $b$.
\end{enumerate}
\item Prove that for all integers $n \geq 1$, $3|(2^{2n}-1)$.
\item Prove the Fundamental Theorem of Arithmetic, that every integer $n \geq 2$ has a unique prime factorization (i.e. prove that the prime factorization from the last proof is unique).
\end{enumerate}


\subsection{References}
Most of this content may be found in Chapter 1 of \cite{proofs}, though many of the examples are my own. \cite{toolsreasoning} is also a great resource, but sadly it is not freely available online or at U of T. 



\newpage

\printbibliography


\end{document}
