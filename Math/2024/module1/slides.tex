\documentclass [aspectratio=169]{beamer}
\usetheme{Boadilla}
\usepackage{textpos} % package for the positioning
\usepackage[]{graphicx}
\usepackage{graphicx}
\usepackage{float}
\usepackage{hyperref}
\usepackage{caption}
\usepackage{subcaption}
\usepackage{algorithm,algpseudocode}
\usepackage{amsmath, amsfonts, amsthm, amssymb}
\usepackage{dsfont}
\usepackage[export]{adjustbox}
\usepackage{tikz}
\usetikzlibrary{positioning}
\usetikzlibrary{arrows, shapes, decorations, automata, backgrounds, fit, petri, calc}
\setbeamertemplate{itemize items}[circle]
\setbeamertemplate{enumerate items}[circle]

\newcommand*{\logofont}{\fontfamily{phv}\selectfont}
\definecolor{uoftblue}{RGB}{0,42,92} % official blue color for uoft
\definecolor{deptgreen}{RGB}{114,192,148} 
\definecolor{deptoran}{RGB}{252,103,63} 

\hypersetup{
  colorlinks   = true, %Colours links instead of boxes
  urlcolor     = uoftblue, %Colour for external hyperlinks
  linkcolor    = black, %Colour of internal links
  citecolor   = black %Colour of citations
}

% lin alg
\newcommand{\bu}{{\mathbf{u}}}
\newcommand{\bv}{{\mathbf{v}}}
\newcommand{\bw}{{\mathbf{w}}}
\newcommand{\bx}{\mathbf{x}}
\newcommand{\zerovec}{{\mathbf{0}}}

% other useful stuff
\newcommand{\Id}{{\mathds{1}}}
\newcommand{\R}{{\mathbb{R}}}
\newcommand{\C}{{\mathbb{C}}}
\newcommand{\F}{{\mathbb{F}}}
\newcommand{\Z}{{\mathbb{Z}}}
\newcommand{\N}{{\mathbb{N}}}
\newcommand{\Q}{{\mathbb{Q}}}
\newcommand{\cP}{\mathcal{P}}



\beamertemplatenavigationsymbolsempty

% block
% example block
% alert block


\title[]{Module 1: Proofs \\ {\large Operational math bootcamp}\\ \includegraphics[width=8cm]{dept_logo.png}\vspace{-1em}}
\author[]{Ichiro Hashimoto}
\institute[]{University of Toronto}
\date{July 8, 2024}

% set color
\setbeamercolor{title in head/foot}{bg=white}
\setbeamercolor{author in head/foot}{bg=white}
\setbeamercolor{date in head/foot}{fg=uoftblue}
\setbeamercolor{date in head/foot}{bg=white}
\setbeamercolor{title}{fg=uoftblue}
\setbeamerfont{title}{series=\bfseries}
\setbeamercolor{frametitle}{fg=uoftblue}
\setbeamerfont{frametitle}{series=\bfseries}
\setbeamercolor*{item}{fg=uoftblue}
\setbeamercolor{block title}{bg=uoftblue}
\setbeamercolor{block title}{fg=white}
\setbeamercolor{block body}{bg=uoftblue!9!white}
\setbeamercolor{block title example}{bg=deptgreen}
\setbeamercolor{block title example}{fg=white}
\setbeamercolor{block body example}{bg=deptgreen!13!white}
\setbeamercolor{block title alerted}{bg=deptoran}
\setbeamercolor{block title alerted}{fg=white}
\setbeamercolor{block body alerted}{bg=deptoran!10!white}


% set logo at non-title pages
\logo{\includegraphics[height=0.8cm]{dept_logo.png}\vspace*{-.045\paperheight}\hspace*{.78\paperwidth}}

% set margin
\setbeamersize{text margin left=10mm,text margin right=10mm}

\begin{document}
{
\setbeamertemplate{logo}{}
\begin{frame}
    %\vspace{0.5in}
    \titlepage
    %\begin{textblock*}{10cm}(3.5cm,-7.5cm)
      %  \includegraphics[width=8cm]{dept_logo.png}
    %\end{textblock*}
\end{frame}
}

\begin{frame}{Outline}
    \begin{itemize}
      \setlength\itemsep{1em}
    	\item Logic
        \item Review of Proof Techniques
    \end{itemize}
\end{frame}


\begin{frame}{Propositional logic}{}
{\bf Propositions} are statements that could be true or false. They have a corresponding {\bf truth value}. \\

\vspace{1em}

ex. ``$n$ is odd'' and ``$n$ is divisible by 2'' are propositions . Let's call them $P$ and $Q$. \\

Whether they are true or not depends on what $n$ is. \\

\vspace{1em}

We can  negate statements: $\neg P$ is the statement ``$n$ is not odd''

\vspace{1em}
 We can combine statements: 
 \begin{itemize}
 \item $P \wedge Q$ is the statement:
 \item $P \vee Q$ is the statement: \\
 We always assume the inclusive or unless specifically stated otherwise.
\end{itemize}
\end{frame}

\begin{frame}\frametitle{Examples}
    \begin{columns}
        \column{.5\textwidth}
        \begin{center}
          \begin{tabular}{|c|c|}
\hline
    Symbol & Meaning  \\
    \hline
     capital letters & propositions  \\
     $\implies$ & implies \\
     $\wedge$ & and \\
     $\vee$ & inclusive or \\
     $\neg$ & not \\
     \hline
\end{tabular}
\end{center}
        \column{.5\textwidth}
          \begin{itemize}
          \setlength\itemsep{1em}
              \item If it's not raining, I won't bring my umbrella.
              \item I'm a banana or Toronto is in Canada.
              \item If I pass this exam, I'll be both happy and surprised.
          \end{itemize}
      \end{columns}
\end{frame}

\begin{frame}{Truth values}

\begin{exampleblock}{Example}
If it is snowing, then it is cold out. \\
It is snowing. \\
Therefore, it is cold out.  
\end{exampleblock}

Write this using propositional logic: \\

\vspace{4em}

\vspace{1em}
How do we know if this statement is true or not?
\end{frame}



\begin{frame}{Truth table}
    \begin{columns}
        \column{.5\textwidth}
        \begin{center}
If it is snowing, then it is cold out. 

\vspace{2em}

\textcolor{deptoran}{When is this true or false?}

\end{center}

        \column{.5\textwidth}
        \begin{center}
        $P \implies Q$ \\
        \vspace{1.5em}
        \begin{tabular}{|c|c| c|}
\hline
     $P$& $Q$ &  $P \implies Q$ \\ \hline
     T& T &  \\ \hline
     T & F &  \\ \hline
     F & T & \\ \hline
     F & F &  \\ \hline
\end{tabular}
\end{center}
\end{columns}
\end{frame}


\begin{frame}{Logical equivalence}
    \begin{columns}
        \column{.5\textwidth}
        \begin{center}
        $P \implies Q$ \\
        \vspace{1.5em}
\begin{tabular}{|c|c| c|}
\hline
     $P$& $Q$ &  $P \implies Q$ \\ \hline
     T& T & T \\ \hline
     T & F & F \\ \hline
     F & T & T \\ \hline
     F & F & T \\ \hline
\end{tabular}
\end{center}
        \column{.5\textwidth}
        \begin{center}
        $\neg P \vee Q$ \\
        \vspace{1.5em}
        \begin{tabular}{|c | c | c | c|}
\hline
     $P$& $Q$ & $\neg P$ & $\neg P \vee Q$  \\ \hline
     T& T &  &  \\ \hline
     T & F &  &  \\ \hline
     F & T &   & \\ \hline
     F & F & & \\ \hline
\end{tabular}
\end{center}
\end{columns}
\vspace{2em}
\centering
What is $\neg (P \implies Q)$?
\end{frame}

\begin{frame}{Quantifiers}


\textbf{For all}

\vspace{1em}

``for all'' (also read ``for any"), $\forall$, is also called the universal quantifier.  

\vspace{1em}

If $P(x)$ is some property that applies to $x$ from some domain, then $\forall x P(x)$ means that the property $P$ holds for every $x$ in the domain. 

\vspace{1em}


``Every real number has a non-negative square.'' We write this as 

\vspace{2em}

How do we prove a for all statement? 
\end{frame}

\begin{frame}{Quantifiers}
\textbf{There exists}

``there exists'', $\exists$, is also called the existential quantifier. 


If $P(x)$ is some property that applies to $x$ from some domain, then $\exists x P(x)$ means that the property $P$ holds for some $x$ in the domain. 

\vspace{1em}

4 has a square root in the reals. We write this as

\vspace{2em}

How do we prove a there exists statement?

\vspace{2em}

There is also a special way of writing when there exists a unique element: $\exists!$ .

For example, we write the statement ``there exists a unique positive integer square root of 64'' as

\vspace{2em}
\end{frame}

\begin{frame}{Combining quantifiers}

Often we will need to prove statements where we combine quantifiers.

Here are some examples:
\vspace{1em}


\begin{tabular}{p{0.45\textwidth} p{0.45\textwidth}}
     Statement & Logical expression \\
     \hline
     Every non-zero rational number has a multiplicative inverse &  \textcolor{white}{$\forall q \in \Q \setminus \{0\}, \, \exists s \in \Q$ such that $qs=1$} \\
          & \\
     Each integer has a unique additive inverse &  \textcolor{white}{$\forall x \in \Z \, , \exists ! y \in \Z$ such that $x+y = 0$} \\
          & \\
     $f:\R \to \R$ is continuous at $x_0\in\R$ \ &  \textcolor{white}{$\forall \epsilon >0 \; \exists \delta > 0$ such that whenever $|x - x_0| < \delta$, $|f(x)-f(x_0)| < \epsilon$}
\end{tabular}

\end{frame}

\begin{frame}{Quantifier order \& negation}

The order of quantifiers is important! Changing the order changes the meaning. Consider the following example. Which are true? Which are false?

\begin{center}

$\forall x \in \R \, \forall y \in \R$  $x + y = 2$ \\
$\forall x \in \R \, \exists y \in \R$  $x + y = 2$ \\
$\exists x \in \R \, \forall y \in \R$  $x + y = 2$ \\
$\exists x \in \R \, \exists y \in \R$ $x + y = 2$ 
\end{center}
\vspace{1em}


Negating quantifiers:


\begin{center}
    $\neg \forall x P(x)$ = $\exists x (\neg P(x))$ \\
$\neg \exists x P(x)$ = $\forall x (\neg P(x))$
\end{center}
\end{frame}


\begin{frame}

The negations of the statements above are: \\
 (Note that we use De Morgan's laws, which are in your exercises: \\
 $\neg (P \wedge Q) = \neg P \vee \neg Q$ and $\neg (P \vee Q) = \neg P \wedge \neg Q$.)

\vspace{1em}

\begin{tabular}{p{0.45\textwidth} p{0.45\textwidth}}
     Logical expression & Negation \\
     \hline
     $\forall q \in \Q \setminus \{0\}, \, \exists s \in \Q$ such that $qs=1$ & \textcolor{white}{$\exists q \in \Q \setminus \{0\}$ such that  $\forall s \in \Q, \, qs \neq 1$ space space space space space space space space space space}\\
     $\forall x \in \Z \, , \exists ! y \in \Z$ such that $x+y = 0$ & \textcolor{white}{$\exists x \in \Z$ such  that $(\forall y \in \Z, x+y \neq 0)$ $\vee$ $(\exists y_1, y_2 \in \Z$ such that $y_1 \neq y_2 \wedge x+y_1 = 0 \wedge x+y_2 = 0$  )} \\
     $\forall \epsilon >0 \; \exists \delta > 0$ such that whenever $|x - x_0| < \delta$, $|f(x)-f(x_0)| < \epsilon$ & \textcolor{white}{$\exists \epsilon >0$ such that $\forall \delta > 0$,  $|x - x_0| < \delta$ and  $|f(x)-f(x_0)| \geq \epsilon$}
\end{tabular}

\vspace{1em}

What do these mean in English?
\end{frame}


\begin{frame}{Types of proof}
\begin{itemize}
	\item Direct
	\item Contradiction
	\item Contrapositive
	\item Induction
\end{itemize}
\end{frame}

\begin{frame}{Direct Proof}

{\bf Approach:} Use the definition and known results. \\
\vspace{1em}

\large{\bf Example}

\begin{exampleblock}{Claim}
The product of an even number with another integer is even.
\end{exampleblock}

\vspace{1em}
Approach: use the definition of even.


%We say that an integer $n$ is {\bf even} if there exists another integer $j$ such that $n=2j$.


\end{frame}


\begin{frame}{Direct Proof}

\begin{exampleblock}{Claim}
The product of an even number with another integer is even.
\end{exampleblock}


\begin{alertblock}{Definition}
We say that an integer $n$ is {\bf even} if there exists another integer $j$ such that $n=2j$. \\
We say that an integer $n$ is {\bf odd} if there exists another integer $j$ such that $n=2j+1$.
\end{alertblock}

\textit{Proof.}

\vspace{6em}


\end{frame}

\begin{frame}
\begin{alertblock}{Definition}
Let $a,b \in \Z$. We say that ``a divides b'', written $a | b$, if the remainder is zero when $b$ is divided by $a$, i.e. $\exists j \in \Z$ such that $b = a j$.
\end{alertblock}

% https://hsm.stackexchange.com/questions/5656/who-invented-the-divisibility-symbol-and-why-is-it-backwards

\begin{example}
Let $a,b,c \in \Z$ with $a \neq 0$. Prove that if $a | b$ and $b | c$, then $a | c$.
\end{example}
\textit{Proof.}
\vspace{6em}

\end{frame}



\begin{frame}{}
\begin{exampleblock}{Claim}
If an integer squared is even, then the integer is itself even.
\end{exampleblock}

\vspace{1em}

How would you approach this proof?

\end{frame}



\begin{frame}{Proof by contrapositive}
    \begin{columns}
        \column{.5\textwidth}
        \begin{center}
        $P \implies Q$ \\
        \vspace{1.5em}
\begin{tabular}{|c|c| c|}
\hline
     $P$& $Q$ &  $P \implies Q$ \\ \hline
     T& T & T \\ \hline
     T & F & F \\ \hline
     F & T & T \\ \hline
     F & F & T \\ \hline
\end{tabular}
\end{center}
        \column{.5\textwidth}
        \begin{center}
        $\neg Q \implies \neg P$ \\
        \vspace{1.5em}
        \begin{tabular}{|c | c | c |  c | c |}
\hline
     $P$& $Q$ & $\neg P$ &  $\neg Q$ & $\neg Q \implies \neg P$ \\ \hline
     T& T & F & F &  \\ \hline
     T & F & F &  T & \\ \hline
     F & T &  T  & F & \\ \hline
     F & F & T & T &  \\ \hline
\end{tabular}
\end{center}
\end{columns}
\end{frame}


\begin{frame}{Proof by contrapositive}
\begin{exampleblock}{Claim}
If an integer squared is even, then the integer is itself even.
\end{exampleblock}

\textit{Proof.}
\vspace{9em}


\end{frame}


\begin{frame}{Proof by contradiction}
\begin{exampleblock}{Claim}
The sum of a rational number and an irrational number is irrational.
\end{exampleblock}

\textit{Proof.}
\vspace{9em}
\end{frame}



\begin{frame}{Summary}

{\bf In sum, to prove $P \implies Q$:} \\

\vspace{1em}


\begin{tabular}{r l}
     Direct proof:  & assume $P$, prove $Q$ \\
     Proof by contrapositive:  & assume $\neg Q$, prove $\neg P$ \\ 
     Proof by contradiction: & assume $P \wedge \neg Q$ and derive something that is impossible \\ 
\end{tabular}

\end{frame}


\begin{frame}{Induction}

\begin{block}{Well-ordering principle for $\mathbb{N}$}
Every nonempty set of natural numbers has a least element.
\end{block}

\begin{block}{Principle of mathematical induction}
Let $n_0$ be a non-negative integer. Suppose $P$ is a property such that 
\begin{enumerate}
\item(base case) $P(n_0)$ is true 
\item (induction step) For every integer $k \geq n_0$, if $P(k)$ is true, then $P(k+1)$ is true.
\end{enumerate}
Then $P(n)$ is true for every integer $n \geq n_0$
\end{block}

Note: Principle of strong mathematical induction: For every integer $k \geq n_0$, if $P(n)$ is true for every $n = n_0, \ldots, k$, then $P(k+1)$ is true.
\end{frame}

\begin{frame}
\begin{exampleblock}{Claim}
$n! > 2^n$ if $n \geq 4$ ($n \in \N$).
\end{exampleblock}


\textit{Proof.}
\vspace{12em}
\end{frame}



\begin{frame}
\begin{exampleblock}{Claim}
Every integer $n \geq 2$ can be written as the product of primes.
\end{exampleblock}

\textit{Proof.} We prove this by strong induction on $n$. \\
\vspace{1em}

{\it Base case:} \\ %$n = 2$ is prime. \\
\vspace{2.5em}

{\it Inductive hypothesis:} \\  %Suppose for some $k \geq 2$ that one can write every integer $n$ such that $2 \leq n \leq k$ as a product of primes. \\
\vspace{2.5em}

{\it Inductive step:} \\
\vspace{4em}


\end{frame}




\begin{frame}{References}
Gerstein, Larry J. (2012). \textit{Introduction to Mathematical Structures and Proofs}. Undergraduate Texts in Mathematics. url: \href{https://link.springer.com/book/10.1007/978-1-4614-4265-3}{https://link.springer.com/book/10.1007/978-1-4614-4265-3}

%U of T log-in: \href{https://link-springer-com.myaccess.library.utoronto.ca/book/10.1007/ 978-1-4614-4265-3l}{https://link-springer-com.myaccess.library.utoronto.ca/book/10.1007/ 978-1-4614-4265-3l}

\vspace{1em}

Lakins, Tamara J. (2016). \textit{The Tools of Mathematical Reasoning}. Pure and Applied Undergraduate Texts. 


\end{frame}



\end{document}