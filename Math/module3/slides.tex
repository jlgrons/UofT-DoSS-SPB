\documentclass [aspectratio=169]{beamer}
\usetheme{Boadilla}
\usepackage{textpos} % package for the positioning
\usepackage[]{graphicx}
\usepackage{graphicx}
\usepackage{float}
\usepackage{hyperref}
\usepackage{caption}
\usepackage{subcaption}
\usepackage{algorithm,algpseudocode}
\usepackage{amsmath, amsfonts, amsthm, amssymb}
\usepackage{dsfont}
\usepackage[export]{adjustbox}
\usepackage{tikz}
\usetikzlibrary{positioning}
\usetikzlibrary{arrows, shapes, decorations, automata, backgrounds, fit, petri, calc}
\setbeamertemplate{itemize items}[circle]
\setbeamertemplate{enumerate items}[circle]

\newcommand*{\logofont}{\fontfamily{phv}\selectfont}
\definecolor{uoftblue}{RGB}{0,42,92} % official blue color for uoft
\definecolor{deptgreen}{RGB}{114,192,148} 
\definecolor{deptoran}{RGB}{252,103,63} 

\hypersetup{
  colorlinks   = true, %Colours links instead of boxes
  urlcolor     = uoftblue, %Colour for external hyperlinks
  linkcolor    = black, %Colour of internal links
  citecolor   = black %Colour of citations
}

% lin alg
\newcommand{\bu}{{\mathbf{u}}}
\newcommand{\bv}{{\mathbf{v}}}
\newcommand{\bw}{{\mathbf{w}}}
\newcommand{\bx}{\mathbf{x}}
\newcommand{\zerovec}{{\mathbf{0}}}

% other useful stuff
\newcommand{\Id}{{\mathds{1}}}
\newcommand{\R}{{\mathbb{R}}}
\newcommand{\C}{{\mathbb{C}}}
\newcommand{\F}{{\mathbb{F}}}
\newcommand{\Z}{{\mathbb{Z}}}
\newcommand{\N}{{\mathbb{N}}}
\newcommand{\Q}{{\mathbb{Q}}}
\newcommand{\cP}{\mathcal{P}}

\newtheorem{proposition}[theorem]{Proposition}


\beamertemplatenavigationsymbolsempty

% block
% example block
% alert block


\title[]{Module 3: Metric Spaces and Sequences I \\ {\large Operational math bootcamp}\\ \includegraphics[width=8cm]{dept_logo.png}\vspace{-1em}}
\author[]{Emma Kroell}
\institute[]{University of Toronto}
\date{\today}

% set color
\setbeamercolor{title in head/foot}{bg=white}
\setbeamercolor{author in head/foot}{bg=white}
\setbeamercolor{date in head/foot}{fg=uoftblue}
\setbeamercolor{date in head/foot}{bg=white}
\setbeamercolor{title}{fg=uoftblue}
\setbeamerfont{title}{series=\bfseries}
\setbeamercolor{frametitle}{fg=uoftblue}
\setbeamerfont{frametitle}{series=\bfseries}
\setbeamercolor*{item}{fg=uoftblue}
\setbeamercolor{block title}{bg=uoftblue}
\setbeamercolor{block title}{fg=white}
\setbeamercolor{block body}{bg=uoftblue!9!white}
\setbeamercolor{block title example}{bg=deptgreen}
\setbeamercolor{block title example}{fg=white}
\setbeamercolor{block body example}{bg=deptgreen!13!white}
\setbeamercolor{block title alerted}{bg=deptoran}
\setbeamercolor{block title alerted}{fg=white}
\setbeamercolor{block body alerted}{bg=deptoran!10!white}


% set logo at non-title pages
\logo{\includegraphics[height=0.8cm]{dept_logo.png}\vspace*{-.045\paperheight}\hspace*{.78\paperwidth}}

% set margin
\setbeamersize{text margin left=10mm,text margin right=10mm}

\begin{document}
{
\setbeamertemplate{logo}{}
\begin{frame}
    %\vspace{0.5in}
    \titlepage
    %\begin{textblock*}{10cm}(3.5cm,-7.5cm)
      %  \includegraphics[width=8cm]{dept_logo.png}
    %\end{textblock*}
\end{frame}
}

\begin{frame}{Outline}
    \begin{itemize}
      \setlength\itemsep{1em}
    	\item 
    \end{itemize}
\end{frame}

\begin{frame}
\begin{definition}
Two sets $A$ and $B$ have same cardinality, $|A| = |B|$, if there exists bijection $f:A \to B$.
\end{definition}

\begin{example}
Which is bigger, $\N$ or $\N_0$? \\
\vspace{3cm}
\end{example}
\end{frame}


\begin{frame}{Cantor-Schr\"{o}der-Bernstein}
\begin{definition}
We say that the cardinality of a set $A$ is less than the cardinality of a set $B$, denoted $|A| \leq |B|$ if there exists an injection $f:A \to B$.
\end{definition}

\begin{theorem}[Cantor-Bernstein]
Let $A$, $B$, be sets. If $|A| \leq |B|$ and $|B| \leq |A|$, then $|A| = |B|$.
\end{theorem}

\end{frame}


\begin{frame}
\begin{example}
$|\N| = |\N \times \N|$
\end{example}
\begin{proof}
\vspace{4cm}
\end{proof}
\end{frame}

\begin{frame}
\begin{definition}
Let $A$ be a set. 
\begin{enumerate}
\item $A$ is \emph{finite} if there exists an $n \in \N$ and a bijection $f:\{1,\ldots,n\} \to A$
\item $A$ is \emph{countably infinite} if there exists a bijection $f:\N\to A$
\item $A$ is \emph{countable} if it is finite or countably infinite
\item $A$ is \emph{uncountable} otherwise
\end{enumerate}
\end{definition}
\end{frame}

\begin{frame}
\begin{example}
The rational numbers are countable, and in fact $|\Q| = |\N|$. \\
\vspace{1em}
Let's look at $\Q^+ := \{ x \in \Q : x > 0\}$. The fact that the rationals are countable relies on this famous way of listing the rational numbers:
\begin{equation*}
\scalebox{1.2}{$
 \begin{array}{cccccc}
 1 & \frac{1}{2} & \frac{1}{3} & \frac{1}{4} &\frac{1}{5} &\ldots \\[0.5em]
 2 & \textcolor{red}{\frac{2}{2}} & \frac{2}{3} &\textcolor{red}{\frac{2}{4}} &\frac{2}{5} & \ldots \\[0.5em]
 3 & \frac{3}{2} & \textcolor{red}{\frac{3}{3}} &\frac{3}{4} &\frac{3}{5} &\ldots \\[0.5em]
 4 & \textcolor{red}{\frac{4}{2}} & \frac{4}{3} & \textcolor{red}{\frac{4}{4}} &\frac{4}{5} &\ldots \\[0.5em]
 \vdots & \vdots & \vdots & \vdots &\vdots & \ddots 
\end{array}  $} 
\end{equation*}
\end{example}
\end{frame}

\begin{frame}
\begin{example}
This is a map from $\N$ to $\Q^+$. As long as we skip any fraction that is already in our list as we go along, it is injective.  Since we can find an injection from $\Q^+$ to $\N \times \N$ (exercise), we have that $|\Q^+| = |\N|$.  We can extend this to $\Q$. To do so, let $f\colon \N \to \Q^+$ be a bijection (which exists by the previous part). Then we can define another bijection $g\colon \N \to \Q$ by setting $g(1) = 0$ and 
\begin{equation*}
    g(n) =\begin{cases}
    f(n) & \text{ if } n \text{ is even,} \\
    -f(n) & \text{ if }  n \text{ is odd},
    \end{cases}
\end{equation*}
for $n>1$.

\end{example}
\end{frame}

\begin{frame}
\begin{theorem}
The cardinality of $\N$ is smaller than that of $(0,1)$.
\end{theorem}
\begin{proof}
First, we show that there is an injective map from $\N$ to $(0, 1)$.

\vspace{2em} 

Next, we show that there is no surjective map from $\N$ to (0, 1). We use the fact that every number $r \in (0,1)$ has a binary expansion of the form $r=0.\sigma_1\sigma_2\sigma_3\ldots$ where $\sigma_i \in \{0, 1\}$, $i \in \N$.
\end{proof}
\end{frame}

\begin{frame}
\begin{proof}
Now we suppose in order to derive a contradiction that there does exist a surjective map $f$ from $\N$ to (0, 1)., i.e. for $n \in \N$ we have $f(n) = 0.\sigma_1(n)\sigma_2(n)\sigma_3(n)\ldots$. This means we can list out the binary expansions, for example like
\begin{align*}
f(1)= & 0.\textcolor{red}{0}0000000\ldots \\
f(2)=& 0.1\textcolor{red}{1}11111111\ldots\\
f(3)=& 0.01\textcolor{red}{0}1010101\ldots  \\
f(4)= & 0.101\textcolor{red}{0}101010\ldots  \\
& 
\end{align*}

We will construct a number $\tilde r \in (0,1)$ that is not in the image of $f$. 
\end{proof}
\end{frame}


\begin{frame}
\begin{proof}
Define $\tilde r = 0.\tilde\sigma_1 \tilde\sigma_2 \ldots$, where we define the $n$th entry of $\tilde r$ to be the the opposite of the  $n$th entry of the $n$th item in our list:
\begin{equation*}
    \tilde\sigma_n = \begin{cases} 1 & \text{if } \sigma_n(n) = 0, \\
    0 & \text{if }  \sigma_n(n) = 1.
    \end{cases}
\end{equation*}
Then $\tilde r$ differs from $f(n)$ at least in the $n$th digit of its binary expansion for all $n\in \N$. Hence, $\tilde r\not\in f(\N)$, which is a contradiction to $f$ being surjective. This technique is often referred to as Cantor's diagonal argument. 
\end{proof}
\end{frame}

\begin{frame}
\begin{proposition}
(0,1) and $\R$ have the same cardinality. 
\end{proposition}
\begin{proof}
The map $f:\R \to (0,1)$ defined by $x \mapsto \frac{1}{\pi} \left( \arctan(x) + \frac{\pi}{2} \right)$ is a bijection.
\end{proof}

We have shown that there are different sizes of infinity, as the cardinality of $\N$ is infinite but still smaller than that of $\R$ or $(0,1)$. In fact, we have
$$ |\N| = |\N_0| = |\Z| = |\Q| < | \R|.$$
Because of this, there are special symbols for these two cardinalities: The cardinality of $\N$ is denoted $\aleph_0$, while the cardinality of $\R$ is denoted $\mathfrak{c}$. 

\end{frame}



\begin{frame}{References}

Runde ,Volker (2005). \textit{A Taste of Topology}. Universitext.  url:  \href{https://link.springer.com/book/10.1007/0-387-28387-0}{https://link.springer.com/book/10.1007/0-387-28387-0} 

\vspace{1em}

Zwiernik, Piotr (2022). \textit{Lecture notes in Mathematics for Economics and Statistics}. url: \href{http://84.89.132.1/~piotr/docs/RealAnalysisNotes.pdf}{http://84.89.132.1/~piotr/docs/RealAnalysisNotes.pdf} \\

\end{frame}




\end{document}