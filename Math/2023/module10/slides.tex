\documentclass [aspectratio=169]{beamer}
\usetheme{Boadilla}
\usepackage{textpos} % package for the positioning
\usepackage[]{graphicx}
\usepackage{graphicx}
\usepackage{float}
\usepackage{hyperref}
\usepackage{caption}
\usepackage{subcaption}
\usepackage{algorithm,algpseudocode}
\usepackage{amsmath, amsfonts, amsthm, amssymb}
\usepackage{dsfont}
\usepackage[export]{adjustbox}
\usepackage{tikz}
\usetikzlibrary{positioning}
\usetikzlibrary{arrows, shapes, decorations, automata, backgrounds, fit, petri, calc}
\setbeamertemplate{itemize items}[circle]
\setbeamertemplate{enumerate items}[circle]

\newcommand*{\logofont}{\fontfamily{phv}\selectfont}
\definecolor{uoftblue}{RGB}{0,42,92} % official blue color for uoft
\definecolor{deptgreen}{RGB}{114,192,148} 
\definecolor{deptoran}{RGB}{252,103,63} 

\hypersetup{
  colorlinks   = true, %Colours links instead of boxes
  urlcolor     = uoftblue, %Colour for external hyperlinks
  linkcolor    = black, %Colour of internal links
  citecolor   = black %Colour of citations
}

% lin alg
\newcommand{\bu}{{\mathbf{u}}}
\newcommand{\bv}{{\mathbf{v}}}
\newcommand{\bw}{{\mathbf{w}}}
\newcommand{\bx}{{\mathbf{x}}}
\newcommand{\by}{{\mathbf{y}}}
\newcommand{\bz}{{\mathbf{z}}}
\newcommand{\zerovec}{{\mathbf{0}}}
\newcommand{\innerprod}[1]{\langle #1 \rangle}
\newcommand{\Tr}{\mathrm{Tr}}

% other useful stuff
\newcommand{\Id}{{\mathds{1}}}
\newcommand{\R}{{\mathbb{R}}}
\newcommand{\C}{{\mathbb{C}}}
\newcommand{\Z}{{\mathbb{Z}}}
\newcommand{\N}{{\mathbb{N}}}
\newcommand{\Q}{{\mathbb{Q}}}
\newcommand{\F}{{\mathbb{F}}}
\newcommand{\cL}{{\mathcal{L}}}
\newcommand{\cP}{\mathcal{P}}
\newcommand{\cT}{\mathcal{T}}
\newcommand{\inv}{{-1}}


\beamertemplatenavigationsymbolsempty

\setbeamercolor{coloredboxstuff}{fg=yellow,bg=uoftblue!10!white}
\AtBeginSection[]{
  \begin{frame}
  \vfill
  \centering
  \begin{beamercolorbox}[sep=8pt,center,shadow=true,rounded=true]{coloredboxstuff}
    \usebeamerfont{title}\insertsectionhead\par%
  \end{beamercolorbox}
  \vfill
  \end{frame}
}



% block
% example block
% alert block


\title[]{Module 10: Differentiation and Integration \\ {\large Operational math bootcamp}\\ \includegraphics[width=8cm]{dept_logo.png}\vspace{-1em}}
\author[]{Emma Kroell}
\institute[]{University of Toronto}
\date{July 28, 2023}

% set color
\setbeamercolor{title in head/foot}{bg=white}
\setbeamercolor{author in head/foot}{bg=white}
\setbeamercolor{date in head/foot}{fg=uoftblue}
\setbeamercolor{date in head/foot}{bg=white}
\setbeamercolor{title}{fg=uoftblue}
\setbeamerfont{title}{series=\bfseries}
\setbeamercolor{frametitle}{fg=uoftblue}
\setbeamerfont{frametitle}{series=\bfseries}
\setbeamercolor*{item}{fg=uoftblue}
\setbeamercolor{block title}{bg=uoftblue}
\setbeamercolor{block title}{fg=white}
\setbeamercolor{block body}{bg=uoftblue!9!white}
\setbeamercolor{block title example}{bg=deptgreen}
\setbeamercolor{block title example}{fg=white}
\setbeamercolor{block body example}{bg=deptgreen!13!white}
\setbeamercolor{block title alerted}{bg=deptoran}
\setbeamercolor{block title alerted}{fg=white}
\setbeamercolor{block body alerted}{bg=deptoran!10!white}


% set logo at non-title pages
\logo{\includegraphics[height=0.8cm]{dept_logo.png}\vspace*{-.045\paperheight}\hspace*{.78\paperwidth}}

% set margin
\setbeamersize{text margin left=10mm,text margin right=10mm}

\begin{document}
{
\setbeamertemplate{logo}{}
\begin{frame}
    %\vspace{0.5in}
    \titlepage
    %\begin{textblock*}{10cm}(3.5cm,-7.5cm)
      %  \includegraphics[width=8cm]{dept_logo.png}
    %\end{textblock*}
\end{frame}
}

\begin{frame}{Outline}
    \begin{itemize}
      \setlength\itemsep{0.75em}
    	\item Differentiation on $\R$
	\begin{itemize}
      \setlength\itemsep{0.25em}
	\item Mean value theorem
	\item l'H\^{o}pital's rule
	\item Smoothness classes
    \end{itemize}
	\item Integration on $\R$
	\begin{itemize}
      \setlength\itemsep{0.25em}
	\item Riemann sums and Riemann integral
	\item Integration rules
	\item Drawbacks of Riemann integration
    \end{itemize}
    \end{itemize}
\end{frame}


\begin{frame}{Block matrices}
\begin{definition}
A block matrix is a matrix that can be broken into sections called blocks, which are smaller matrices.
\end{definition}

\begin{example}
$$ \begin{bmatrix} 2 & 1 & 0 & 1  \\ 0 & 2 &1 & 1 \\ 0 & 0 & 1 & 0 \\ 0 & 0 & 2 & 1 \end{bmatrix} \qquad \qquad \qquad \qquad$$
\vspace{2cm}
\end{example}

\end{frame}


\begin{frame}

\begin{definition}
A square matrix is called \emph{block diagonal} if it can be written as a block matrix where the main-diagonal blocks are all square matrices and the off-diagonal blocks are all zero.
\end{definition}

\begin{example}
The matrix
$$ \begin{bmatrix} 2 & 1 & 0 & 0  \\ 0 & 2 &0 & 0 \\ 0 & 0 & 1 & 0 \\ 0 & 0 & 2 & 1 \end{bmatrix} $$
is block diagonal.
\end{example}
\end{frame}



\begin{frame}
\begin{definition}
A vector $\bv$ is called a \emph{generalized eigenvector} of $A$ corresponding to an eigenvalue $\lambda$ if there exists $k \geq 1$ such that 
$$ (A - \lambda I)^k \bv = 0.$$
\end{definition}
The set of generalized eigenvectors of an eigenvalue $\lambda$ (plus $\zerovec$) is called the \emph{generalized eigenspace} of $\lambda$.

\vspace{1em}

\begin{exampleblock}{Proposition}
The algebraic multiplicity of an eigenvalue $\lambda$ is the same as the dimension of the corresponding generalized eigenspace.
\end{exampleblock}

\end{frame}


\begin{frame}
\begin{theorem}[Jordan decomposition theorem]
For any operator $A$ there exists a basis such that $A$ is block diagonal with blocks that have eigenvalues on the diagonal and 1s on the upper off-diagonal. In other words, $A$ can be written in the form
$$ A = \begin{bmatrix} J_1 & 0 & \ldots & 0\\ 0 & J_2 & \ldots & 0 \\ \vdots & \vdots & \ddots & \vdots \\ 0 & 0 & \ldots & J_k \end{bmatrix} $$
where the blocks $J_i$ on the main diagonal are Jordan block of the form 
$$\begin{bmatrix} \lambda  \end{bmatrix}, \begin{bmatrix} \lambda & 1  \\ 0 &\lambda \end{bmatrix}, \begin{bmatrix} \lambda & 1 & 0 \\ 0 &\lambda & 1 \\ 0 & 0 & \lambda \end{bmatrix}, \text{ etc.}$$
This form is called \emph{Jordan canonical form}.
\end{theorem}
\end{frame}


\begin{frame}
Connection to algebraic and geometric multiplicity:
\begin{itemize}
    \item The algebraic multiplicity of an eigenvalue $\lambda$ is the number of times $\lambda$ appears on the diagonal.
    \item The geometric multiplicity of $\lambda$ is the number of Jordan blocks associated with $\lambda$.
\end{itemize}
\vspace{1em}

Why is Jordan form useful?

\vspace{4cm}

\end{frame}


\begin{frame}{Singular value decomposition}
\begin{itemize}
\item $A^T A$ is symmetric
\item therefore it is orthogonally diagonalizable and has real eigenvalues 
\item In fact, the eigenvalues are non-negative (exercise)
\end{itemize}


\begin{definition}
Let $A$ be an $m \times n$ matrix.  Let $\lambda_1, \ldots, \lambda_n$ be the eigenvalues of $A^T A$. Then the \emph{singular values} of $A$ are defined as
\begin{equation*}
    \sigma_1 = \sqrt{\lambda_1}, \ldots, \sigma_n = \sqrt{\lambda_n}.
\end{equation*}
\end{definition}
\end{frame}

\begin{frame}
\begin{theorem}[Singular value decomposition]
If $A$ is an $m \times n$ matrix of rank $k$, then we can write
$$ A = U \Sigma V^T$$
where $\Sigma$ is an $m \times n$ matrix of the form
$$ \begin{bmatrix} D_{k \times k} & 0_{k \times (n - k)} \\ 0_{(m -k) \times k} & 0_{(m -k) \times (n-k)} \end{bmatrix},$$
$D$ is a diagonal matrix with the singular values of $A$, $ \sigma_1, \ldots, \sigma_k$, on the diagonal and $U$ and $V$ are both orthogonal matrices (of size $m \times m$ and $n \times n$, respectively).
\end{theorem}
\end{frame}

\begin{frame}
Uses of SVD:

\vspace{3cm}

Differences between JCF and SVD:

\vspace{3cm}

\end{frame}

\begin{frame}{$LU$-decomposition}
\begin{definition}
The $LU$-decomposition of a square matrix $A$ is the factorization of $A$ into a lower triangular matrix $L$ and an upper triangular matrix $U$ as follows:
\begin{equation*}
    A = LU.
\end{equation*}
\end{definition}

Why is this useful? Consider the linear system $A \bx = \mathbf{b}$
\vspace{3cm}
\end{frame}

\begin{frame}{Recall: orthonormal basis}
\begin{definition}
Two vectors $\bx,\by \in V$ are called \emph{orthogonal} if $\innerprod{\bx,\by} = 0$, denoted $\bx \perp \by$. We call them \emph{orthonormal} if additionally the vectors are normalized, i.e. $\Vert \bx \Vert =\Vert \by\Vert = 1 $. A basis $\bx_1,\ldots, \bx_n$ of $V$ is called \emph{orthonormal basis (ONB)}, if the vectors are pairwise orthogonal and normalized.  
\end{definition}

\vspace{1em}

Starting from a set of linearly independent vectors, we can construct another set of vectors which are orthonormal and span the same space using the \textbf{\textcolor{deptoran}{Gram-Schmidt Algorithm}}.
\end{frame}



\begin{frame}{$QR$-decomposition}
\begin{definition}[$QR$-decomposition]
The $QR$-decomposition of an $m \times n$ matrix $A$ with linearly independent column vectors is the factorization of $A$ as follows:
\begin{equation*}
    A = QR,
\end{equation*}
where $Q$ is an $m \times n$ matrix with orthonormal column vectors and $R$ is an $n\times n$ invertible upper triangular matrix.
\end{definition}
\end{frame}

\begin{frame}
One obtains the factorization by applying the Gram-Schmidt algorithm to the columns of $A$. Let $\bu_1, \ldots, \bu_n$ be the column vectors of $A$. Let $\mathbf{q}_1, \ldots, \mathbf{q}_n$ be the orthonormal vectors obtained by applying Gram Schmidt. Then one can write:
\begin{align*}
    \bu_1 &= \innerprod{\bu_1,\mathbf{q}_1}\mathbf{q}_1 + \innerprod{\bu_2,\mathbf{q}_2}\mathbf{q}_2+ \ldots +  \innerprod{\bu_n,\mathbf{q}_n} \mathbf{q}_n \\
    \bu_2 &= \innerprod{\bu_2,\mathbf{q}_2}\mathbf{q}_2 + \ldots +  \innerprod{\bu_n,\mathbf{q}_n} \mathbf{q}_n \\
    & \vdots \\
    \bu_n &=  \innerprod{\bu_n,\mathbf{q}_n} \mathbf{q}_n
\end{align*}

Thus the orthonormal vectors obtained using Gram-Schmidt form the columns of $Q$, while $R$ is the terms needed to go between the columns of $A$ and thsoe of $Q$, i.e.
$$ R = \begin{bmatrix}
\innerprod{\bu_1,\mathbf{q}_1} & \innerprod{\bu_2,\mathbf{q}_2} & \ldots & \innerprod{\bu_n,\mathbf{q}_n} \\
0 & \innerprod{\bu_2,\mathbf{q}_2} & \ldots & \innerprod{\bu_n,\mathbf{q}_n} \\
\vdots &\vdots & \ddots & \vdots \\
0 & 0 & \ldots & \innerprod{\bu_n,\mathbf{q}_n} 
\end{bmatrix}.$$
\end{frame}

\begin{frame}
Why use $QR$-decomposition? 

\vspace{3cm}

\end{frame}



\section{Differentiation}

\begin{frame}{Derivative}
Recall the definition of the derivative:
\begin{definition}
A function $f:(a,b) \to \R$ is \emph{differentiable} at $x\in (a,b)$ if
\begin{equation*}
    L := \lim_{h \to 0} \frac{f(x+h) - f(x)}{h}
\end{equation*}
exists. $L$ is the \emph{derivative} of $f$ at $x$, denoted $L= f'(x)$. If $f$ is differentiable at every $x \in (a,b)$, we say $f$ is \emph{differentiable}.
\end{definition}
\end{frame}


\begin{frame}
\begin{exampleblock}{Proposition}
The following are key rules for differentiation:
\begin{enumerate}
    \item If $f$ is differentiable at $x$, then it is continuous at $x$.
    \item The derivative of a constant function is zero.
    \item If $f$ and $g$ are differentiable at $x$, then so is $f+g$ with $(f+g)'(x) = f'(x) + g'(x)$.
    \item Product rule: If $f$ and $g$ are differentiable at $x$, then so is $fg$ with $$(fg)'(x) = f'(x) g(x) + f(x) g'(x).$$
    \item Quotient rule: If $f$ and $g$ are differentiable at $x$ and $g(x) \neq 0$, then so is $f/g$ with $$(f/g)'(x) = \frac{f'(x) g(x) - f(x) g'(x)}{g^2(x)}.$$
    \item Chain rule: If $f$ is differentiable at $x$ and $g$ is differentiable at $f(x)$, then so is $g \circ f$ with $$(g \circ f)'(x)= g'(f(x))f'(x).$$
\end{enumerate}
\end{exampleblock}
\end{frame}

\begin{frame}
\begin{alertblock}{Lemma}
If $f:(a,b) \to \R$ is differentiable on $(a,b)$ and achieves a (local) maximum or (local) minimum at $c \in (a,b)$, then $f'(c) = 0$.
\end{alertblock}
\begin{block}{Proof}
\vspace{3cm}
\end{block}
\end{frame}

\begin{frame}
\begin{block}{Proof continued}
\vspace{6cm}
\end{block}
\end{frame}

\begin{frame}
Next we recall a theorem we proved in greater generality when we discussed compactness in the topology section:
\begin{theorem}[Extreme value theorem]
If $f:[a,b] \to \R$ is continuous, then $f$ attains a maximum and a minimum, i.e. there exists $c,d \in [a,b]$ such that
$$ f(c) \leq f(x) \leq f(d) \quad \forall x \in [a,b] .$$
\end{theorem}

This theorem is used to prove the following important result:

\begin{theorem}[Mean value theorem]
If $f$ is continuous on $[a,b]$ and differentiable on $(a,b)$, then there exists $c \in (a,b)$ such that 
$$f(b) - f(a) = f'(c)(b-a) .$$
\end{theorem}
\end{frame}

\begin{frame}
\begin{proof}
\vspace{6cm}
\end{proof}
\end{frame}


\begin{frame}
\begin{corollary}
If $f:(a,b) \to \R$ is differentiable and has a bounded derivative (i.e. $|f'(x)| \leq M$ for some $M >0$ and for all $x \in (a,b)$), then $f$ is Lipschitz.
\end{corollary}
\begin{proof}
\vspace{4cm}
\end{proof}
\end{frame}



\begin{frame}{l'H\^{o}pital's rule}
\begin{theorem}[l'H\^{o}pital's rule]
If $f,g$ are differentiable on $(a,b)$, where $a,b$ may be $\pm \infty$, and $\lim_{x \to b} f(x) = 0 = \lim_{x \to b} g(x)$, or both limits equal $\pm \infty$, then 
$$ \lim_{x \to b} \frac{f'(x)}{g'(x)} = L$$
implies
$$ \lim_{x \to b} \frac{f(x)}{g(x)} = L$$
\end{theorem}
\end{frame}



\begin{frame}
\begin{example}
$$ \lim_{x\to 0} \frac{5^x - 2^x}{x^2-x} \qquad \qquad \qquad \qquad \qquad \qquad \qquad \qquad \qquad \qquad \qquad \qquad \qquad \qquad $$
\vspace{2cm}
$$ \lim_{x\to -\infty} x e^x \qquad \qquad \qquad \qquad \qquad \qquad \qquad \qquad \qquad \qquad \qquad \qquad \qquad \qquad \qquad $$
\vspace{2cm}
\end{example}
\end{frame}


\begin{frame}{Higher order derivatives}
\begin{definition}
We define higher-order derivatives inductively as $f^{(r)}(x) = (f^{(r-1)})'(x)$. If $f^{(r)}$ exists (at $x$), we say that $f$ is $r^\text{th}$-order differentiable (at $x$). 
\end{definition}

\vspace{1em}

\begin{definition}
If $f^{(r)}$ exists for all $r \in \N$ and for all $x \in (a,b)$, then we say $f$ is infinitely differentiable or \emph{smooth}. We denote this $f \in C^\infty$.
\end{definition}
\end{frame}

\begin{frame}{Smoothness classes}
\begin{definition}
If $f$ is differentiable and its derivative $f'(x)$ is continuous, we say that $f$ is \emph{continuously differentiable}, and that $f \in C^1$. If $f^{(r)}$ exists and is continuous, we say that $f \in C^r$. If $f$ is continuous, we say $f \in C^0$.
\end{definition}

\vspace{1em}

Since differentiability implies continuity, we have $C^\infty \subset \cdots \subset C^2 \subset C^1 \subset C^0$.

\end{frame}




\begin{frame}
\begin{example}
\begin{itemize}
\item The function $f(x) = |x|$ is $C^0$ but not $C^1$. 
\item The function $f(x) = x|x|$ is $C^1$ but not $C^2$.
\item $f(x) = e^x$ and $f(x) = x$ are smooth functions, i.e. in $C^\infty$.
\end{itemize}
\end{example}
\vspace{4cm}
\end{frame}

\section{Integration}


\begin{frame}{Riemann integration}
\begin{definition}[Riemann sum]
Let $f\colon [a,b] \to \R$ be a function. We call a set of points $P=\{x_0, \ldots, x_n\}\subseteq [a,b]$ a \emph{partition} of $[a,b]$ if the following holds
\begin{equation*}
    a = x_0 \leq x_1\leq \ldots \leq x_{n-1} \leq x_n = b.
\end{equation*}
We call the largest sub-interval of the partition $P$ the \emph{mesh} of $P$, denoted $|P|$, i.e.
$$ |P| = \max_{i=1,\ldots,n} ( x_i - x_{i-1} ).$$

\end{definition}

\end{frame}


\begin{frame}
\begin{block}{Definition continued (Riemann sum)}
Given a partition $P=\{x_0, \ldots, x_n\}\subseteq [a,b]$ of $[a,b]$ and a set of points $T=\{t_1,\ldots, t_n\}\subseteq [a,b]$ such that $t_i \in [x_{i-1},x_i]$ for $i=1,\ldots, n$, we define the \emph{Riemann sum} $R(f,P,T)$ corresponding to $f,P,T$ as 
\begin{equation*}
    R(f,P,T) = \sum_{i=1}^n f(t_i) (x_i-x_{i-1}) := \sum_{i=1}^n f(t_i) \Delta x_i,
\end{equation*}
where we used $ \Delta x_i = x_i-x_{i-1}$. 

\end{block}
\end{frame}




\begin{frame}
The idea is to define the Riemann integral as the ``limit'' of the Riemann sums over all partition such that their mesh is becoming arbitrarily small:

\vspace{1em}

\begin{definition}[Riemann integrable]
A function $f\colon [a,b] \to \R$ is called \emph{Riemann integrable} if there exists $I\in \R$ such that for all $\epsilon>0$, there exists a $\delta>0$ such that for any partition $P = \{x_0,\ldots,x_n \}$ of $[a,b]$ with $\vert P \vert <\delta$ and set of points $T=\{t_1,\ldots, t_n\}\subseteq [a,b]$ such that $t_i \in [x_{i-1},x_i]$ for $i=1,\ldots,n$ we have $\vert R(f,P,T) - I \vert < \epsilon$. 

We say that $I$ is the Riemann integral of $f$, denoted $I = \int_a^b f(x) \mathrm{d}x$.
\end{definition}
\vspace{1em}

If $f$ is Riemann integrable, then $I$ is unique.
\end{frame}




\begin{frame}
Let $\mathcal{R}([a,b])$ denote the set of functions that are Riemann integrable on $[a,b]$.

\vspace{2em}

\begin{theorem}
Riemann integration is linear, i.e. if $f,g \in \mathcal{R}([a,b])$ and $c \in \R$, then $f+ cg \in \mathcal{R}([a,b])$.
\end{theorem}
\end{frame}


\begin{frame}
\begin{block}{Proof}
\vspace{6cm}
\end{block}
\end{frame}


\begin{frame}
\begin{block}{Proof continued}
\vspace{6cm}
\end{block}
\end{frame}

\begin{frame}
\begin{exampleblock}{Proposition (Rules for integration on $[a,b]$)}
\begin{enumerate}
    \item The constant function $f(x) = c$ is integrable and its integral is $c(b-a)$.
    \item If f is Riemann integrable, then it is bounded.
    \item If $f,g \in \mathcal{R}([a,b])$ and $f(x) \leq g(x)$ for all $x \in [a,b]$, then
    $$ \int_a^b f(x) dx \leq  \int_a^b g(x) dx.$$
    \item If $f \in \mathcal{R}([a,b])$ and $g:[c,d] \to [a,b]$ is a continuously differentiable bijection with $g'>0$, then 
     $$ \int_a^b f(y) dy =  \int_c^d f(g(x)) g'(x) dx.$$
     \item If $f,g : [a,b] \to \R$ are differentiable and $f',g' \in \mathcal{R}([a,b])$, then 
     $$ \int_a^b f(x) g'(x) dx =  f(b) g(b) - f(a) g(a) - \int_a^b f'(x) g(x) dx.$$
\end{enumerate}
\end{exampleblock}
\end{frame}

\begin{frame}
\begin{theorem}[Fundamental Theorem of Calculus]
\textbf{\emph{First part:}}\\
If $f:[a,b] \to \R$ is Riemann integrable then its indefinite integral
$$ F(x) = \int_a^x f(t) dt$$
is a continuous function of $x$. In addition, the derivative of $F$ exists and $F'(x) = f(x)$ at all $x \in [a,b]$ where $f$ is continuous.

\vspace{1em}

\textbf{\emph{Second part:}}\\
Let $f:[a,b]\to \R$ and let $F$ be a continuous function on $[a,b]$ with antiderivative $f$ on $(a,b)$, i.e. $F'(x) = f(x)$. Then if $F$ is Riemann integrable on $[a,b]$,
$$ \int_a^b f(x) dx = F(b) - F(a).$$
\end{theorem}

\end{frame}

\begin{frame}{Drawbacks of the Riemann integral}
\begin{itemize}
      \setlength\itemsep{1em}
\item Riemann integration has many nice properies, but it also has some drawbacks
\item To see this, we first introduce a nice alternative characterization of Riemann integrability
\item  Instead of looking at all the Riemann sums, we can restrict our attention to two special forms of the sum
\end{itemize}

\end{frame}

\begin{frame}
\begin{definition}
Given a function $f\colon [a,b] \to \R$ and a partition $P = \{x_0,\ldots, x_n\}$ of $[a,b]$, we define the \emph{lower} and \emph{upper sum} of $f$ via \begin{align*}
    L(f,P) = \sum_{i=1}^n m_i \Delta x_i, \qquad U(f,P) =\sum_{i=1}^n M_i \Delta x_i,
\end{align*}
where $m_i = \inf \{f(t) \colon t\in [x_{i-1}, x_i]\}$ and $M_i = \sup \{f(t) \colon t\in [x_{i-1}, x_i]\}$. We define the \emph{lower} and \emph{upper integral} of $f$ to be
\begin{align*}
    \underline{I} = \sup_P L(f,P), \qquad \overline{I} = \inf_P U(f,P).
\end{align*}
\end{definition}
\end{frame}

\begin{frame}
Since $f$ is defined on a compact set, it will be bounded and hence the lower and upper integral exist and are finite.
\vspace{1em}

One can think of the lower and upper integral as lower and upper bounds for the Riemann integral. 
\vspace{2em}

\begin{theorem}
Let $f\colon [a,b] \to \R$ be a function. Then $f$ is Riemann integrable if and only if $\underline{I} = \overline{I}$ and we have $\underline{I} = \overline{I} = I$.
\end{theorem}
\end{frame}

\begin{frame}{A function that is not Riemann integrable}
\begin{equation*}
    f \colon [0,1] \to \R \colon x\mapsto \begin{cases} 0 &\text{ if } x\in \Q,\\
    1 &\text{ otherwise}.
    \end{cases}
\end{equation*}
\vspace{1em}

Is this function Riemann integrable? Should it be integrable?
\vspace{3cm}
\end{frame}


\section{The End}


\begin{frame}{References}

Charles C. Pugh (2015). \textit{Real Mathematical Analysis}. Undergraduate Texts in Mathematics. \href{https://link-springer-com.myaccess.library.utoronto.ca/book/10.1007/978-3-319-17771-7}{https://link-springer-com.myaccess.library.utoronto.ca/book/10.1007/978-3-319-17771-7}

\end{frame}




\end{document}