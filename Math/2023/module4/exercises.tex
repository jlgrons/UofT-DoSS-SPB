\documentclass{article}
\usepackage[utf8]{inputenc}
\usepackage[margin=1in]{geometry}
\usepackage{hyperref}
\usepackage{setspace}
\pagenumbering{arabic}
\usepackage{graphicx}
\usepackage[dvipsnames]{xcolor}
\usepackage{fancyhdr} 
\usepackage{amsmath, amsfonts, amsthm, amssymb}
\usepackage{bbm}
\usepackage{nth}
\usepackage{dsfont}

\hypersetup{
  colorlinks   = true, %Colours links instead of boxes
  urlcolor     = black, %Colour for external hyperlinks
  linkcolor    = black, %Colour of internal links
  citecolor   = black %Colour of citations
}

\allowdisplaybreaks % fixes align environment weird spacing on page
\setlength{\parindent}{0cm}


%For references using Lemma, Theorem, etc, use \cref
\usepackage[nameinlink,capitalize,sort]{cleveref}

% theorems
\newtheorem{theorem}{Theorem}[section]
\newtheorem{lemma}[theorem]{Lemma}
\newtheorem{definition}[theorem]{Definition}
\newtheorem{proposition}[theorem]{Proposition}
\newtheorem{example}[theorem]{Example}
\newtheorem{corollary}[theorem]{Corollary}
%theoremstyle{plain} %boldface title, italicized body. Commonly used in theorems, lemmas, corollaries, propositions and conjectures.
%\theoremstyle{definition} %boldface title, Roman body. Commonly used in definitions, conditions, problems and examples
\theoremstyle{remark} %italicized title, Roman body. Commonly used in remarks, notes, annotations, claims, cases, acknowledgments and conclusions.
\newtheorem{exercise}[theorem]{Exercise}

\newenvironment{solution}
  {\renewcommand\qedsymbol{$\blacksquare$}\begin{proof}[Solution]}
  {\end{proof}}

% weird hack to get rid of dot:
\usepackage{xpatch}
\makeatletter
\AtBeginDocument{\xpatchcmd{\@thm}{\thm@headpunct{.}}{\thm@headpunct{}}{}{}}



% lin alg
\newcommand{\bu}{{\mathbf{u}}}
\newcommand{\bv}{{\mathbf{v}}}
\newcommand{\bw}{{\mathbf{w}}}
\newcommand{\bx}{\mathbf{x}}
\newcommand{\zerovec}{{\mathbf{0}}}

% other useful stuff
\newcommand{\Id}{{\mathds{1}}}
\newcommand{\R}{{\mathbb{R}}}
\newcommand{\C}{{\mathbb{C}}}
\newcommand{\Z}{{\mathbb{Z}}}
\newcommand{\N}{{\mathbb{N}}}
\newcommand{\Q}{{\mathbb{Q}}}
\newcommand{\F}{{\mathbb{F}}}
\newcommand{\cL}{{\mathcal{L}}}
\newcommand{\cP}{\mathcal{P}}
\newcommand{\cT}{\mathcal{T}}
\newcommand{\inv}{{-1}}

%\DeclareMathOperator{\dim}{dim}
\DeclareMathOperator{\range}{range}
\DeclareMathOperator{\rank}{rank}
\DeclareMathOperator{\nullity}{nullity}


\begin{document}
\begin{center}
\Large{Module 4: Metric Spaces and Sequences II}
\end{center}

1. Show that the infinite intersection of open sets may not be open and that the infinite union of closed sets may not be closed.
\vspace{9.5cm} % delete this

% \begin{proof}  
% ADD your content 
% \end{proof}  


2.  Find the closure, interior, and boundary of the following sets using Euclidean distance:
    \begin{enumerate}
        \item[(i)] $\{(x,y)\in \R^2 : y < x^2 \} \subseteq \R^2$
        \item[(ii)]  $[0,1)\times[0,1) \subseteq \R^2$
        \item[(iii)] $\{0 \} \cup  \{1/n \colon n \in \N\} \subseteq \R$ 
    \end{enumerate}

\vspace{10cm} % delete this

% \begin{proof}  
% ADD your content 
% \end{proof}  



3. Prove the following: Let $(x_n)_{n\in\N}$ be a sequence in a metric space $(X,d)$ that converges to a point $x \in X$. Then $x$ is unique.

\vspace{11cm} % delete this

% \begin{proof}  
% ADD your content 
% \end{proof}  


4. Let $(x_n)_{n \in \N}$ and $(y_n)_{n \in \N}$ be sequences in $\R$ such that $x_n \to x$ and and $y_n \to y$, with $\alpha, x,y, \in \R$. 
    \begin{enumerate}
        \item[(i)] Show that $\alpha \, x_n \to \alpha \, x$.
        \item[(i)] Show that $x_n + y_n \to x + y$.
    \end{enumerate}

\vspace{11cm} % delete this

% \begin{proof}  
% ADD your content 
% \end{proof}  



5. Show that discrete metric spaces (i.e. those with the metric defined as define $d\colon X \times X \to \R$ by $d(x,x) = 0$ and $d(x,y)=1$ for $x\neq y \in X$) are complete. 
\vspace{11cm} % delete this

% \begin{proof}  
% ADD your content 
% \end{proof}  



% 5. Let $(X,d_X)$ and $(Y,d_Y)$ be metric spaces and let $f:X\to Y$. Prove that
% $$f \text{ is Lipschitz continuous } \Rightarrow f \text{ is uniformly continuous } \Rightarrow f \text{ is continuous}.$$
% Provide examples to show that the other directions do not hold.

% \vspace{21cm} % delete this

% \begin{proof}  
% ADD your content 
% \end{proof}  



% 6. Show that the function $f(x) = \frac{1}{2} \left(x + \frac{5}{x} \right)$ has a unique fixed point on $(0,\infty)$. What is it? (Hint: you will have to restrict the interval.)
% \vspace{11cm} % delete this

% \begin{proof}  
% ADD your content 
% \end{proof}  



\end{document}


