\documentclass [aspectratio=169]{beamer}
\usetheme{Boadilla}
\usepackage{textpos} % package for the positioning
\usepackage[]{graphicx}
\usepackage{graphicx}
\usepackage{float}
\usepackage{hyperref}
\usepackage{caption}
\usepackage{subcaption}
\usepackage{algorithm,algpseudocode}
\usepackage{amsmath, amsfonts, amsthm, amssymb}
\usepackage{dsfont}
\usepackage[export]{adjustbox}
\usepackage{tikz}
\usetikzlibrary{positioning}
\usetikzlibrary{arrows, shapes, decorations, automata, backgrounds, fit, petri, calc}
\setbeamertemplate{itemize items}[circle]
\setbeamertemplate{enumerate items}[circle]
\usepackage{tikz}
\usepackage{accents}

\newcommand*{\logofont}{\fontfamily{phv}\selectfont}
\definecolor{uoftblue}{RGB}{0,42,92} % official blue color for uoft
\definecolor{deptgreen}{RGB}{114,192,148} 
\definecolor{deptoran}{RGB}{252,103,63} 

\hypersetup{
  colorlinks   = true, %Colours links instead of boxes
  urlcolor     = uoftblue, %Colour for external hyperlinks
  linkcolor    = black, %Colour of internal links
  citecolor   = black %Colour of citations
}

% lin alg
\newcommand{\bu}{{\mathbf{u}}}
\newcommand{\bv}{{\mathbf{v}}}
\newcommand{\bw}{{\mathbf{w}}}
\newcommand{\bx}{\mathbf{x}}
\newcommand{\zerovec}{{\mathbf{0}}}

% other useful stuff
\newcommand{\Id}{{\mathds{1}}}
\newcommand{\R}{{\mathbb{R}}}
\newcommand{\C}{{\mathbb{C}}}
\newcommand{\F}{{\mathbb{F}}}
\newcommand{\Z}{{\mathbb{Z}}}
\newcommand{\N}{{\mathbb{N}}}
\newcommand{\Q}{{\mathbb{Q}}}
\newcommand{\cP}{\mathcal{P}}

% set theory
\newcommand{\interior}{\accentset{\circ}}

\newtheorem{proposition}[theorem]{Proposition}


\beamertemplatenavigationsymbolsempty

% block
% example block
% alert block


\title[]{Module 3: Metric Spaces and Sequences I \\ {\large Operational math bootcamp}\\ \includegraphics[width=8cm]{dept_logo.png}\vspace{-1em}}
\author[]{Emma Kroell}
\institute[]{University of Toronto}
\date{July 14, 2022}

% set color
\setbeamercolor{title in head/foot}{bg=white}
\setbeamercolor{author in head/foot}{bg=white}
\setbeamercolor{date in head/foot}{fg=uoftblue}
\setbeamercolor{date in head/foot}{bg=white}
\setbeamercolor{title}{fg=uoftblue}
\setbeamerfont{title}{series=\bfseries}
\setbeamercolor{frametitle}{fg=uoftblue}
\setbeamerfont{frametitle}{series=\bfseries}
\setbeamercolor*{item}{fg=uoftblue}
\setbeamercolor{block title}{bg=uoftblue}
\setbeamercolor{block title}{fg=white}
\setbeamercolor{block body}{bg=uoftblue!9!white}
\setbeamercolor{block title example}{bg=deptgreen}
\setbeamercolor{block title example}{fg=white}
\setbeamercolor{block body example}{bg=deptgreen!13!white}
\setbeamercolor{block title alerted}{bg=deptoran}
\setbeamercolor{block title alerted}{fg=white}
\setbeamercolor{block body alerted}{bg=deptoran!10!white}

\setbeamercolor{coloredboxstuff}{fg=yellow,bg=uoftblue!10!white}


% set logo at non-title pages
\logo{\includegraphics[height=0.8cm]{dept_logo.png}\vspace*{-.045\paperheight}\hspace*{.78\paperwidth}}

\AtBeginSection[]{
  \begin{frame}
  \vfill
  \centering
  \begin{beamercolorbox}[sep=8pt,center,shadow=true,rounded=true]{coloredboxstuff}
    \usebeamerfont{title}\insertsectionhead\par%
  \end{beamercolorbox}
  \vfill
  \end{frame}
}


% set margin
\setbeamersize{text margin left=10mm,text margin right=10mm}

\begin{document}
{
\setbeamertemplate{logo}{}
\begin{frame}
    %\vspace{0.5in}
    \titlepage
    %\begin{textblock*}{10cm}(3.5cm,-7.5cm)
      %  \includegraphics[width=8cm]{dept_logo.png}
    %\end{textblock*}
\end{frame}
}

\begin{frame}{Outline}
    \begin{itemize}
      \setlength\itemsep{1em}
    	\item More on set theory
	\item Cardinality of sets
	\item Metrics and norms
    \end{itemize}
\end{frame}

\begin{frame}{Recall}
\begin{definition}[Image and pre-image]
Let $f:X \to Y$ and $A \subseteq X$ and $B \subseteq Y$. 
\begin{itemize}
\item The \emph{image} of $f$ is the set $f(A) := \{f(x): x \in A \}$.
\item The \emph{pre-image} of $f$ is the set $f^{-1}(B) := \{x: f(x) \in B \}$.
\end{itemize}
\end{definition}


\begin{definition}[Surjective, injective and bijective]
Let $f:X \to Y$, where $X$ and $Y$ are sets. Then
\begin{itemize}
    \item $f$ is \emph{injective} if $x_1 \neq x_2$ implies $f(x_1) \neq f(x_2)$
    \item $f$ is \emph{surjective} if for every $y \in Y$, there exists an $x \in X$ such that $y = f(x)$
    \item $f$ is \emph{bijective} if it is both injective and bijective
\end{itemize}
\end{definition}

\end{frame}

\begin{frame}
\begin{proposition}
Let $f: X \to Y$ and $A \subseteq X$. Prove that $A \subseteq f^{-1}(f(A))$, with equality iff $f$ is injective. 
\end{proposition}
\begin{proof}

\vspace{4cm}
\end{proof}

\end{frame}


\begin{frame}{Cardinality}
Intuitively, the \emph{cardinality} of a set $A$, denoted $|A|$, is the number of elements in the set. For sets with only a finite number of elements, this intuition is correct. We call a set with finitely many elements finite. 

\vspace{1em}

We say that the empty set has cardinality 0 and is finite.



\end{frame}


\begin{frame}

\begin{proposition}
 If $X$ is finite set of cardinality $n$, then the cardinality of $\cP(X)$ is $2^n$.
\end{proposition}

\textit{Proof.}
\vspace{4cm}

\end{frame}

\begin{frame}


\end{frame}


\begin{frame}
\begin{definition}
Two sets $A$ and $B$ have same cardinality, $|A| = |B|$, if there exists bijection $f:A \to B$.
\end{definition}

\begin{example}
Which is bigger, $\N$ or $\N_0$? \\
\vspace{4cm}
\end{example}
\end{frame}


\begin{frame}{Cantor-Schr\"{o}der-Bernstein}
\begin{definition}
We say that the cardinality of a set $A$ is less than the cardinality of a set $B$, denoted $|A| \leq |B|$ if there exists an injection $f:A \to B$.
\end{definition}

\vspace{1em}

\begin{alertblock}{Theorem (Cantor-Bernstein)}
Let $A$, $B$, be sets. If $|A| \leq |B|$ and $|B| \leq |A|$, then $|A| = |B|$.
\end{alertblock}

\vspace{1em}


\begin{example}
$|\N| = |\N \times \N|$
\end{example}

\end{frame}


\begin{frame}
Proof that $|\N| = |\N \times \N|$:
\vspace{6cm}
\end{frame}

\begin{frame}
\begin{definition}
Let $A$ be a set. 
\vspace{0.25em}
\begin{enumerate}
\setlength\itemsep{0.5em}
\item $A$ is \emph{finite} if there exists an $n \in \N$ and a bijection $f:\{1,\ldots,n\} \to A$
\item $A$ is \emph{countably infinite} if there exists a bijection $f:\N\to A$
\item $A$ is \emph{countable} if it is finite or countably infinite
\item $A$ is \emph{uncountable} otherwise
\end{enumerate}
\end{definition}
\end{frame}

\begin{frame}
\begin{example}
The rational numbers are countable, and in fact $|\Q| = |\N|$.
\end{example}

\vspace{0.5em}

\textit{Proof.}
First we show $|\N| \leq |\Q^+|$.
\vspace{4.5cm}

\end{frame}

\begin{frame}
Next, we show that $|\Q^+| \leq |\N \times \N|$.

\vspace{5.5cm}


\end{frame}

\begin{frame}
We can extend this to $\Q$ as follows:
\vspace{5.5cm}


\end{frame}

\begin{frame}
\begin{alertblock}{Theorem}
The cardinality of $\N$ is smaller than that of $(0,1)$.
\end{alertblock}
\begin{proof}
First, we show that there is an injective map from $\N$ to $(0, 1)$.

\vspace{3em} 

Next, we show that there is no surjective map from $\N$ to (0, 1). We use the fact that every number $r \in (0,1)$ has a binary expansion of the form $r=0.\sigma_1\sigma_2\sigma_3\ldots$ where $\sigma_i \in \{0, 1\}$, $i \in \N$.
\end{proof}
\end{frame}

\begin{frame}
\begin{proof}
Now we suppose in order to derive a contradiction that there does exist a surjective map $f$ from $\N$ to (0, 1)., i.e. for $n \in \N$ we have $f(n) = 0.\sigma_1(n)\sigma_2(n)\sigma_3(n)\ldots$. This means we can list out the binary expansions, for example like
\begin{align*}
f(1)= & 0.{0}0000000\ldots \\
f(2)=& 0.1{1}11111111\ldots\\
f(3)=& 0.01{0}1010101\ldots  \\
f(4)= & 0.101{0}101010\ldots  \\
& 
\end{align*}

We will construct a number $\tilde r \in (0,1)$ that is not in the image of $f$. 
\end{proof}
\end{frame}


\begin{frame}
\begin{proof}
Define $\tilde r = 0.\tilde\sigma_1 \tilde\sigma_2 \ldots$, where we define the $n$th entry of $\tilde r$ to be the the opposite of the  $n$th entry of the $n$th item in our list:
\begin{equation*}
    \tilde\sigma_n = \begin{cases} 1 & \text{if } \sigma_n(n) = 0, \\
    0 & \text{if }  \sigma_n(n) = 1.
    \end{cases}
\end{equation*}
Then $\tilde r$ differs from $f(n)$ at least in the $n$th digit of its binary expansion for all $n\in \N$. Hence, $\tilde r\not\in f(\N)$, which is a contradiction to $f$ being surjective. This technique is often referred to as Cantor's diagonal argument. 
\end{proof}
\end{frame}

\begin{frame}
\begin{exampleblock}{Proposition}
(0,1) and $\R$ have the same cardinality. 
\end{exampleblock}
\begin{proof}
\vspace{1.5cm}
%The map $f:\R \to (0,1)$ defined by $x \mapsto \frac{1}{\pi} \left( \arctan(x) + \frac{\pi}{2} \right)$ is a bijection.
\end{proof}

We have shown that there are different sizes of infinity, as the cardinality of $\N$ is infinite but still smaller than that of $\R$ or $(0,1)$. In fact, we have
$$ |\N| \quad |\N_0| \quad  |\Z|  \quad |\Q|  \quad |\R|.$$
Because of this, there are special symbols for these two cardinalities: The cardinality of $\N$ is denoted $\aleph_0$, while the cardinality of $\R$ is denoted $\mathfrak{c}$. 

\end{frame}

\section{Metric Spaces}

\begin{frame}
\begin{definition}[Metric]
A \emph{metric} on a set $X$ is a function $d:X \times X \to \R$ that satisfies:
\vspace{0.5em}
\begin{enumerate}
 \setlength\itemsep{0.75em}
    \item[(a)] Positive definiteness: 
    \item[(b)] Symmetry: 
    \item[(c)] Triangle inequality: 
\end{enumerate}
\vspace{0.5em}
A set together with a metric is called a metric space.
\end{definition}
\end{frame}

\begin{frame}
\begin{exampleblock}{Example ($\R^n$ with the Euclidean distance)}
\vspace{4cm}
\end{exampleblock}
\end{frame}


\begin{frame}
\begin{definition}[Norm]
A \emph{norm} on an $\F$-vector space $E$ is a function $\|\cdot\|:E \to \R$ that satisfies:
\vspace{0.5em}
\begin{enumerate}
 \setlength\itemsep{0.5em}
    \item[(a)] Positive definiteness: 
    \item[(b)] Homogeneity: 
    \item[(c)] Triangle inequality: 
\end{enumerate}
\vspace{0.5em}
A vector space with a norm is called a normed space. A normed space is a metric space using the metric $d(x,y) = \| x-y \|$.
\end{definition}

\end{frame}


\begin{frame}
\begin{example}[$p$-norm on $\R^n$]
The $p$-norm is defined for $p \geq 1$ for a vector $x = (x_1, \ldots, x_n) \in \R^n$ as

\vspace{2.5cm}

The infinity norm is the limit of the $p$-norm as $p \to \infty$, defined as
\vspace{2cm}

\end{example}
\end{frame}


\begin{frame}
\begin{example}[$p$-norm on  {$C([0,1];\R)$} ]
If we look at the space of continuous functions $C([0,1];\R)$, the $p$-norm is 

\vspace{2.5cm}

and the $\infty-$norm (or sup norm) is 
\vspace{2cm}

\end{example}
\end{frame}

\begin{frame}
\begin{definition}
A subset $A$ of a metric space $(X,d)$ is \emph{bounded} if there exists $M>0$ such that $d(x,y) < M$ for all $x,y \in A$. 
\end{definition}
\end{frame}

\begin{frame}
\begin{definition}
Let $(X,d)$ be a metric space. We define the \emph{open ball} centred at a point $x_0 \in X$ of radius $r > 0$ as
\begin{equation*}
    B_r(x_0) := \{x \in X : d(x,x_0) < r_0 \}.
\end{equation*}
\end{definition}

\vspace{1cm}

\begin{example}
In $\R$ with the usual norm (absolute value), open balls are symmetric open intervals, \\
i.e. 
\vspace{1em}
\end{example}
\end{frame}


\begin{frame}{Example: Open ball in $\R^2$ with different metrics}

\begin{figure}[h]
    \centering
    \subfloat[1-norm (taxicab metric)]{\begin{tikzpicture}[scale=0.4]
    \draw[help lines, color=gray!30, dashed] (-4.9,-4.9) grid (4.9,4.9);
    \draw[->,ultra thick] (-5,0)--(5,0) node[right]{$x_1$};
    \draw[->,ultra thick] (0,-5)--(0,5) node[above]{$x_2$};
    \node at (0,3) [circle,fill,inner sep=1.5pt]{};
    \node at (3,0) [circle,fill,inner sep=1.5pt]{};
    \node at (3.95,-0.5) {$(r,0)$};
    \node at (1,3.5) {$(0,r)$};
    
%    \draw[densely dotted, thick, color=red] (3,0)--(0,3);
%    \draw[densely dotted, thick, color=red] (-3,0)--(0,-3);
%    \draw[densely dotted, thick, color=red] (-3,0)--(0,3);
%    \draw[densely dotted, thick, color=red] (3,0)--(0,-3);

    \end{tikzpicture}}
    \subfloat[2-norm (Euclidean metric)]{\begin{tikzpicture}[scale=0.4]
    \draw[help lines, color=gray!30, dashed] (-4.9,-4.9) grid (4.9,4.9);
    \draw[->,ultra thick] (-5,0)--(5,0) node[right]{$x_1$};
    \draw[->,ultra thick] (0,-5)--(0,5) node[above]{$x_2$};
    \node at (0,3) [circle,fill,inner sep=1.5pt]{};
    \node at (3,0) [circle,fill,inner sep=1.5pt]{};
    \node at (3.95,-0.5) {$(r,0)$};
    \node at (1,3.5) {$(0,r)$};
    
%    \draw[densely dotted,thick, color = red] (0,0) circle [radius=3];

    \end{tikzpicture}}
    \subfloat[$\infty$-norm]{\begin{tikzpicture}[scale=0.4]
    \draw[help lines, color=gray!30, dashed] (-4.9,-4.9) grid (4.9,4.9);
    \draw[->,ultra thick] (-5,0)--(5,0) node[right]{$x_1$};
    \draw[->,ultra thick] (0,-5)--(0,5) node[above]{$x_2$};
    \node at (0,3) [circle,fill,inner sep=1.5pt]{};
    \node at (3,0) [circle,fill,inner sep=1.5pt]{};
    \node at (3.95,-0.5) {$(r,0)$};
    \node at (1,3.5) {$(0,r)$};
    
%    \draw[densely dotted, thick, color=red] (3,-3) --(3,3);
%    \draw[densely dotted, thick, color=red] (3,3) --(-3,3);
%    \draw[densely dotted, thick, color=red] (-3,-3) --(-3,3);
%    \draw[densely dotted, thick, color=red] (3,-3) --(-3,-3);

    \end{tikzpicture}}
    \caption{$B_r(0)$ for different metrics}
\end{figure}
\end{frame}





\begin{frame}{References}

Runde, Volker (2005). \textit{A Taste of Topology}. Universitext.  url:  \href{https://link.springer.com/book/10.1007/0-387-28387-0}{https://link.springer.com/book/10.1007/0-387-28387-0} 

\vspace{1em}

Zwiernik, Piotr (2022). \textit{Lecture notes in Mathematics for Economics and Statistics}. url: \href{http://84.89.132.1/~piotr/docs/RealAnalysisNotes.pdf}{http://84.89.132.1/~piotr/docs/RealAnalysisNotes.pdf} \\

\end{frame}




\end{document}